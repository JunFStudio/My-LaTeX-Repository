%!TEX program = xelatex
\documentclass[cn,hazy,black,10pt,normal]{elegantnote}
\usepackage{hyperref}
\usepackage{amssymb}
\usepackage[version=3]{mhchem}

% font settings
\definecolor{mgreen}{RGB}{0,166,82}
\definecolor{guess}{RGB}{47,79,79}
\newenvironment{guess}{
  \color{guess}}{\newline \color{black}}

% cover settings
\title{《奥数教程》新题解答}

\author{Johnny Tang}
\institute{DEEP Team}

\date{\zhtoday}

% customised commands
\usepackage{ulem}
	\newcommand{\tk}{\uline{\hspace{4em}}}
\DeclareSymbolFont{yh}{OMX}{yhex}{m}{n}
\DeclareMathAccent{\hu}{\mathord}{yh}{"F3}
\newcommand{\xl}[1]{\overrightarrow{#1}}
\newcommand{\nd}[1]{〔#1〕}
\newcommand{\cor}{~\textit{或}~}
\newcommand{\ssb}[1]{\left( #1 \right)}
\newcommand{\sw}[1]{\boxed{\text{解法 #1}} \ }
\newcommand{\buzhou}[1]{$#1^{\circ} \ $}
\newcommand{\R}{\mathbb{R}}
\newcommand{\hlt}[1]{\color{red} #1 \color{black}}
\DeclareMathOperator{\card}{card}

% 行距设置
\setlength{\lineskiplimit}{5pt} %至少宽度
\setlength{\lineskip}{4pt} %正常宽度
\setlength{\normallineskiplimit}{5pt} %正常宽度
\setlength{\normallineskip}{5pt} %正常宽度


\begin{document}

\maketitle

\chapter{代数}

\begin{problem}
	(II,p7,A4)实数$x,y,z,w$满足$x+y+z+w=1$,求$$M=xw+2yw+3xy+3zw+4xz+5yz$$的最大值.
\end{problem}
\begin{solution}
	注意到,
	\begin{align*}
		M &= x(w+y+z)+2y(x+w+z)+3z(x+y+w) \\
		&= x(1-x) + 2y(1-y) + 3z(1-z) \\
		&\leq \dfrac{1}{4} + 2\times \dfrac{1}{4} + 3\times \dfrac{1}{4} = \dfrac{3}{2}
	\end{align*}
	等号在$x=y=z=\dfrac{1}{2},w=-\dfrac{1}{2}$时取到.
\end{solution}

\begin{problem}
	(II,p7,B11)设数$x_1,\cdots ,x_{1991}$满足条件$$|x_1-x_2| + |x_2-x_3| + \cdots + |x_{1990}-x_{1991}|=1991$$
	记$y_k=\dfrac{1}{k}(x_1+ \cdots +x_k),~k=1, \cdots ,1991$.求$$|y_1-y_2|+|y_2-y_3|+\cdots + |y_{1990}-y_{1991}|$$可能取得的最大值.
\end{problem}
\begin{solution}
	对于每一项,有
	\begin{align*}
		|y_k-y_{k+1}| &= \left| \dfrac{1}{k}(x_1+\cdots +x_k)-\dfrac{1}{k+1}(x_1+\cdots +x_{k+1}) \right| \\
		&= \left| \dfrac{1}{k(k+1)}(x_1+ \cdots + x_k-kx_{k+1}) \right| \\
		&\leq \dfrac{1}{k(k+1)}(|x_1-x_2|+2|x_2-x_3|+\cdots +k|x_k-x_{k+1}|)
	\end{align*}
	对$k=1, \cdots ,1990$进行累加,得到
	\begin{align*}
		S_0 \leq &|x_1-x_2|\ssb{ \dfrac{1}{1\times 2}+\dfrac{1}{2\times 3}+\cdots +\dfrac{1}{1990\times 1991} }+2|x_2-x_3|\ssb{\dfrac{1}{2\times 3}+\cdots +\dfrac{1}{1990\times 1991}} + \cdots + \\
		&1990|x_{1990}-x_{1991}|\dfrac{1}{1990 \times 1991} \\
		=& |x_1-x_2|\ssb{1-\dfrac{1}{1991}} + |x_2-x_3|\ssb{1-\dfrac{2}{1991}} + \cdots + |x_{1990}-x_{1991}|\ssb{1-\dfrac{1990}{1991}} \\
		\leq &1991 \times \ssb{1-\dfrac{1}{1991}} = 1990
	\end{align*}
	上述不等式可在$x_1=1991,x_2=\cdots =x_{1991}$时取等.
\end{solution}
\begin{remark}
	在放缩的最后一步,实际上是下列形式:给定$\omega _1+\cdots +\omega _n$为定值,求$\omega _1a_1+\cdots +\omega _na_n$的最大值,其中$a_1 > \cdots > a_n$,只需要把$a_1$的系数调到最大即可.
\end{remark}

\begin{problem}
	(II,p187,A2)设$a=\sqrt{3x+1}+\sqrt{3y+1}+\sqrt{3z+1}$,其中$x+y+z=1,~x,y,z\geq 0$.求$[a]$.
\end{problem}
\begin{solution}
	最大值:$$a \leq 3 \cdot \sqrt{\dfrac{3x+1+3y+1+3z+1}{3}} = 3 \sqrt{2}$$
	最小值:由$0 \leq x,y,z \leq 1$,有$x(1-x) \geq 0$,即$x \geq x^2$,$y,z$同理.故
	\begin{align*}
		a &\geq \sqrt{x^2+2x+1} + \sqrt{y^2+2y+1} + \sqrt{z^2+2z+1} \\
		&= x+1+y+1+z+1 = 4
	\end{align*}
	综上,$4 \leq a <5$,即$[a]=4$.
\end{solution}
\begin{remark}
	对于变量的上下界约束,常常使用类似解答中的思路处理.
\end{remark}

\begin{problem}
	(II,p187,A3)设$a,d \geq 0,~b,c >0$,且$b+c \geq a+d$,则$\dfrac{b}{c+d}+\dfrac{c}{a+b}$的最小值为\tk .
\end{problem}
\begin{solution}
	\begin{guess}
		题目所给条件和结论不太对称,进行一些变换: \\
		由$b+c \geq a+d$,可知$b+c \geq \dfrac{1}{2}(a+b+c+d)$;所求即为$\dfrac{b+c}{c+d}+c\ssb{\dfrac{1}{a+b}-\dfrac{1}{c+d}}$. \\
		于是放缩如下:$$S_0 \geq \dfrac{1}{2}+\dfrac{b+c}{2(a+b)} + c\ssb{\dfrac{1}{a+b}-\dfrac{1}{c+d}}$$
		想要将后面括号中的$\dfrac{1}{c+d}$放缩掉,即将$c$放为$c+d$,也就要求$d\ssb{\dfrac{1}{a+b}-\dfrac{1}{c+d}} \leq 0$. \\
		上式成立的条件是$\ssb{\dfrac{1}{a+b}-\dfrac{1}{c+d}} \leq 0$.很明显,这只是两种情况中的一种.回顾我们一开始做的变形,也可以将所求式子变为$\dfrac{b+c}{a+b} + b\ssb{\dfrac{1}{c+d}-\dfrac{1}{a+b}}$.于是考虑进行分类讨论.
	\end{guess}
	\buzhou{1}当$\ssb{\dfrac{1}{a+b}-\dfrac{1}{c+d}} \leq 0$时,由$b+c \geq \dfrac{1}{2}(a+b+c+d)$,
	\begin{align*}
		S_0 &= \frac{b+c}{c+d}+c\ssb{\frac{1}{a+b}-\frac{1}{c+d}} \\
		&\geq \frac{1}{2}+\frac{b+c}{2(a+b)} + c\ssb{\frac{1}{a+b}-\frac{1}{c+d}} \\
		&\geq \frac{1}{2}+\frac{b+c}{2(a+b)} + (c+d)\ssb{\frac{1}{a+b}-\frac{1}{c+d}} \\
		&= \frac{b+c}{2(a+b)} + \frac{c+d}{a+b} - \frac{1}{2} \\
		&\geq 2\cdot \sqrt{\frac{b+c}{2(a+b)} \cdot \frac{c+d}{a+b}} - \frac{1}{2} = \sqrt{2} - \frac{1}{2}
	\end{align*}
	等号在$(a,b,c,d)=(\sqrt{2}+1,\sqrt{2}-1,2,0)$时取到. \\
	\buzhou{2}当$\ssb{\dfrac{1}{a+b}-\dfrac{1}{c+d}} \geq 0$时,同理可得
	\begin{align*}
		S_0 &= \frac{b+c}{a+b} + b\ssb{\frac{1}{c+d}-\frac{1}{a+b}} \\
		&\geq \frac{1}{2} + \frac{c+d}{2(a+b)} + (a+b)\ssb{\frac{1}{c+d}-\frac{1}{a+b}} \\
		&= \frac{c+d}{2(a+b)} + \frac{a+b}{c+d} - \frac{1}{2} \\
		&\geq 2\cdot \sqrt{\frac{c+d}{2(a+b)} \cdot \frac{a+b}{c+d}} - \frac{1}{2} = \sqrt{2} - \frac{1}{2}
	\end{align*}
	等号在$(a,b,c,d)=(0,2,\sqrt{2}-1,\sqrt{2}+1)$时取到.
\end{solution}

\chapter{几何}

\chapter{组合}

\end{document}
