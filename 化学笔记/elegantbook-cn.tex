\documentclass[lang=cn, zihao=5, color=none]{elegantbook}
\definecolor{structurecolor}{RGB}{32,31,115}
\definecolor{main}{RGB}{70,15,100}
\definecolor{second}{RGB}{255,129,0}
\definecolor{third}{RGB}{13,162,251}
\usepackage{hyperref}
\usepackage[version=4]{mhchem}

% font settings



% watermark settings
%\usepackage{ctex, draftwatermark, everypage}
%	\SetWatermarkText{DEEP Team 讲义模版}
%	\SetWatermarkLightness{0.95}
%	\SetWatermarkScale{0.3}

% customised commands


% cover settings

\title{Johnny化学学习笔记}
\subtitle

\author{Johnny Tang}
\institute{DEEP Team}
\date{April 9, 2023}

\extrainfo{请:相信时间的力量,敬畏概率的准则}


\cover{cover.png}



% 修改标题页的橙色带
\definecolor{customcolor}{RGB}{103,29,128}
\colorlet{coverlinecolor}{customcolor}



\begin{document}

\maketitle

\frontmatter

\mainmatter

\tableofcontents

\chapter{物质及其变化}

\section{无机物分类}


成盐氧化物:
\begin{enumerate}
	\item 酸性氧化物:与碱发生非氧化还原反应生成盐和水.例如,
		$$\ce{CO2 + 2OH- = CO3^{2-} + H2O}$$
		$$\ce{SO3 + 2OH- = SO4^{2-} + H2O}$$
		$$\ce{SO2 + 2OH- = SO3^{2-} + H2O}$$
		分别告诉我们$\ce{CO2},\ce{SO3},\ce{SO2}$是酸性氧化物.注意这里$\ce{C}$和$\ce{S}$不变价.
	\item 碱性氧化物:与酸发生非氧化还原反应生成盐和水.例如,
		$$\ce{CuO + 2H+ = Cu^{2+} + H2O}$$
		$$\ce{Fe2O3 + 6H+ = 2Fe^{3+} + 3H2O}$$
		$$\ce{Fe3O4 + 6H+ = 3Fe^{2+} + 3H2O}$$
		分别告诉我们$\ce{CuO},\ce{Fe2O3},\ce{Fe3O4}$是碱性氧化物.
	\item 两性氧化物:与酸碱均能发生非氧化还原反应生成盐和水.常见的有$\ce{Al2O3},\ce{BeO}$等.例如,
		$$\ce{Al2O3 + 6H+ = 2Al^{3+} + 3H2O}$$
		$$\ce{Al2O3 + 2OH- = 2AlO2- + H2O}$$
\end{enumerate}

氧化物的性质:
\begin{enumerate}
	\item 酸性氧化物:\\
		(1)$\ce{\textit{可溶于水的酸性氧化物} + \textit{水} -> \textit{对应的酸}}$.例如,
		$$\ce{SO3 + H2O = H2SO4} \qquad \ce{CO2 + H2O <=> H2CO3}$$
		这其实就是酸的电离方程式. \\
		(2)$\ce{\textit{酸性氧化物} + \textit{碱} -> \textit{对应含氧酸盐} + \textit{水}}$.例如,
		$$\ce{SO3 + 2NaOH = Na2SO4 + H2O} ~~ i.e. ~~ \ce{SO3 + 2OH- = SO4^{2-} + H2O}$$
		(3)$\ce{\textit{酸性氧化物} + \textit{碱性氧化物} -> \textit{对应含氧酸盐}$.例如,
		$$\ce{CaO + CO2 = CaCO3} \qquad \ce{SO3 + Na2O = Na2SO4}$$
\end{enumerate}


\section{电解质理论}


\end{document}





















