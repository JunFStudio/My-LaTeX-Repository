% !Mode:: "TeX:UTF-8"

%%  可通过增加或减少 setup/format.tex中的
%%  第274行 \setlength{\@title@width}{8cm}中 8cm 这个参数来 控制封面中下划线的长度。

\cheading{泰勒公式与无穷小在物理学中的应用}      % 设置正文的页眉,需要填上对应的毕业年份
\ctitle{泰勒公式与无穷小在物理学中的应用}    % 封面用论文标题,自己可手动断行
% \caffil{管理与经济学部} % 学院名称
% \csubject{工业工程}   % 专业名称
\cgrade{六年一贯衔接一班}            % 年级
\cauthor{Johnny Tang}            % 学生姓名
% \cnumber{3012209017}        % 学生学号
% \csupervisor{杨道箭}        % 导师姓名
% \crank{副教授}              % 导师职称

\cdate{\the\year~年~\the\month~月~\the\day~日}

\cabstract{
在数学分析中有一个重要定理:泰勒公式.该公式将无穷逼近的思想转化为具体的计算方法,从而可以通过忽略相对较小的量近似计算.物理学中也有很多利用高阶无穷小、等价无穷小进行计算的地方.本文收集了部分物理学中与泰勒公式、无穷小计算的部分问题.
}

\ckeywords{泰勒公式;无穷小;学科融合}

\eabstract{
In analysis there is a vital theorem, Taylor's theorem. It gives an approximation of a function by calculating the value around a specific point, therefore we can ignore those relatively smaller amount. We also do approximation using infinitesimal of higher order and equivalent infinitesimal in physics. This thesis concentrates on some related problems. 
}

\ekeywords{Taylor's theorem, infinitesimal, fusion of subjects}

\makecover

\clearpage
