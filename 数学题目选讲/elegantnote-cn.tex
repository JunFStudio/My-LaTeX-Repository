%!TEX program = xelatex
\documentclass[cn,hazy,black,10pt,normal]{elegantnote}
\usepackage{hyperref}
	\hypersetup{
		colorlinks=true,
		linkcolor=cyan,
		filecolor=blue,      
		urlcolor=guess,
		citecolor=green,
	}
\usepackage{amssymb}

% font settings
\definecolor{mgreen}{RGB}{0,166,82}
\definecolor{guess}{RGB}{47,79,79}
\newenvironment{guess}{
  \color{guess}}{\newline \color{black}}

% cover settings
\title{数学题目选讲}

\author{Johnny Tang}
\institute{Chengdu Jiaxiang Foreign Languages School}

\date{\zhtoday}

% customised commands
\usepackage{ulem}
	\newcommand{\tk}{\uline{\hspace{4em}}}
\DeclareSymbolFont{yh}{OMX}{yhex}{m}{n}
\DeclareMathAccent{\hu}{\mathord}{yh}{"F3}
\newcommand{\xl}[1]{\overrightarrow{#1}}
\newcommand{\nd}[1]{〔#1〕}
\newcommand{\cor}{~\textit{或}~}
\newcommand{\ssb}[1]{\left( #1 \right)}
\newcommand{\sw}[1]{\boxed{\text{解法 #1}} \ }
\newcommand{\buzhou}[1]{$#1^{\circ} \ $}
\newcommand{\R}{\mathbb{R}}
\newcommand{\hlt}[1]{\color{red} #1 \color{black}}
\newcommand{\tutor}[1]{\color{guess} \noindent \kaishu{讲解视频:\href{https://www.bilibili.com/video/#1}{#1}} \songti \color{black}}
\DeclareMathOperator{\card}{card}

% 行距设置
\setlength{\lineskiplimit}{5pt} %至少宽度
\setlength{\lineskip}{4pt} %正常宽度
\setlength{\normallineskiplimit}{5pt} %正常宽度
\setlength{\normallineskip}{5pt} %正常宽度


\begin{document}

\maketitle

\part{《奥数教程》}

\chapter{代数}

\section{不等式}

\begin{problem}
	(II,p7,A4)实数$x,y,z,w$满足$x+y+z+w=1$,求$$M=xw+2yw+3xy+3zw+4xz+5yz$$的最大值.
\end{problem}
\begin{solution}
	注意到,
	\begin{align*}
		M &= x(w+y+z)+2y(x+w+z)+3z(x+y+w) \\
		&= x(1-x) + 2y(1-y) + 3z(1-z) \\
		&\leq \dfrac{1}{4} + 2\times \dfrac{1}{4} + 3\times \dfrac{1}{4} = \dfrac{3}{2}
	\end{align*}
	等号在$x=y=z=\dfrac{1}{2},w=-\dfrac{1}{2}$时取到.
\end{solution}

\begin{problem}
	(II,p7,A9)设实数$a,b$满足$a=x_1+x_2+x_3=x_1x_2x_3,~ab=x_1x_2+x_2x_3+x_3x_1$,其中$x_1,x_2,x_3>0$.则$p=\dfrac{a^2+6b+1}{a^2+a}$的最大值.
\end{problem}
\begin{solution}
	由$\dfrac{x_1+x_2+x_3}{3} \geq \sqrt[3]{x_1x_2x_3}$,可得$a \geq 3\sqrt{3}$.由$3(x_1x_2+x_2x_3+x_3x_1) \leq (x_1+x_2+x_3)^2$,可得$3ab \leq a^2$,即$3b \leq a$. \\
	故$$p \leq \frac{a^2+2a+1}{a^2+a} = 1+\frac{1}{a} \leq \frac{9+\sqrt{3}}{9}$$其中,不等式取等条件为$x_1=x_2=x_3=\sqrt{3}$.
\end{solution}

\begin{problem}
	(II,p7,B11)设数$x_1,\cdots ,x_{1991}$满足条件$$|x_1-x_2| + |x_2-x_3| + \cdots + |x_{1990}-x_{1991}|=1991$$
	记$y_k=\dfrac{1}{k}(x_1+ \cdots +x_k),~k=1, \cdots ,1991$.求$$|y_1-y_2|+|y_2-y_3|+\cdots + |y_{1990}-y_{1991}|$$可能取得的最大值.
\end{problem}
\begin{solution}
	对于每一项,有
	\begin{align*}
		|y_k-y_{k+1}| &= \left| \dfrac{1}{k}(x_1+\cdots +x_k)-\dfrac{1}{k+1}(x_1+\cdots +x_{k+1}) \right| \\
		&= \left| \dfrac{1}{k(k+1)}(x_1+ \cdots + x_k-kx_{k+1}) \right| \\
		&\leq \dfrac{1}{k(k+1)}(|x_1-x_2|+2|x_2-x_3|+\cdots +k|x_k-x_{k+1}|)
	\end{align*}
	对$k=1, \cdots ,1990$进行累加,得到
	\begin{align*}
		S_0 \leq &|x_1-x_2|\ssb{ \dfrac{1}{1\times 2}+\dfrac{1}{2\times 3}+\cdots +\dfrac{1}{1990\times 1991} }+2|x_2-x_3|\ssb{\dfrac{1}{2\times 3}+\cdots +\dfrac{1}{1990\times 1991}} + \cdots + \\
		&1990|x_{1990}-x_{1991}|\dfrac{1}{1990 \times 1991} \\
		=& |x_1-x_2|\ssb{1-\dfrac{1}{1991}} + |x_2-x_3|\ssb{1-\dfrac{2}{1991}} + \cdots + |x_{1990}-x_{1991}|\ssb{1-\dfrac{1990}{1991}} \\
		\leq &1991 \times \ssb{1-\dfrac{1}{1991}} = 1990
	\end{align*}
	上述不等式可在$x_1=1991,x_2=\cdots =x_{1991}$时取等.
\end{solution}
\begin{remark}
	在放缩的最后一步,实际上是下列形式:给定$\omega _1+\cdots +\omega _n$为定值,求$\omega _1a_1+\cdots +\omega _na_n$的最大值,其中$a_1 > \cdots > a_n$,只需要把$a_1$的系数调到最大即可.
\end{remark}

\begin{problem}
	(II,p15,A8)证明:不等式$$|x|+|y|+|z|-|x+y|-|y+z|-|z+x|+|x+y+z| \geq 0$$
	对所有的实数$x,y,z$成立.
\end{problem}

\begin{solution}
	\begin{guess}
		不妨设$|x| \leq |y| \leq |z|$,并将原不等式重组:$$(|x|+|y|-|x+y|)+(|x+y+z|-|y+z|-|x+z|+|z|) \geq 0$$
		前半部分显然非负.我们想要尽可能多地拆掉后半部分的绝对值符号,从而证明其非负.以$|x+z|$为例,在$z \geq 0$时$x+z \geq 0$,而在$z \leq 0$时$x+z \leq 0$.为了避免对$z \leq 0$时的复杂讨论,不妨在不等式两侧同除$|z|$,即只需证明$$\left|\frac{x}{z} + \frac{y}{z} + 1 \right| - \left|\frac{x}{z} + 1\right| - \left|\frac{y}{z} + 1\right| + 1 \geq 0$$
	\end{guess}
	不妨设$|x| \leq |y| \leq |z|$. \\
	在$z=0$时,原不等式显然成立.当$z \neq 0$时,记$S = \left|\dfrac{x}{z} + \dfrac{y}{z} + 1 \right| - \left|\dfrac{x}{z} + 1\right| - \left|\dfrac{y}{z} + 1\right| + 1$.由$-1 \leq \dfrac{x}{z} \leq 1,~-1 \leq \dfrac{y}{z} \leq 1$,可知$$S = \left|\dfrac{x}{z} + \dfrac{y}{z} + 1 \right| - \ssb{\dfrac{x}{z} + \dfrac{y}{z} + 1} \geq 0$$
	故$|z|S = |x+y+z|-|y+z|-|x+z|+|z| \geq 0$.又因为$|x| + |y| - |x+y| \geq 0$,原不等式成立.
\end{solution}

\begin{problem}
	(II,p15,B11)设$a,b,c$是一个三角形的三边长.证明:$$\frac{\sqrt{b+c-a}}{\sqrt{b}+\sqrt{c}-\sqrt{a}} + \frac{\sqrt{c+a-b}}{\sqrt{c}+\sqrt{a}-\sqrt{b}} + \frac{\sqrt{a+b-c}}{\sqrt{a}+\sqrt{b}-\sqrt{c}} \leq 3$$
\end{problem}
\begin{solution}
	令$(x,y,z)=\ssb{\sqrt{b}+\sqrt{c}-\sqrt{a},\sqrt{c}+\sqrt{a}-\sqrt{b},\sqrt{a}+\sqrt{b}-\sqrt{c}}$. \\
	以第一项为例,$b+c-a = \dfrac{x^2+xy+xz-yz}{2}$,故$$\frac{\sqrt{b+c-a}}{\sqrt{b}+\sqrt{c}-\sqrt{a}} = \sqrt{\frac{x^2+xy+xz-yz}{2x^2}} = \sqrt{1 - \frac{x^2 - xy - xz + yz}{2x^2}} = \sqrt{1 - \frac{(x-y)(x-z)}{2x^2}}$$
	注意到,$\sqrt{1-t} \leq 1 - \dfrac{1}{2}t$,于是$$\frac{\sqrt{b+c-a}}{\sqrt{b}+\sqrt{c}-\sqrt{a}} \leq 1 - \frac{(x-y)(x-z)}{4x^2}$$
	由对称性,现在只需证明$$\frac{(x-y)(x-z)}{x^2} + \frac{(y-z)(y-x)}{y^2} + \frac{(z-x)(z-y)}{z^2} \geq 0$$
	不妨设$x \leq y \leq z$.由$$\frac{(x-y)(x-z)}{x^2} = \frac{(y-x)(z-x)}{x^2} \geq \frac{(y-x)(z-y)}{y^2}$$
	可得$$LHS \geq \frac{(z-x)(z-y)}{z^2} \geq 0$$
\end{solution}

\begin{problem}
	(II,p15,B12)设实数$x,y,z$满足$x^2+y^2+z^2=2$,求证:$$x+y+z \leq xyz+2$$
\end{problem}
\begin{solution}
	\begin{guess}
		本题的取等条件有些不同.例如,$x=y=1,z=0$时可以取得等号.于是必然要利用一些特殊的构造.
	\end{guess}
	由于$x^2+y^2=2-z^2 \leq 2$,有$x+y \leq \sqrt{2(x^2+y^2)} \leq 2$与$xy \leq \ssb{\dfrac{x+y}{2}}^2 \leq 1$,两不等式等号均在$x=y=1,z=0$时取得.于是,$$S = x+y+z-xyz-2 = x+y+z(1-xy)-2$$
	\buzhou{1}若$x,y,z$中有至少一个非正数,不妨设$z \leq 0$,由上式可得$S \leq 0$. \\
	\buzhou{2}若$x,y,z$全为非负数,有$$x+y+z \leq \sqrt{2[(x+y)^2 + z^2]} = 2\sqrt{xy+1} \leq 2+xy$$
	当$z>1$时,由上式可得$x+y+z \leq 2+xy < 2+xyz$; \\
	当$0 \leq z \leq 1$时,有$1-z \geq 0$.注意到$-S = (1-z)(1-xy) + (1-x)(1-y)$,不妨设$x \leq y \leq z$\hlt{(注意!这里的不妨设实际上应该写在分类讨论开头,按这样的顺序写只是为了便于理解思考过程)},于是有$-S \geq 0$,即$S \leq 0$.
\end{solution}

\begin{problem}
	(II,p187,A2)设$a=\sqrt{3x+1}+\sqrt{3y+1}+\sqrt{3z+1}$,其中$x+y+z=1,~x,y,z\geq 0$.求$[a]$.
\end{problem}
\begin{solution}
	最大值:$$a \leq 3 \cdot \sqrt{\dfrac{3x+1+3y+1+3z+1}{3}} = 3 \sqrt{2}$$
	最小值:由$0 \leq x,y,z \leq 1$,有$x(1-x) \geq 0$,即$x \geq x^2$,$y,z$同理.故
	\begin{align*}
		a &\geq \sqrt{x^2+2x+1} + \sqrt{y^2+2y+1} + \sqrt{z^2+2z+1} \\
		&= x+1+y+1+z+1 = 4
	\end{align*}
	综上,$4 \leq a <5$,即$[a]=4$.
\end{solution}
\begin{remark}
	对于变量的上下界约束,常常使用类似解答中的思路处理.
\end{remark}

\begin{problem}
	(II,p187,A3)设$a,d \geq 0,~b,c >0$,且$b+c \geq a+d$,则$\dfrac{b}{c+d}+\dfrac{c}{a+b}$的最小值为\tk .
\end{problem}
\begin{solution}
	\begin{guess}
		题目所给条件和结论不太对称,进行一些变换: \\
		由$b+c \geq a+d$,可知$b+c \geq \dfrac{1}{2}(a+b+c+d)$;所求即为$\dfrac{b+c}{c+d}+c\ssb{\dfrac{1}{a+b}-\dfrac{1}{c+d}}$. \\
		于是放缩如下:$$S_0 \geq \dfrac{1}{2}+\dfrac{b+c}{2(a+b)} + c\ssb{\dfrac{1}{a+b}-\dfrac{1}{c+d}}$$
		想要将后面括号中的$\dfrac{1}{c+d}$放缩掉,即将$c$放为$c+d$,也就要求$d\ssb{\dfrac{1}{a+b}-\dfrac{1}{c+d}} \leq 0$. \\
		上式成立的条件是$\ssb{\dfrac{1}{a+b}-\dfrac{1}{c+d}} \leq 0$.很明显,这只是两种情况中的一种.回顾我们一开始做的变形,也可以将所求式子变为$\dfrac{b+c}{a+b} + b\ssb{\dfrac{1}{c+d}-\dfrac{1}{a+b}}$.于是考虑进行分类讨论.
	\end{guess}
	\buzhou{1}当$\ssb{\dfrac{1}{a+b}-\dfrac{1}{c+d}} \leq 0$时,由$b+c \geq \dfrac{1}{2}(a+b+c+d)$,
	\begin{align*}
		S_0 &= \frac{b+c}{c+d}+c\ssb{\frac{1}{a+b}-\frac{1}{c+d}} \\
		&\geq \frac{1}{2}+\frac{b+c}{2(a+b)} + c\ssb{\frac{1}{a+b}-\frac{1}{c+d}} \\
		&\geq \frac{1}{2}+\frac{b+c}{2(a+b)} + (c+d)\ssb{\frac{1}{a+b}-\frac{1}{c+d}} \\
		&= \frac{b+c}{2(a+b)} + \frac{c+d}{a+b} - \frac{1}{2} \\
		&\geq 2\cdot \sqrt{\frac{b+c}{2(a+b)} \cdot \frac{c+d}{a+b}} - \frac{1}{2} = \sqrt{2} - \frac{1}{2}
	\end{align*}
	等号在$(a,b,c,d)=(\sqrt{2}+1,\sqrt{2}-1,2,0)$时取到. \\
	\buzhou{2}当$\ssb{\dfrac{1}{a+b}-\dfrac{1}{c+d}} \geq 0$时,同理可得
	\begin{align*}
		S_0 &= \frac{b+c}{a+b} + b\ssb{\frac{1}{c+d}-\frac{1}{a+b}} \\
		&\geq \frac{1}{2} + \frac{c+d}{2(a+b)} + (a+b)\ssb{\frac{1}{c+d}-\frac{1}{a+b}} \\
		&= \frac{c+d}{2(a+b)} + \frac{a+b}{c+d} - \frac{1}{2} \\
		&\geq 2\cdot \sqrt{\frac{c+d}{2(a+b)} \cdot \frac{a+b}{c+d}} - \frac{1}{2} = \sqrt{2} - \frac{1}{2}
	\end{align*}
	等号在$(a,b,c,d)=(0,2,\sqrt{2}-1,\sqrt{2}+1)$时取到.
\end{solution}

\begin{problem}
	(II,p187,A6)设$x,y,z$均为正实数,且$xyz=1$.证明:$$\frac{x^6+2}{x^3} + \frac{y^6+2}{y^3} + \frac{z^6+2}{z^3} \geq 3\ssb{\frac{x}{y} + \frac{y}{z} + \frac{z}{x}}$$
\end{problem}
\begin{solution}
	对左式进行重组,得$$LHS = \sum \ssb{x^3+\frac{2}{y^3}} = \sum \ssb{ x^3 + \frac{1}{y^3} +1 } + \sum \frac{1}{x^3} -3 \geq 3\sum \frac{x}{y}$$
	该不等式在$x=y=z=1$时取到等号.
\end{solution}

\begin{problem}
	(II,p187,A7)设$n \in \mathbb{N}^*$,证明:$$n[(1+n)^{\frac{1}{n}}-1] < 1 + \frac{1}{2} + \frac{1}{3} + \cdots + \frac{1}{n}$$
\end{problem}
\begin{solution}
	只需证明:$$\sqrt[n]{n+1} < \frac{1}{n} \ssb{ 1 + \frac{1}{2} + \frac{1}{3} + \cdots + \frac{1}{n} + n }$$
	将最后一个“$n$”平均分给每个分数,即$$RHS = \frac{1}{n} \ssb{ 2 + \frac{3}{2} + \frac{4}{3} + \cdots + \frac{n+1}{n} } \geq \sqrt[n]{n+1}$$
	这里的等号是取不到的.故原不等式成立.
\end{solution}

\begin{problem}
	(II,p188,B9)设$x,y,z$为正数,且$x^2+y^2+z^2=1$.求$S=\dfrac{xy}{z} + \dfrac{yz}{x} + \dfrac{zx}{y}$的最小值.
\end{problem}
\begin{solution}
	此时$S$是一次的.将$S$平方,可得$$S^2 = \frac{x^2y^2}{z^2} + \frac{y^2z^2}{x^2} + \frac{z^2x^2}{y^2} + 2(x^2+y^2+z^2)$$
	注意到,$\dfrac{x^2y^2}{z^2} + \dfrac{y^2z^2}{x^2} \geq 2y^2$.由对称性,有$$S^2 \geq 3(x^2+y^2+z^2) = 3$$
	故$S$的最小值为$\sqrt{3}$.该最小值在$x=y=z=\dfrac{\sqrt{3}}{3}$时取得.
\end{solution}

\begin{problem}
	(II,p188,B12)对任意大于$1$的正整数$n$和正数$a_1,\cdots ,a_n$,求最大的正数$\lambda$,使$$\frac{\sqrt{a_1^{n-1}}}{\sqrt{a_1^{n-1} + (n^2-1)a_2a_3 \cdots a_n}} + \cdots + \frac{\sqrt{a_n^{n-1}}}{\sqrt{a_n^{n-1} + (n^2-1)a_1a_2 \cdots a_{n-1}}} \geq \lambda$$恒成立.
\end{problem}
\begin{solution}
	XXX
\end{solution}

\begin{problem}
	(II,p197,A3)设$a,b,c,d$均为实数,满足$a+2b+3c+4d=\sqrt{10}$,则$a^2+b^2+c^2+d^2+(a+b+c+d)^2$的最小值为\tk .
\end{problem}
\begin{solution}
	待定系数$t$,考虑
	\begin{align*}
		&\left[ a^2+b^2+c^2+d^2+(a+b+c+d)^2 \right] \cdot \left[ t^2+(1-t)^2+(2-t)^2+(3-t)^2+(4-t)^2 \right] \\
		&\geq \left[ t(a+b+c+d) + (1-t)a + (2-t)b + (3-t)c + (4-t)d \right]^2
	\end{align*}
	即得$S_0 \geq \dfrac{10}{5t^2-20t+30}$,对于任意的$t$该式子都应成立.又由$RHS \leq 1$可知$S_0 \geq 1$.
\end{solution}

\chapter{组合}

\part{《启东中学奥赛教程》}

\setcounter{chapter}{0}

\chapter{集合}

\section{集合的概念与运算}

\begin{problem}
	(P2,例2)已知函数$f(x)=x^2+ax+b(a,b \in \mathbb{R})$,且集合$A=\{x|x=f(x)\}$,$B=\{x|x=f(f(x))\}$. \\
	(1)求证:$A \subseteq B$;\\
	(2)当$A=\{-1,3\}$时,用列举法表示$B$;\\
	(3)求证:若$A$只含有一个元素,则$A=B$. 
\end{problem}
\tutor{BV1pG4y1q7LD}

\begin{solution}
	(1)设$f(x_0)=x_0$,则$f(f(x_0))=f(x_0)=x_0$.因此,$A$中任意一个元素都在$B$中,即$A \subseteq B$. \\
	(2)由题,方程$x^2+ax+b=0$有两根$-1,3$,则解得$a=-1,b=-3$. \\
	那么集合$B$的元素就是方程$$(x^2-x-3)^2 - (x^2-x-3) - 3 = x$$
	的解,即方程$$(x+1)(x-3)(x+\sqrt{3})(x-\sqrt{3}) = 0$$
	的解$-1,3,-\sqrt{3},\sqrt{3}$.所以$B=\{ -1,3,-\sqrt{3},\sqrt{3} \}$. \\
	(3)由于$f(f(x))=x$,即$$x^4 + 2ax^3 + (a^2+2b+a)x^2 + (2ab+a^2-2a-1)x + (b^2+ab+b) = 0$$
	可以分解为$$[x^2 + (a-1)x + b] [x^2 + (a+1)x +a+b+1] = 0$$
	而后式$x^2 + (a+1)x +a+b+1$显然不为$0$,所以$B$也只有一个元素.由$A \subseteq B$,可知$A=B$.
\end{solution}

\newpage
\begin{problem}
	(P4,例5)设$a_1,a_2,a_3,a_4$是$4$个有理数,使得$\{a_ia_j|1 \leq i < j \leq 4 \}=\{-24,-2,-\dfrac{3}{2},-\dfrac{1}{8},1,3\}$. 求$a_1+a_2+a_3+a_4$的值.
\end{problem}
\tutor{BV1pG4y1q7LD}

\begin{solution}
	不妨设$|a_1| \leq |a_2| \leq |a_3| \leq |a_4|$.则有$$|a_1a_2| \leq |a_1a_3| \leq \min \{ |a_2a_3|,|a_1a_4| \} \leq \max \{ |a_2a_3|,|a_1a_4| \} \leq |a_2a_4| \leq |a_3a_4|$$
	则可得$$
	\begin{cases}
		|a_1a_2| = \dfrac{1}{8} \\
		|a_1a_3| = 1 \\
		|a_2a_4| = 3 \\
		|a_3a_4| = 24
	\end{cases}
	\quad \Longrightarrow \quad
	\begin{cases}
		a_1a_2 = -\dfrac{1}{8} \\
		a_1a_3 = 1 \\
		a_2a_4 = 3 \\
		a_3a_4 = -24
	\end{cases}
	$$
	为了找出$|a_2a_3|$与$|a_1a_4|$到底谁大,不妨用$a_1$将其他量表示出来,即$$a_2 = -\frac{1}{8a_1} , \ a_3=\frac{1}{a_1} , \ a_4=-24a_1$$
	所以$$a_2a_3 = -\frac{1}{8a_1^2} , \ a_1a_4 = -24a_1^2$$
	如果$a_2a_3 = -2$,解得$a_1 = \pm \dfrac{1}{4}$,此时$a_1+a_2+a_3+a_4 = \pm \dfrac{9}{4}$; \\
	如果$a_1a_4 = -2$,解得$a_1 = \pm \dfrac{\sqrt{12}}{12}$,与题意矛盾. \\
	综上,$a_1+a_2+a_3+a_4 = \pm \dfrac{9}{4}$.
\end{solution}

\newpage
\begin{problem}
	(P4,例6)已知集合$A=\{a_1,a_2,a_3,a_4\}$,且$a_1<a_2<a_3<a_4$,$a_i \in \mathbb{N}^{*} (i=1,2,3,4)$. 记$a_1+a_2+a_3+a_4=S$,集合$B=\{(a_i,a_j)|(a_i+a_j)|S,a_i,a_j\in A , i<j\}$中的元素个数为$4$个,求$a_1$的值.
\end{problem}
\tutor{BV1pG4y1q7LD}

\begin{solution}
	由$a_1<a_2<a_3<a_4$,有$$a_1+a_2 < a_1+a_3 < \min \{ a_2+a_3,a_1+a_4 \} \leq \max \{ a_2+a_3,a_1+a_4 \} < a_2+a_4 < a_3+a_4$$
	其中,由于$a_2+a_4 > a_1+a_3$,可知$\dfrac{S}{2} < a_2+a_4 < a_3+a_4$,所以自然$(a_1+a_2),(a_2+a_3),(a_1+a_4),(a_3+a_4)$均能整除$S$. \\
	如果此时$a_2+a_3 \neq a_1+a_4$,它们之中一定有一个大于$\dfrac{S}{2}$,所以$a_2+a_3 = a_1+a_4 = \dfrac{S}{2}$. \\
	接下来,对$a_1+a_3$等项的具体取值进行讨论:\\
	\buzhou{1} 设$a_1+a_3 = \dfrac{S}{3}$,$a_1+a_2 = \dfrac{S}{k}$($k \geq 4$).由此解得$$(a_1,a_2,a_3,a_4) = \frac{S}{12k} (6-k,6+k,5k-6,7k-6)$$
	又因为$0<a_1<a_2<a_3<a_4$,则$$0 < 6-k < 6+k < 5k-6 < 7k-6$$
	解得$3 < k < 6$,即$k=4,5$. \\
	如果$k=4$,可知$(a_1,a_2,a_3,a_4) = \dfrac{S}{24} (1,5,7,11)$,即$a_1=\dfrac{S}{24}$; \\
	如果$k=5$,可知$(a_1,a_2,a_3,a_4) = \dfrac{S}{60} (1,11,19,29)$,即$a_1=\dfrac{S}{60}$. \\
	\buzhou{2} 设$a_1+a_3 = \dfrac{S}{4}$,$a_1+a_2 = \dfrac{S}{k}$($k \geq 5$).由此解得$$(a_1,a_2,a_3,a_4) = \frac{S}{8k} (4-k,k+4,3k-4,5k-4)$$
	所以有$0 < 4-k < k+4 < 3k-4 < 5k-4$,解得$k<4$且$k<4$,即这样的$k$不存在. \\
	\buzhou{3} 由以上两步,尝试证明接下来的情况均不成立. \\
	设$a_1+a_3 = \dfrac{S}{m}$($m \geq 4$),$a_1+a_2 = \dfrac{S}{k}$($k \geq m+1$).由此解得$$(a_1,a_2,a_3,a_4) = \frac{S}{4mk} (2m+2k-mk,2m-2k+mk,2k-2m+mk,3mk-2m-2k)$$
	同理,可以解得$m+1 \leq k < \dfrac{2m}{m-2}$.构造函数$$f(x) = (m+1) - \frac{2m}{m-2} = (m-2) - \frac{4}{m-2} + 1$$
	显然$f(x)$在$\R$上单调递增,即有$$\forall x \in \R, f(x) \geq f(4) = 1 > 0$$
	也就意味着$m+1 > \dfrac{2m}{m-2}$,即$k$无解. \\
	综上,$a_1=\dfrac{S}{24} \ or \ \dfrac{S}{60}$
\end{solution}

\newpage
\begin{problem}
	(P5,例7)$X$是非空的正整数集合,满足下列条件:\\
	(1)若$x \in X$,则$4x \in X$;(2)若$x \in X$,则$\left \lfloor \sqrt{x} \right \rfloor  \in X$. \\
	求证:$X$是全体正整数的集合.
\end{problem}
\tutor{BV1pG4y1q7LD}

\begin{problem}
	(P5,例8)设$S$为非空数集,且满足:(1) $2 \notin S$;(2) 若$a \in S$,则$\dfrac{1}{2-a} \in S$. 证明:\\
    (1) 对一切$n \in \mathbb{N}^*$,$n \geq 3$,有$\dfrac{n}{n-1} \notin S$;\\
    (2) $S$或者为单元素集,或者是无限集.
\end{problem}
\tutor{BV1pG4y1q7LD}

\begin{problem}
	(P6,习题10)称有限集$S$的所有元素的乘积为$S$的“积数”,给定数集$M=\{\dfrac12,\dfrac13, \cdots ,\dfrac{1}{100}\}$,则集合$M$的所有含偶数个元素的子集的“积数”的和为\tk .
\end{problem}
\tutor{BV1pG4y1q7LD}

\begin{problem}
	(P7,习题11)设集合$M=\{u|u=12m+8n+4l,m,n,l \in \mathbb{Z}\}$,$N=\{v|v=20p+16q+12r,p,q,r \in \mathbb{Z}\}$. 求证:$M=N$.
\end{problem}
\tutor{BV1pG4y1q7LD}

\begin{problem}
	(P7,习题13)以某些整数为元素的集合$P$具有下列性质:(1)$P$中的元素有正数,有负数;(2)$P$中的元素有奇数,有偶数;(3)$-1 \notin P$;(4)若$x,y \in P$,则$x+y \in P$. 试证明:\\
    (1)$0 \in P$;(2)$2 \notin P$.
\end{problem}
\tutor{BV1pG4y1q7LD}

\begin{problem}
	(P7,习题14)已知数集$A$具有以下性质:\\
    (1)$0 \in A$,$1 \in A$;\\
    (2)若$x,y \in A$,则$x-y \in A$;\\
    (3)若$x \in A$,$x \neq 0$,则$\frac{1}{x} \in A$. \\
    求证:当$x,y \in A$时,则$xy \in A$.
\end{problem}
\tutor{BV1pG4y1q7LD}

\newpage
\section{有限集合的元素个数}

\begin{problem}
	(P11,例8)设$n,k \in \mathbb{N}^*$,且$k \leq n$,并设$S$是含有$n$个互异实数的集合,$T=\{a|a=x_1+x_2+ \cdots +x_k,x_i \in S,x_i \neq x_j(i \neq j),1 \leq i,j \leq k\}$.求证:$|T| \geq k(n-k)+1$.
\end{problem}
\tutor{BV1qN4y1K7Mw}

\begin{problem}
	(P12,习题5)设集合$M=\{1,2,3,\cdots,1995\}$,$A$是$M$的子集且满足条件:当$x \in A$时,$15x \notin A$,则$A$中元素的个数最多是\tk .
\end{problem}
\tutor{BV1qN4y1K7Mw}

\begin{problem}
	(P12,习题11)求最大正整数$n$,使得$n$元集合$S$同时满足:\\
    (1)$S$中的每个数均为不超过$2002$的正整数;\\
    (2)对于$S$的两个元素$a$和$b$(可以相同),它们的乘积$ab$不属于$S$.
\end{problem}
\tutor{BV1qN4y1K7Mw}

\newpage
\section{子集的性质}

\begin{problem}
	(P14,例1)设$S$为集合$\{1,2,3, \cdots ,50\}$的非空子集,$S$中任何两个数之和不能被$7$整除. 求$card(S)$的最大值.
\end{problem}
\tutor{BV1bP4y1S76v}

\begin{problem}
	(P14,例2)已知集合$A=\{1,2, \cdots ,10\}$. 求集合$A$的具有下列性质的子集个数:每个子集至少含有$2$个元素,且每个子集中任何两个元素的差的绝对值大于$1$.
\end{problem}
\tutor{BV1bP4y1S76v}

\begin{problem}
	(P15,例3)证明:任何一个有限集的全部子集可以这样地排列顺序,使任何两个相邻的集相差一个元素.
\end{problem}
\tutor{BV1bP4y1S76v}

\begin{problem}
	(P15,例4)对于整数$n(n \geq 2)$,如果存在集合$\{1,2, \cdots ,n\}$的子集族$A_1,A_2, \cdots ,A_n$满足以下条件,则称$n$是“好数”:\\
    (a)$i \notin A_i,i=1,2, \dots ,n$;\\
    (b)若$i \neq j,i,j \in \{1,2, \cdots ,n\}$,则$i \in A_j$当且仅当$j \notin A_i$;\\
    (c)任意$i,j \in \{1,2, \cdots ,n\}$,$A_i \bigcap A_j \neq \Phi$.\\
    证明:(1)$7$是“好数”;(2)当且仅当$n \geq 7$时,$n$是“好数”.
\end{problem}
\tutor{BV1bP4y1S76v}

\begin{problem}
	(P17,例8)设$k,n$为给定的整数,$n>k \geq 2$,对任意$n$元的数集$P$,作$P$的所有$k$元子集的元素和,记这些和组成的集合为$Q$,集合$Q$中元素个数是$C_Q$.求$C_Q$的最大值和最小值.
\end{problem}
\tutor{BV1bP4y1S76v}

\begin{problem}
	(P17,例9)设集合$S_n=\{1,2, \cdots ,n\}$. 若$X$是$S_n$的子集,把$X$中所有数的和为$X$的“容量”(规定空集的容量为$0$),若$X$的容量为奇(偶)数,则称$X$为$S_n$的奇(偶)子集. \\
    (1)证明:$S_n$的奇子集与偶子集的个数相等;\\
    (2)证明:当$n>2$时,$S_n$的所有奇子集的容量之和等于所有偶子集的容量之和;\\
    (3)当$n>2$时,求$S_n$的所有奇子集的容量之和.
\end{problem}
\tutor{BV1bP4y1S76v}

\begin{problem}
	(P19,习题11)设$p$是一个奇质数,考虑集合$\{1,2, \cdots ,2p\}$满足以下两个条件的子集$A$:\\
    (i)$A$恰有$p$个元素;(ii)$A$中所有元素之和可被$p$整除.\\
    试求所有这样的子集$A$的个数.
\end{problem}
\tutor{BV1bP4y1S76v}

\begin{problem}
	(P19,习题12)设$n \in \mathbb{N}^*,n \geq 2$,$S$是一个$n$元集合. 求最小的正整数$k$,使得存在$S$的子集$A_1,A_2, \cdots ,A_k$具有如下性质:对$S$中的任意两个不同元素$a,b$,存在$j \in \{1,2, \cdots ,k\}$,使得$A_j \bigcap \{a,b\}$为$S$的一元子集.
\end{problem}
\tutor{BV1bP4y1S76v}

\begin{problem}
	(P19,习题14)设$S$表示不超过$79$的所有奇合数组成的集合. \\
    (1)试证:$S$可以划分为三个子集,而每个子集的元素都构成等差数列;\\
    (2)讨论:$S$能否划分为两个上述集合?
\end{problem}
\tutor{BV1bP4y1S76v}

\newpage
\section{综合题解}

\begin{problem}
	(补1)对于任何集合$S$,用$|S|$表示集合$S$中的元素个数,用$n(S)$表示集合$S$的子集个数. 若$A,B,C$是三个有限集,且满足条件:(1)$|A|=|B|=1000$;(2)$n(A)+n(B)+n(C)=n(A \cup B \cup C)$. 求$|A \cap B \cap C|$的最大值.
\end{problem}
\tutor{BV1gv4y1U7VL}

\begin{problem}
	(补2)给定集合$A= \{1,2,3, \cdots ,2n+1 \}$. 试求一个包含元素最多的集合$A$的子集$B$,使$B$中任意三个元素$x,y,z$(可相同)都有$x+y \neq z$.
\end{problem}
\tutor{BV1gv4y1U7VL}

\begin{problem}
	(补3)有$1987$个集合,每个集合有$45$个元素,任意两个集合的并集有$89$个元素,问此$1987$个集合的并集有多少个元素?
\end{problem}
\tutor{BV1gv4y1U7VL}

\begin{problem}
	(P20,例3)在前$200$个自然数中,任取$101$个数,求证:一定存在两个数,其中一个是另一个的整数倍.
\end{problem}
\tutor{BV1gv4y1U7VL}

\begin{problem}
	(P21,例5)已知$S_1,S_2,S_3$为非空整数集合,且对于$1,2,3$的任意一个排列$i,j,k$,若$x \in S_i, y \in S_j$,则$x-y \in S_k$.\\
    (1)证明:$S_1,S_2,S_3$三个集合中至少有两个相等.\\
    (2)这三个集合中是否可能有两个集合无公共元素?
\end{problem}
\tutor{BV1gv4y1U7VL}

\begin{problem}
	(P22,例7)设$a_1,a_2, \cdots ,a_{20} \in \{  1,2,3,4,5  \}$,$b_1,b_2, \cdots ,b_{20} \in \{  1,2,3,\cdots ,10  \}$,集合$X=\{  (i,j)|1 \leq i < j \leq 20 , (a_i-a_j)(b_i-b_j)<0  \}$,求$X$的元素个数的最大值.
\end{problem}
\tutor{BV1gv4y1U7VL}

\begin{problem}
	(P22,例8)设$S=\{  (a,b)|a \in \{ 1,2,\cdots ,m \},b \in \{ 1,2, \cdots ,n \}  \}$,其中正整数$m \geq 2,n \geq 3$,$A$为$S$的子集.若$A$满足:不存在正整数$x_1,x_2,y_1,y_2,y_3$,使得$x_1 < x_2$,$y_1 < y_2 < y_3$,且$(x_1,y_1),(x_1,y_2),(x_1,y_3),(x_2,y_2) \in A$,求$A$的元素个数的最大值.
\end{problem}
\tutor{BV1gv4y1U7VL}

\begin{problem}
	(P24,习题4)在集合$M = \{  1,2,\cdots ,10  \}$的所有子集中,有这样一族不同的子集,它们两两的交集都不是空集,求这族子集的个数最大值.
\end{problem}
\tutor{BV1gv4y1U7VL}

\begin{problem}
	(P24,习题10)已知$A$和$B$是集合$\{  1,2,3, \cdots ,100  \}$的两个子集,满足:$A$与$B$的元素个数相同,且$A \cap B = \varnothing$,若$n \in A$时,总有$2n+2 \in B$,求集合$A \cup B$的元素个数的最大值.
\end{problem}
\tutor{BV1gv4y1U7VL}

\begin{problem}
	(P24,习题11)集合$S=\{  1,2, \cdots ,1990  \}$,考察$S$的$31$元子集.如果子集中$31$个元素之和可被$5$整除,则称为是好的.求$S$的好子集个数.
\end{problem}
\tutor{BV1gv4y1U7VL}

\chapter{函数}

\section{函数概念}
\section{函数的性质与图像}

\begin{problem}
	(P35,例3)已知$x,y \in \left[ -\dfrac{\pi}{4},\dfrac{\pi}{4} \right]$,$a \in \mathbb{R}$,且$x^3+\sin x -2a = 0$,$4y^3 + \sin y \cos y + a=0$.求$\cos (x+2y)$的值.
\end{problem}
\tutor{BV1ks4y1W7rw}

\begin{problem}
	(P35,例5)求函数$f(x)=|\sin x|+|\cos x|$的最小正周期.
\end{problem}
\tutor{BV1ks4y1W7rw}

\begin{problem}
	(P37,例8)设$f$和$g$是定义在整数集上且取值为整数的两个函数,满足对任意整数$x,y$,都有$$f(g(x)+y)=g(f(y)+x)$$
	假设$f$是有界的,证明:$g$是周期函数,即存在正整数$T$,使得$$g(x+T)=g(x)$$
	对所有整数$x$成立.
\end{problem}
\tutor{BV1ks4y1W7rw}

\newpage
\section{二次函数、幂函数、指数函数与对数函数}

与二次函数相关的问题:调整,放缩,参变互换,换元

参变互换的一个例子:已知$\sqrt{3}y - 3z = x$,求证:$y^2 \geq 4xz$.

\begin{problem}
	(P45,例8)设二次函数$f(x)=ax^2+bx+c~(a>0)$,方程$f(x)-x=0$的两个根$x_1,x_2$满足$0<x_1<x_2< \dfrac{1}{a}$. \\
	(1)当$x \in (0,x_1)$时,证明:$x<f(x)<x_1$; \\
	(2)设函数$f(x)$的图像关于直线$x=x_0$对称,证明:$x_0 < \dfrac{1}{2}x_1$.
\end{problem}
\tutor{BV1ks4y1W7rw}

\begin{problem}
	(P45,习题1)已知$f(x)=ax^2-c$满足$-4 \leq f(1) \leq -1,~ -1 \leq f(2) \leq 5$,那么$f(3)$应该满足\tk .
\end{problem}
\tutor{BV1ks4y1W7rw}

\begin{problem}
	(P46,习题10)已知奇函数$f(x)$在区间$(-\infty ,0)$上是增函数,且$f(-2)=-1,~f(1)=0$,当$x_1>0,~ x_2>0$时,有$f(x_1x_2)=f(x_1)+f(x_2)$,则不等式$\log_{2}{|f(x)+1|}<0$的解集为\tk .
\end{problem}
\tutor{BV1ks4y1W7rw}

\begin{problem}
	(P46,习题14)二次函数$f(x)=px^2+qx+r$中,实数$p,q,r$满足$\dfrac{p}{m+2}+\dfrac{q}{m+1}+\dfrac{r}{m}=0$,其中$m>0$.求证: \\
	(1)$~~pf \ssb{\dfrac{m}{m+1}} < 0$; \\
	(2)方程$f(x)=0$在$(0,1)$内恒有解.
\end{problem}
\tutor{BV1ks4y1W7rw}

\begin{problem}
	(P46,习题15)已知$a,b,c$是实数,函数$f(x)=ax^2+bx+c,~g(x)=ax+b$,当$-1 \leq x \leq 1$时,$|f(x)| \leq 1$. \\
	(1)证明:$|c| \leq 1$; \\
	(2)证明:当$-1 \leq x \leq 1$时,$|g(x)| \leq 2$; \\
	(3)设$a>0$,当$-1 \leq x \leq 1$时,$g(x)$的最大值为$2$,求$f(x)$.
\end{problem}
\tutor{BV1ks4y1W7rw}

\newpage
\section{函数的最大值与最小值}

\begin{problem}
	(P47,例1)已知函数$y=\dfrac{ax^2+8x+b}{x^2+1}$的最大值为$9$,最小值为$1$.试求函数$y=\sqrt{ax^2+8x+b}$的值域. \\
	(习题15)设关于$x$的一元二次方程$2x^2-tx-2=0$的两个根为$\alpha , \beta ~ (\alpha < \beta)$.若$x_1,x_2$为区间$[\alpha ,\beta]$上的两个不同的点,求证:$4x_1x_2-t(x_1+x_2)-4<0$.
\end{problem}
\tutor{BV1ks4y1W7rw}

\begin{problem}
	(P49,例5)设$x,y,z \in \mathbb{R}^{+}$,且$x+y+z=1$,求$$u=\frac{3x^2-x}{1+x^2}+\frac{3y^2-y}{1+y^2}+\frac{3z^2-z}{1+z^2}$$的最小值.
\end{problem}
\tutor{BV1ks4y1W7rw}

\begin{problem}
	(P50,例6)设函数$f:(0,1) \to \mathbb{R}$定义为$$\kaishu
	f(x)=\begin{cases}
		x, &\text{当}~x~\text{是无理数时;} \\
		\dfrac{p+1}{q}, &\text{当}~x=\dfrac{p}{q},~(p,q)=1,~0<p<q~\text{时}
	\end{cases}$$\songti 求$f(x)$在区间$\ssb{\dfrac{7}{8},\dfrac{8}{9}}$上的最大值.
\end{problem}
\tutor{BV1ks4y1W7rw}

\begin{problem}
	(P56,习题7)函数$y=\dfrac{\sqrt{x^2+1}}{x-1}$的值域是\tk .
\end{problem}
\tutor{BV1ks4y1W7rw}

\begin{problem}
	(P56,习题9)设$f(x)=x^2+px+q~(p,q \in \mathbb{R})$.若$|f(x)|$在$[-1,1]$上的最大值为$M$,则$M$的最小值为\tk .
\end{problem}
\tutor{BV1ks4y1W7rw}

\begin{problem}
	(P56,习题13)已知$f(x)=\lg (x+1),~g(x)=2\lg (2x+t)$(其中$t$为参数,且$t \in \mathbb{R}$).如果$x \in [0,1]$时,$f(x) \leq g(x)$恒成立,求参数$t$的取值范围.
\end{problem}
\tutor{BV1ks4y1W7rw}

\begin{problem}
	(P56,习题14)已知$\alpha ,\beta$是方程$4x^2-4tx-1=0~(t \in \mathbb{R})$的两个不等实根,函数$f(x)=\dfrac{2x-t}{x^2+1}$的定义域为$[\alpha ,\beta]$. \\
	(1)求$g(t)=f(x)_{\text{max}} - f(x)_{\text{min}}$; \\
	(2)证明:对于$u_i \in \ssb{0,\dfrac{\pi}{2}} ~ (i=1,2,3)$,若$\sin u_1 + \sin u_2 + \sin u_3 = 1$,则$$\frac{1}{g(\tan u_1)} + \frac{1}{g(\tan u_2)} + \frac{1}{g(\tan u_3)} < \frac{3}{4} \sqrt{6}$$
\end{problem}
\tutor{BV1ks4y1W7rw}

\chapter{数列}

\chapter{数学归纳法}

\chapter{三角函数}

\section{三角函数的性质}

\begin{problem}
	(P124,例3)设$\omega$是正实数,若存在$a,b~(\pi \leq a < b \leq 2\pi)$,使得$\sin \omega a + \sin \omega b =2$,求$\omega$的取值范围.
\end{problem}

\newpage
\begin{problem}
	(P126,例5)设函数$f(x)=\sin \ssb{\dfrac{11}{6} \pi x + \dfrac{\pi}{3}}$. \\
	(1)对于任意的正数$\alpha$,是否总能找到不小于$\alpha$,且不大于$(\alpha +1)$的两个数$a,b$,使$f(a)=1,f(b)=-1$?请回答并讨论. \\
	(2)若$\alpha$是任意自然数,请重新回答和论证上述问题.
\end{problem}

\newpage
\begin{problem}
	(P127,例6)求证:存在唯一的一对实数$\alpha ,\beta \in \ssb{0,\dfrac{\pi}{2}}$,且$\alpha < \beta$,使得$\sin (\cos \alpha) = \alpha ,~\cos (\sin \beta ) = \beta$.
\end{problem}

\vspace{27em}

\begin{problem}
	(P128,例8)函数$F(x)=|\cos ^2 x + 2\sin x \cos x - \sin ^2 x + Ax+B|$在$0 \leq x \leq \dfrac{3}{2}\pi$上的最大值$M$与参数$A,B$有关.问$A,B$取什么值时,$M$为最小?证明你的结论.
\end{problem}

\newpage
\section{三角函数的恒等变形}

\begin{problem}
	(P136,例7)求证下述恒等式:(其中$n \in \mathbb{N}^{*}$)$$\cos \frac{2\pi}{2n+1} + \cos \frac{4\pi}{2n+1} + \cdots + \cos \frac{2n\pi}{2n+1} = -\frac{1}{2}$$
\end{problem}

\vspace{23em}

\begin{problem}
	(P140,习题15)设$n$是一个大于$3$的质数,求$$\ssb{1+2\cos \frac{2\pi}{n}} \ssb{1+2\cos \frac{4\pi}{n}} \ssb{1+2\cos \frac{6\pi}{n}} \cdots \ssb{1+2\cos \frac{2n\pi}{n}}$$
	的值.
\end{problem}

\newpage
\section{三角不等式与三角极值}

\begin{problem}
	(P145,例6)在$\vartriangle ABC$中,$x,y,z$为任意实数,求证:$$x^2+y^2+z^2 \geq 2xy\cos C + 2yz\cos A + 2zx \cos B$$
	其中当且仅当$x:y:z=\sin A : \sin B : \sin C$时取等号.
\end{problem}

\vspace{23em}

\begin{problem}
	(P146,例8)设$n,m$都是正整数,并且$n>m$.证明:对一切$x \in \ssb{ 0,\dfrac{\pi}{2} }$,都有$$2|\sin ^n x - \cos ^n x| \leq 3|\sin ^m x - \cos ^m x|$$
\end{problem}

\newpage
\begin{problem}
	(P148,习题15)在锐角$\vartriangle ABC$中,若$n \in \mathbb{N}$,证明:$$\frac{\cos ^n A}{\cos B + \cos C} + \frac{\cos ^n B}{\cos C + \cos A} + \frac{\cos ^n C}{\cos A + \cos B} \geq \frac{3}{2^n}$$
\end{problem}

\newpage
\section{反三角函数及三角方程}
\section{综合题解}


\end{document}
