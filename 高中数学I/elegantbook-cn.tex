\documentclass[lang=cn, zihao=4.5]{elegantbook}
\usepackage{hyperref}

% font settings


% watermark settings
%\usepackage{ctex, draftwatermark, everypage}
%	\SetWatermarkText{DEEP Team 讲义模版}
%	\SetWatermarkLightness{0.95}
%	\SetWatermarkScale{0.3}

% customised commands
\usepackage{ulem}
	\newcommand{\tk}{\uline{\hspace{4em}}}
	
\DeclareSymbolFont{yh}{OMX}{yhex}{m}{n}
\DeclareMathAccent{\hu}{\mathord}{yh}{"F3}

\newcommand{\xl}[1]{\overrightarrow{#1}}
\newcommand{\nd}[1]{〔#1〕}
\newcommand{\ssb}[1]{\left( #1 \right)}
\newcommand{\sw}[1]{\boxed{\text{解法 #1}} \ }
\newcommand{\buzhou}[1]{$#1^{\circ} \ $}

% cover settings

\title{高中数学I}
\subtitle{适用于联赛一试与强基计划}

\author{Johnny Tang}
\institute{DEEP Team}
\date{January 21, 2022}

\extrainfo{请:相信时间的力量,敬畏概率的准则}


\cover{cover.png}

% 本文档命令


% 修改标题页的橙色带
% \definecolor{customcolor}{RGB}{32,178,170}
% \colorlet{coverlinecolor}{customcolor}


\begin{document}

\maketitle

\frontmatter

\mainmatter

\tableofcontents

\newpage

\part{预备篇}

\setcounter{chapter}{-1}
\chapter{数理逻辑}

充分条件与必要条件,命题的关系,命题的逻辑运算,形式逻辑词

\chapter{代数变换基础}

\section{多项式的概念}

多项式、多项式的根、多项式相等的概念,多项式的运算,多项式的带余除法

\section{多项式的根与Vieta定理}

余数定理,因式定理,Vieta定理

\section{整式恒等变形}

换元技巧,齐次性原理

\section{简单的不等式}

绝对值不等式,糖水不等式,均值不等式,线性规划

\section{求和符号}

求和符号

\chapter{数域与运算}

\section{复数初步}

基本概念

\section{对数运算}

基本概念,运算法则,特殊对数

\part{基础篇}

\setcounter{chapter}{0}
\chapter{集合}

\section{集合的概念}

集合的概念、表示、性质,常见集合,集合中的元素,集合间的关系

\section{集合间的运算与运算律}

交集与并集,运算律

\section{集合元素的个数}

有限元集合的元素个数公式,容斥公式

\chapter{函数}

\section{映射与函数}

映射、映射相等的概念,特殊的映射,逆映射,映射的复合,函数的概念

\section{常见初等函数}

基本初等函数、初等函数的概念

\subsection{二次函数}

二次函数的性质,最值问题,实根分布问题

\subsection{对勾函数}

对勾函数、垃圾函数的性质

\subsection{常值函数、指数函数、幂函数、对数函数}

常值函数的概念、幂函数的概念,指数函数的概念、图像与性质,对数函数的概念、图像与性质

\subsection{三角函数与双曲函数}

详见下一章.

\section{函数的性质}

\subsection{单调性}

单调性的概念,函数单调性的运算,区间根定理

\subsection{奇偶性}

奇偶性的概念,函数奇偶性的运算

\subsection{对称性}

对称性的概念,函数的对称变换,含绝对值的函数

\subsection{周期性}

周期性的概念

\section{函数迭代与函数方程}

\subsection{函数的迭代与不动点}

函数迭代的概念,函数不动点的概念

\subsection{简单的函数方程}

函数方程问题,Cauchy方程

\chapter{三角函数}

\section{三角函数的概念}

任意角,弧度制,三角函数的定义,诱导公式,三角函数的复数表示

\section{三角函数的函数性质}

三角函数的图像与性质

\section{三角恒等变形}

和差角公式,二倍角、半角公式,三倍角公式,万能公式,辅助角公式,积化和差、和差化积公式

\section{正弦定理与余弦定理}

正弦定理,余弦定理

\section{三角换元与三角恒等式}

\subsection{三角换元}

三角换元常见形式

\subsection{三角恒等式}

常见的三角恒等式

\section{反三角函数}

反三角函数的概念、图像与性质

\section{双曲函数}

双曲函数的定义,双曲恒等变形公式

\chapter{平面向量与复数}

\section{平面向量的概念}

平面向量的概念,向量间的关系

\section{平面向量基本定理与坐标表示}

\section{复数的概念与计算}

\chapter{数列}

\section{等差数列与等比数列}

\section{数列的变形}

\section{数列与数学归纳法}

\chapter{极限与导数}

\section{极限的概念与运算}

\section{导数的概念与运算}

\section{导数的应用}

\chapter{不等式}

\section{均值不等式}

\section{Cauchy不等式}

\section{排序不等式}

\section{函数的凹凸性与Jensen不等式}

\section{若干著名不等式}

\chapter{概率、计数与组合}

\section{概率与数学期望}

\section{计数原理}

\section{排列数、组合数}

\section{组合数模型}

\section{二项式定理}

\chapter{三角、向量与几何}

\section{常见平面几何结论——三角}

\section{常见平面几何结论——平面向量}

\chapter{解析几何}

\section{直线与圆}

\section{圆锥曲线}

\chapter{立体几何}

\section{空间中的基本元素}

\section{空间中的位置关系}

\section{空间中的距离与角度}

\section{多面体与球}

\section{空间向量}

\end{document}





















