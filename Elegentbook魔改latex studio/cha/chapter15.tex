\chapter{Nginx}
\rightline{\textcolor[RGB]{255, 0, 0}{\faHeart}~我们终会相知,在那悠远的苍穹{\faHeart}}

\section{Nginx安装}
~

\begin{ascolorbox10}{命令练习}
	\begin{ascboxJ}{1.下载Nginx源代码nginx.org官网:}
%		\begin{lstlisting}[style=linux]
%[root@localhost opt]# ls
%nginx-1.21.1.tar.gz
%		\end{lstlisting}
	\end{ascboxJ}

	\begin{ascboxJ}{2.解压缩Nginx源代码}
%	\begin{lstlisting}[style=linux]
%[root@localhost opt]# tar -zxf nginx-1.21.1.tar.gz
%[root@localhost opt]# ls
%nginx-1.21.1  nginx-1.21.1.tar.gz
%	\end{lstlisting}
\end{ascboxJ}

	\begin{ascboxJ}{3.复制Nginx默认提供的vim语法插件}
%	\begin{lstlisting}[style=linux]
%[root@localhost opt]# mkdir ~/.vim
%[root@localhost opt]# cd nginx-1.21.1
%[root@localhost nginx-1.21.1]# ls
%auto  CHANGES  CHANGES.ru  conf  configure  contrib  html  LICENSE  man  README  src
%[root@localhost nginx-1.21.1]# cp -r contrib/vim/* ~/.vim/
%	\end{lstlisting}
\end{ascboxJ}	

	\begin{ascboxJ}{4.开始编译Nginx,扩展编译模块}
列出Nginx的编译选项,如制定安装路径,配置文件、日志文件等路径,指定开启模块功能等
	\begin{lstlisting}[style=linux]
[root@localhost nginx-1.21.1]# ./configure --help

--help                             print this message

--prefix=PATH                      set installation prefix
--sbin-path=PATH                   set nginx binary pathname
--modules-path=PATH                set modules path
--conf-path=PATH                   set nginx.conf pathname
--error-log-path=PATH              set error log pathname
--pid-path=PATH                    set nginx.pid pathname
--lock-path=PATH                   set nginx.lock pathname
.........

#编译Nginx初步,
[root@localhost nginx-1.21.1]# ./configure --prefix=/opt/Learn_Nginx/nginx/ --with-http_ssl_module  --with-http_flv_module --with-http_gzip_static_m odule --with-http_stub_status_module  --with-threads  --with-file-aio
checking for OS
+ Linux 3.10.0-957.el7.x86_64 x86_64
checking for C compiler ... found
+ using GNU C compiler
+ gcc version: 4.8.5 20150623 (Red Hat 4.8.5-44) (GCC)
checking for gcc -pipe switch ... found
checking for -Wl,-E switch ... found


	\end{lstlisting}
\end{ascboxJ}

	\begin{ascboxJ}{5.执行make编译件}
	\begin{lstlisting}[style=linux]
[root@localhost nginx-1.21.1]# make
make -f objs/Makefile
make[1]: Entering directory `/opt/nginx-1.21.1'
	\end{lstlisting}
\end{ascboxJ}

	\begin{ascboxJ}{6.首次编译安装,生成Nginx的可执行命令}
	\begin{lstlisting}[style=linux]
[root@localhost nginx-1.21.1]# make install
make -f objs/Makefile install
make[1]: Entering directory `/opt/nginx-1.21.1'
test -d '/opt/Learn_Nginx/nginx/' || mkdir -p '/opt/Learn_Nginx/nginx/'
test -d '/opt/Learn_Nginx/nginx//sbin' \
	\end{lstlisting}
\end{ascboxJ}

	\begin{ascboxJ}{7.检查Prefix指定的安装目录}
	\begin{lstlisting}[style=linux]
[root@localhost opt]# ls /opt/Learn_Nginx/
nginx
	\end{lstlisting}
\end{ascboxJ}

	\begin{ascboxJ}{8.Nginx的程序目录}
	\begin{lstlisting}[style=linux]
[root@localhost nginx]# pwd
/opt/Learn_Nginx/nginx
[root@localhost nginx]# ls
conf  html  logs  sbin

依次是配置文件,静态文件,日志,二进制命令目录
	\end{lstlisting}
\end{ascboxJ}

	\begin{ascboxJ}{9.创建nginx的环境变量文件,修改如下,创建/etc/profile.d/nginx.sh脚本文件便于以后维护}
	\begin{lstlisting}[style=linux]
[root@localhost nginx]# cat /etc/profile.d/nginx.sh
export PATH=/opt/Learn_Nginx/nginx/sbin:$PATH
	\end{lstlisting}
\end{ascboxJ}

	\begin{ascboxJ}{10.退出会话,重新登录终端,此时可以正常使用nginx}
	\begin{lstlisting}[style=linux]
[root@localhost ~]# echo $PATH
/opt/Learn_Nginx/nginx/sbin:/usr/local/sbin:/usr/local/bin:/usr/sbin:/usr/bin:/root/.local/bin:/root/bin
	\end{lstlisting}
\end{ascboxJ}

	\begin{ascboxJ}{11.检查nginx的编译模块信息}
	\begin{lstlisting}[style=linux]
[root@localhost ~]# nginx -V
nginx version: nginx/1.21.1
built by gcc 4.8.5 20150623 (Red Hat 4.8.5-44) (GCC)
built with OpenSSL 1.0.2k-fips  26 Jan 2017
TLS SNI support enabled
configure arguments: --prefix=/opt/Learn_Nginx/nginx/ --with-http_ssl_module --with-http_flv_module --with-http_gzip_static_module --with-http_stub_status_module --with-threads --with-file-aio
	\end{lstlisting}
\end{ascboxJ}	
\end{ascolorbox10}




