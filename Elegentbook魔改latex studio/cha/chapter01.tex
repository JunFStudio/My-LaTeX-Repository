\chapter{Linux基础}
\begin{center}
	{\textcolor[RGB]{255, 0, 0}{\faHeart}~我想要和你们一起旅行。去看看生命的小溪到海之前,会经历的夹岸繁花与天际云霞。~\textcolor[RGB]{255, 0, 0}{\faHeart}}
	
	\pgfornament[width=0.36\linewidth,color=lsp]{88}
\end{center}
\section{计算机硬件的五大单元}
关于电脑的硬件组成部分,其实你可以观察你的台式机来分析一下,依外观来说
这家伙主要可分为三部分,分别是:

$\bullet$ 输入单元:包括键盘、鼠标、读卡机、扫描仪、手写板、触摸屏等等一堆;

$\bullet$主机部分:这个就是系统单元,被主机机箱保护住了,里面含有一堆板子、CPU 与
内存等;

$\bullet$输出单元:例如屏幕、打印机等等

\subsection{中央处理器}
整部主机的重点在于中央处理器 (Central Processing Unit, CPU),CPU\md{为一个具有特定功能的芯片, 里头含有微指令集},由于 CPU 的工作主要在于管理与运算,因此在 CPU 内又可分为两个主要的单元,分别是: \md{算数逻辑单元与控制单元。其中算数逻辑单元主要负责程序运算与逻辑判断,控制单元则主要在协调各周边组件与各单元间的工作}。

一般企业里的服务器,CPU 个(颗)数为 2-4 颗,单个(颗)CPU 是四核,内存总量一般是 16G-256G(32G, 64G)

做虚拟化的宿主机(eg:安装 vmware(虚拟化软件)的服务器),CPU 颗数 4-8 颗,内存总量一般是 48G-128G,6- 10 个虚拟机。

\subsection{内存}
CPU所使用的数据都是来自于主存储器(main memory),不论是软件程序还是数据,都必须要读入主存储器后CPU才能利用。 \md{个人计算机的主存储器主要组件为动态随机存取内存(Dynamic Random Access Memory, DRAM)}, 随机存取内存只有在通电时才能记录与使用,断电后数据就消失了。因此我们也称这种RAM为挥发性内存。

\md{以服务器来说,主存储器的容量有时比CPU的速度还要来的重要的!}


\begin{ascolorbox17}{内存数据提升用户体验}
\md{核心思想就是,由于内存特性,将数据放入内存读写,比磁盘要快的多。}

\begin{ascboxC}{门户(大网站 )极端案例}
大并发写入案例(抢红包、微博) 高并发、大数据量“写”数据:会把数据先写到内存,积累一定的量后,然后再定时或者定量的写到磁盘

(减轻磁盘的压力,减少磁盘 IO Input/Output 磁盘的输入/输出 磁盘读写),最终还是会把 数据加载到内存中再对外提供访问。
\begin{itemize}
	\item 优点: 写数据到内存,性能高速度快(微博,微信,SNS,秒杀)。
	\item 缺点:可能会丢失一部分在内存中还没有来得及存入磁盘的数据。 解决数据不丢的方法:
	\begin{itemize}
		\item 服务器主板上安装蓄电池,在断电瞬间把内存数据回写到磁盘。
		\item UPS(一组蓄电池)不间断供电(持续供电 10 分钟,IDC 数据中心机房-UPS 1 小时)。 UPS
		(Uninterruptible Power System/Uninterruptible Power Supply),即不间断电源,是将蓄电池(多 为铅酸免维护蓄电池)与主机相连接,通过主机逆变器等模块电路将直流电转换成市电的系统 设备。
		\item 选双路电的机房,使用双电源、分别接不同路的电,服务器要放到不同的机柜、地区。
	\end{itemize}
\end{itemize}
\end{ascboxC}
\begin{ascboxC}{中小企业案例}
对于并发不是很大、数据也不是特别大的网站,读多写少的业务,会先把数据写入到磁盘,然后再通过程序把写到磁盘的数据读入到内存里,再对外通过读内存提供访问服务。
\end{ascboxC}
\end{ascolorbox17}
\subsection{显卡}
显卡(Video card、Display card、Graphics card、Video adapter)是个人计算机基础的组成部分之一,将计算机系统需要的显示信息进行转换驱动显示器,并向显示器提供逐行或隔行扫描信号,控制显示器的正确显示,是连接显示器和个人计算机主板的重要组件,是“人机”的重要设备之一,其内置的并行计算能力现阶段也用于深度学习等运算。

\begin{dinglist}{118}
\item 是计算机中最重要的图像输出设备
\item 是将计算机系统所需要的显示信息进行转换驱动显示器,并向显示器提供逐行或隔行扫描信号,控制显示器的正确显示
\item 是连接显示器和个人计算机主板的重要组件
\item 是“人机对话”的重要设备之一
\end{dinglist}

\subsection{磁盘}
由于计算机在工作时,CPU、输入输出设备与存储器之间要大量地交换数据,因此存储器的 存取速度和容量也是影响计算机运行速度的主要因素之一。

\md{特别是在服务器优化场景,硬盘的性能 是决定网站性能的重要因素。}

磁盘就是永久存放数据的存储器,磁盘上也是有缓存的(芯片)。

常用的磁盘(硬盘)都是 3.5 英寸的(sas,sata),常规的机械硬盘,读取(性能不高)性能比内存 差很多,所以,在企业工作中,我们才会把大量的数据缓存到内存,写入到缓冲区,这是当今互联 网网站必备的解决网站访问速度慢的方案。

目前常用的硬盘分为机械硬盘和固态硬盘两种,相比来说,固态硬盘速度快但是容量较小,价格高;

机械硬盘速度慢但是容量大,价格便宜。

\begin{dinglist}{104}
\item 磁盘的接口:IDE,SCSI,SAS,SATA,IDE(SCSI 退出历史舞台)

\item 磁盘的类型:机械磁盘和 ssd 固态硬盘

\item 性能与价格:SSD(固态)>SAS> SATA
\end{dinglist}

Raid卡:磁盘阵列(Redundant Arrays of Independent Drives,RAID),有“独立磁盘构成的具有冗余能力的阵列”之意。

磁盘阵列是由很多块独立的磁盘,组合成一个容量巨大的磁盘组,利用个别磁盘提供数据所产生加成效果提升整个磁盘系统效能。

利用这项技术,将数据切割成许多区段,分别存放在各个硬盘上。

磁盘阵列还能利用同位检查(Parity Check)的观念,在数组中任意一个硬盘故障时,仍可读出数据,在数据重构时,将数据经计算后重新置入新硬盘中。

你有很多土地,单独管理不方便,且效率很低,整合到一块,统一管理,发挥最大性能。

互联网公司一般都会购买Raid卡

\begin{dinglist}{104}
\item 冗余从好到坏:raid1、raid10、raid5、raid0

\item 性能从好到坏:raid0、raid10、raid5、raid1

\item 成本从低到高:raid0、raid5、raid1、raid10

\item 单台服务器,很重要,盘不多,系统盘 raid1。
\item 数据库/存储服务器,主库 raid10,从库 raid5\ raid0(为了维护成本,raid10)
\item web 服务器,如果没有太多数据的话,raid5,raid0(单盘)
\item 有多台,监控/应用服务器,raid0,raid5。
\end{dinglist}

\subsection{主板}
主板是计算机中最重要的平台部件,也是电脑中最大的集成电路板,它直接或间接的将所有的设备连接在一起。主板的好坏直接决定了计算机速度的快慢和运行稳定。

同时主板也提供了大量的设备接口,为计算机扩展功能提供了可能。

主板一般为矩形电路板,上面安装了组成计算机的主要电路系统,一般有BIOS芯片、I/O控制芯片、键和面板控制开关接口、指示灯插接件、扩充插槽、主板及插卡的直流电源供电接插件等元件。

现在主板一般情况下都集成了三卡(显卡、网卡、声卡),也有的只集成了声卡和网卡。
\subsection{电源}
机箱用来装载计算机硬件,对硬件起到防尘,保护的作用,也有相应的防静电等作用

1)抗静电

2)机箱质量

3)机箱散热

4)机箱质量不易变形

5)机箱空间能满足扩展需求
\begin{ascolorbox11}{电源供应器(Power)}
在机箱内,有一个大大的特盒子,包含着很多电源线,这个就是电源供应器了。

计算机硬件中的CPU、内存、主板、硬盘等等都必须得供电方可使用,随着硬件性能逐步提升,性能较差的电源很可能造成供电不足,导致内存数据丢失等等问题。
\end{ascolorbox11}
\begin{dinglist}{104}
\item 服务器电源就是指使用在服务器上的电源(POWER),它和 PC(个人电脑)电源一样,都是 一种开关电源。
\item 服务器电源按照标准可以分为 ATX 电源和 SSI 电源两种。ATX 标准使用较为普遍,主要用于台 式机、工作站和低端服务器;而 SSI 标准是随着服务器技术的发展而产生的,适用于各种档次的服务器。
\item 服务器电源相当于人体的心脏,保障电源供应,要选择质量好的电源。
\item 生产中一般单个服务器核心业务最好使用双电源 AB 线路。
\item 如果集群(一堆机器做一件事)的情况可以不用双电源。
\end{dinglist}
\begin{ascolorbox11}{UPS不间断电源}
UPS(Uninterruptible Power System/Uninterruptible Power Supply),即不间断电源,是将蓄电池(多为铅酸免维护蓄电池)与主机相连接,通过主机逆变器等模块电路将直流电转换成市电的系统设备。

主要用于给单台计算机、计算机网络系统或其它电力电子设备如电磁阀、压力变送器等提供稳定、不间断的电力供应。

当市电输入正常时,UPS 将市电稳压后供应给负载使用,此时的UPS就是一台交流式电稳压器,同时它还向机内电池充电;

当市电中断(事故停电)时, UPS 立即将电池的直流电能,通过逆变器切换转换的方法向负载继续供应220V交流电,使负载维持正常工作并保护负载软、硬件不受损坏。

UPS 设备通常对电压过高或电压过低都能提供保护。
\end{ascolorbox11}

\subsection{服务器}
服务器也就是台计算机而已,同样的由CPU、主板、内存、磁盘、网卡等硬件组成。

不同的是,服务器的定义是\md{高性能计算机},作为网络中的节点,处理网络通信中的数据、信息,是网络时代的根本灵魂。

服务器通常指\md{一个管理资源且为用户提供服务的计算机},通常服务器分为\md{文件服务器、数据库服务器、应用程序服务器}。

服务器对比普通PC、\md{稳定性、安全性、性能、可扩展性、可管理性}等方面要求更高。

服务器对于屏幕显示的要求很低,基本上都是无显示器,通过远程管理的方式即可,因此服务器基本都是集成显卡,而无需单独装显卡。

我们很难见识到真实的物理服务器,因为服务器一般都防止在机房托管,闲人免进,比如appe.com苹果公司网站的数据就放在了 云上贵州的服务器机房。
\begin{ascolorbox11}{服务器分类}
\begin{itemize}
\item 按规模分:
	\begin{itemize}
		\item 大型服务器,计算中心、企业级
		\item 中级服务器,公司部分级
		\item 小型服务器,入门级服务器,个人云服务器
	\end{itemize}
\item 按用途分:
	\begin{itemize}
	\item web服务器
	\item 数据库服务器
	\item 文件服务器
	\item 邮件服务器
	\item 视频点播服务器
\end{itemize}
\item 服务器以外形分类
\item 按用途分:
\begin{itemize}
	\item 机架式服务器
	\begin{itemize}
		\item 机架式服务器的外形看来不像计算机,而像“抽屉”,有 1U、2U、4U 等规格。
		\item 机架式服务器安装在标准的 19 英寸机柜里面。这种结构的多为功能型服务器。
	\end{itemize}
	\item 刀片式服务器
		\begin{itemize}
		\item 刀片式服务器是指在标准高度的机架式机箱内可插装多个卡式的服务器单元,实现高可用和高密度。
		\item 打个形象的比喻,刀片式服务器就像是箱子里摆放整齐的书。
		
		每一块"刀片"实际上就是一块系统主板。
		
		它们可以通过"板载"硬盘启动自己的操作系统,如Windows NT/2000、Linux·等,类似于一个个独立的服务器,在这种模式下,每一块母板独立运 行自己的系统,服务于指定的不同用户群,相互之间没有关联,因此相较于机架式服务器和机 柜式服务器,单片母板的性能较低。
		
		不过,管理员可以使用系统软件将这些母板集合成一个服 务器集群。在集群模式下,所有的母板可以连接起来提供高速的网络环境,并同时共享资源, 为相同的用户群服务。在集群中插入新的"刀片",就可以提高整体性能。而由于每块"刀片"都是热插拔的,所以,系统可以轻松地进行替换,并且将维护时间减少到最小。
	\end{itemize}
	\item 塔式服务器-更强壮的服务器
		\begin{itemize}
	\item 塔式服务器(Tower Server)应该是最容易理解的一种服务器结构类型。
	\item 因为它的外形以及结构都跟立式 PC 差不多,当然,由于服务器的主板扩展性较强、插槽也多出一堆,所以个头比普通主板大一些,因此塔式服务器的主机机箱也比标准的 ATX 机箱要大,一般都会预留足够的内部空间以便日后进行硬盘和电源的冗余扩展。
	\item 但这种类型服务器也有不少局限性,在需要采用多台服务器同时工作以满足较高的服务器应用需求时,由于其个体比较大,占用空间多,也不方便管理,便显得很不适合。
\end{itemize}
\end{itemize}
\end{itemize}
\end{ascolorbox11}


\begin{ascolorbox16}{服务器分类}
首先,为什么说要将服务器放入机房而不是直接放在办公室或企业小机房,有以下几个原因:
\begin{itemize}
	\item 1、企业的机房无法保证365天7*24小时都供电充足;
	
	\item 2、企业的机房无硬件防护,病毒容易入侵;
	
	\item 3、企业的机房接入的宽带或光纤是经过分流的民用带宽,速度慢;
	
	\item 4、企业必须以较高成本雇佣较高技术能力的工程师进行长期维护;
	
	\item 5、企业无法为服务器提供一个真正的机房运营环境,服务器使用寿命会缩短,并且容易出现故障,造成数据流失或损毁。
\end{itemize}
那么,真正的数据机房正是为了服务器更好、更稳、更快、更安全运行而建设的,IDC数据中心服务器托管业务它能提供更适合服务器运行的环境,能提供更强有力的安全保障,能提供更高效的带宽资源。

其次,在当下机房林立的IDC环境中,选择哪些机房做服务器托管会更安全,性价比更高呢?
\begin{itemize}
\item 	1、专业的电信或联通或双线机房更能保证稳定;	
\item 	2、位于国家CHINA NET骨干网上的机房更能保证速度;
\item 	3、技术和业务口碑都比较好的机房更能提供好的技术服务和安全防护
\end{itemize}
\end{ascolorbox16}


\begin{ascolorbox17}{云服务器}

\begin{itemize}
	\item 1、云服务器操作及升级更方便
	
	传统服务器中的资源都是有限的,如果想要获得更好的技能,只能升级云服务器,所谓“云”,就是网络、互联网的意思,云服务器就是一种简单高效、安全可靠、处理能力可弹性伸缩的计算服务。其操作起来更加简便,如果原来使用的配置过低,完全可以在不重装系统的情况下升级CPU、硬盘、内存等,不会影响之前的使用。
	
	\item 2、云服务器的访问速度更快
	
	云服务器又叫云主机。其使用的带宽通常是多线互通,网络能够自动检测出那种网络速度更快,并自动切换至相对应的网络上进行数据传输。
	
	\item 3、云服务器的存储更便捷
	
	云服务器上能够进行数据备份,因此即使是硬件出现问题,其数据也不会丢失。并且,使用云服务器只需要服务商后期正常维护就可以了,为企业解决了很多后顾之忧。
	
	\item 4、云服务器安全稳定
	
	云服务器是一种集群式的服务器,可以虚拟出多个类似独立服务器的部分,具有很高的安全稳定性。而且云服务器是支持异节点快速重建的,即使计算节点异常中断或损坏,也可以在极短时间内通过其他不同节点重建虚拟机,且不影响数据完整。
	
	\item 5、云服务器有更高的性价比
	
	云服务器是按需付费的,与传统服务器相比,具有更高的性价比,而且并不会造成资源浪费。
	
	当然,除开这些显著特点以外,更重要的是要选择一个知名的服务商,这样云服务器才能更加简便高效,不会给企业带来不必要的损失。
\end{itemize}
\end{ascolorbox17}
远程管理卡是安装在服务器上的硬件设备,提供一个以太网接口,使它可以连接到局域网内,提供远程访问。

\section{软件基本知识}
于计算机是通过电位记录信息的,因此仅能识别 0 和 1 这两个数字,故而在计算机内部,数据都 是以二进制的形式存储和运算的,下面列出来计算机数据的常用计量单位。

二进制,是计算技术中广泛采用的一种数制,由德国数理哲学大师莱布尼茨于1679年发明。

二进制数据是用0和1两个数码来表示的数。它的基数为2,进位规则是“逢二进一”,借位规则是“借一当二”。

当前的计算机系统使用的基本上是二进制系统,数据在计算机中主要是以补码的形式存储的。

计算机中的二进制则是一个非常微小的开关,用“开”来表示1,“关”来表示0。
\begin{itemize}
	\item 1.位(bit) 计算机存储数据的最小单位为位(bit),中文称为比特,一个二进制位表示 0 或 1 两种状态,要表 示更多的信息,就要把多个位组合成一个整体,一般以 8 位二进制数组成一个基本单位。
	\item 2.字节(Byte) 字节是计算机数据处理的基本单位。字节(Byte)简记为 B,规定一个字节为 8 位,即 1B=8b 每个 字节都是由 8 个二进制位组成。
\end{itemize}

\begin{lstlisting}[style=text]
1Bite=8bit,1KB=1024B,1MB=1024KB,1GB=1024MB
1TB=1024GB,1PB=1024TB,1EB=1024PB,1ZB=1024EB
\end{lstlisting}	

\section{LINUX 系统介绍与环境搭建准备}
\subsection{操作系统}
操作系统,英文名称 Operating System,简称 OS,是计算机系统中必不可少的基础系统软件,它是 应用程序运行以及用户操作必备的基础环境支撑,是计算机系统的核心。

操作系统的作用是管理和控制计算机系统中的硬件和软件资源,例如,它负责直接管理计算机系统 的各种硬件资源,如对 CPU、内存、磁盘等的管理,同时对系统资源供需的优先次序进行管理。

操 作系统还可以控制设备的输入、输出以及操作网络与管理文件系统等事务。

同时,它也负责对计算 机系统中各类软件资源的管理。例如各类应用软件的安装、运行环境设置等。下图给出了操作系统 与计算机硬件、软件之间的关系示意图。

\md{操作系统就是处于用户与计算机硬件之间用于传递信息的系统程序软件。}

操作系统在接收到用户输入后,将其传递给计算机系统硬件核心进行处理,然后再讲计算机硬件的处理结果返回给用户。

\subsection{什么是Linux}
Linux类似Windows,也就是款操作系统软件

Linux是一套开放源代码程序的、可以自由传播的类Unix操作系统软件,且支持多用户、多任务且支持多线程、多CPU的操作系统。

Linux主要用在服务器端、嵌入式开发和个人PC桌面中,服务器端是重中之重。

我们熟知的大型、超大型互联网企业(百度,Sina,淘宝等)都在使用 Linux 系统作为其服务器端的程序运行平台,全球及国内排名前十的网站使用的主流系统几乎都是 Linux 系统。

从上面的内容可以看出,Linux 操作系统之所以如此流行,是因为它具有如下一些特点:
\begin{dinglist}{118}
\item 是开放源代码的程序软件,可自由修改;
\item Unix系统兼容,具备几乎所有Unix的优秀特性;
\item 可自由传播,无任何商业化版权制约;
\item 适合 Intel 等 x86 CPU 系列架构的计算机,可移植性很高
\end{dinglist}

\subsection{Unix操作系统}
Unix系统在1969年的AT\&T的贝尔实验室诞生,20世纪70年代,它逐步盛行,这期间,又产生 了一个比较重要的分支,就是大约 1977 年诞生的 BSD(Berkeley Software Distribution)系统。

从BSD 系统开始,各大厂商及商业公司开始了根据自身公司的硬件架构,并以 BSD 系统为基础进行Unix 系统的研发,从而产生了各种版本的 Unix 系统

\begin{ascolorbox17}{Unix的五大优势}
\begin{enumerate}
	\item 技术成熟、可靠性高
	\item 可伸缩性,Unix 支持的 CPU 处理器体系架构非常多,包括 Intel/AMD 及 HP-PA、MIPS、PowerPC、UltraSPARC、ALPHA 等 RISC 芯片,以及 SMP、MPP 等技术。
	\item 强大的网络功能,Internet 互联最重要的协议 TCP/IP 就是在 Unix 上开发和发展起来的。此外,Unix 还支持非常多的 常用的网络通信协议,如 NFS、DCE、IPX/SPX、SLIP、PPP 等。
	\item 强大的数据库能力,Oracle、DB2、Sybase、Informix 等大型数据库,都把 \item Unix 作为其主要的数据库开发和运行平台, 一直到目前为止,依然如此。
	\item 强大的开发性,促使C语言诞生
\end{enumerate}
\end{ascolorbox17}

\begin{ascolorbox17}{Unix操作系统的革命}
	\begin{enumerate}
		\item 70 年代中后期,由于各厂商及商业公司开发的 Unix 及内置软件都是针对自己公司特定硬件的,因此在其他公司的硬件上基本上无法直接运行。
		\item 70年代末,Unix又面临了突如其来的被AT\&T回收版权的重大问题,特别是要求\md{禁止对学生群体提供Unix系统源码。}
		\item 在80年代初期,同样是之前Unix系统版权和源代码限制的问题,使得大学授课Unix系统束缚很多,因此,一位名为Andrew Tanenbaum(谭宁邦)的大学教授为了教学开发了Minix操作系统。
		\item 1984年,\md{Richard Stallman斯托曼发起了开发自由软件的运动,且成立自有软件基金会(Free Software Foundation,FSF)和GNU项目}
	\end{enumerate}
\end{ascolorbox17}

\begin{ascboxY}{GNU项目}
当时发起这个自由软件运动和创建 GNU 项目的目的其实很简单,就是想开发一个类似 Unix 系统、 并且是自由软件的完整操作系统,也就是要解决 70 年代末 Unix 版权问题以及软件源代码面临闭源的问题,

这个系统叫做GNU 操作系统。

这个 GNU 系统后来没有流行起来。现在的 GNU 系统通常是使用 Linux 系统的内核, 以及使用了GNU项目贡献的一些组件加上其它相关程序组成,这样的组合被称为 GNU/Linux操作 系统。
\end{ascboxY}

\subsection{Linux系统发展历程}
\begin{ascolorbox5}{Linux系统发展历程}
\begin{itemize}
	\item 1)1984 年,Andrew S. Tanenbaum 开发了用于教学的 Unix 系统,命名为 MINIX。
	
	\item 	2)1989 年,Andrew S. Tanenbaum 将 MINIX 系统运行于 x86 的 PC 计算机平台。
	
	\item 	3)1990年,芬兰赫尔辛基大学学生LinusTorvalds首次接触MINIX系统。
	
	\item 	4)1991年,LinusTorvalds开始在MINIX上编写各种驱动程序等操作系统内核组件。
	
	\item 	5)1991 年底,Linus Torvalds 公开了 Linux 内核源码 0.02 版(http://www.kernel.org),注意,这
		里公开的 Linux 内核源码并不是我们现在使用的 Linux系统的全部,而仅仅是 Linux 内核 kernel
		部分的代码。
	
	\item 	6) 1993 年,Linux 1.0 版发行,Linux 转向 GPL 版权协议。
	
	\item 	7) 1994 年,Linux 的第一个商业发行版 Slackware 问世。
	
	\item 	8) 1996 年,美国国家标准技术局的计算机系统实验室确认Linux版本 1.2.13 (由 Open Linux公司打包)符合 POSIX 标准。
	
	\item 	9) 1999 年,Linux 的简体中文发行版问世。
	
	\item 	10) 2000 年后,Linux 系统日趋成熟,涌现大量基于 Linux 服务器平台的应用,并广泛应用于基于 ARM 技术的嵌入式系统中。
\end{itemize}
\end{ascolorbox5}

\subsection{Linux 核心概念知识}
关于Linux 核心概念知识有如下几点:
\begin{ascboxB}{自由软件}
	自由软件的核心就是没有商业化软件版权制约,源代码开放,可无约束自由传播。
\end{ascboxB}

\begin{ascboxD}{自由软件基金会FSF}
	FSF(Free Software Foundation)的中文意思是自由软件基金会,是 Richard Stallman于 1984年发起和创办的。
	
	FSF 的主要项目是 GNU 项目。
	
	GNU 项目本身产生的主要软件包括:Emacs 编辑软件、gcc 编译软件、bash命令解释程序和编程语言,以及 gawk (GNU’s awk)等。
\end{ascboxD}

\begin{ascboxD}{GNU知识}
	GNU,GNU 计划,又称革奴计划,是由Richard Stallman 在 1984 年公开发起的,是 FSF 的主要项目。前面已经提到过,这个项目的目标是
	\md{建立一套完全自由的和可移植的类 Unix 操作系统。}
	
	但是 GNU 自己的内核 Hurd 仍在开发中,离实用还有一定的距离。
	
	现在的 GNU 系统通常是使用 Linux 系统的内核、加上 GNU 项目贡献的一些组件,以及其他相关程 序组成的,这样的组合被称为 GNU/Linux 操作系统。
	
	到 1991 年 Linux 内核发布的时候,GNU 项目已经完成了除系统内核之外的各种必备软件的开发。
	
	在 Linus Torvalds 和其他开发人员的努力下, GNU 项目的部分组件又运行到了 Linux 内核之上,例 如:GNU 项目里的 Emacs、gcc、bash、gawk 等,至今都是 Linux 系统中很重要的基础软件。
\end{ascboxD}

\begin{ascboxD}{GPL知识}
	GPL 全称为General Public License,中文名为通用公共许可,是一个最著名的开源许可协议,开源社区最著名的 Linux 内核就是在 GPL 许可下发布的。
	
	GPL 许可是由自由软件基金会(Free Software foundation)创建的。
	
	1984 年,Richard Stallman 发起开发自由软件的运动后不久,在其他人的协作下,他创立了通用公共许可证(GPL),这对推动自由软件的发展起了至关重要的作用,那么,这个 GPL 到底是什么意思呢?
	
	\md{GPL许可的核心,是保证任何人有共享和修改自由软件的自由权利,任何人有权取得、修改和重新发布自由软件的源代码权利,但是必须同时给出具体更改的源代码。}
\end{ascboxD}


\begin{ascboxD}{重点回顾}
	FSF 自由软件基金会(公司)==> GNU(项目)==> emacs gcc bash(命令解释器) gawk
	
	FSF(公司)===>GPL(员工守则)==>自由传播 修改源代码 但是必须把修改后也要发布出来。
	Linus Torvalds==>linux 内核
	
	\md{Linux 操作系统=linux 内核+GNU 软件及系统软件+必要的应用程序}
\end{ascboxD}

\section{Centos7安装}
VMware虚拟机常见的网络类型有bridged(桥接)、**NAT**(地址转换)、host-only(仅主机)3种,在分析如何选择之前,先要简单和大家介绍下这三种网络类型。

\subsection{NAT(地址转换)}
NAT(Network Address Translation),网络地址转换,NAT模式是比较简单的实现虚拟机上网的方式,简单的理解,NAT模式的虚拟机就是通过宿主机(物理电脑)上网和交换数据的。

在NAT 模式下,虚拟机的网卡连接到宿主机的 VMnet8 上。

此时系统的 VMWare NAT Service 服务就充当了路由器, 负责将虚拟机发到 VMnet8 的包进行地址转换之后发到实际的网络上,再将实际网络上返回的包进行地址转换后通 过 VMnet8 发送给虚拟机。

VMWare DHCP Service 负责为虚拟机分配 IP 地址。NAT 网络类型的原理逻辑图如图所示。

NAT 网络特别适合于家庭里电脑直接连接网线的情况,当然办公室的局域网环境也是适合的,优势就是不会和其他物理主机 IP 冲突,且在没有路由器的环境下也可以通过 SSH NAT 连接虚拟机学习,换了网络环境虚拟机 IP 等不影响,这是推荐的选择。

\subsection{Bridged(桥接模式)}

桥接模式可以简单理解为通过物理主机网卡架设了一座桥,从而连入到了实际的网络中。

因此,虚拟机可以被分配与物理主机相同网段的独立IP,所有网络功能和网络中的真实机器几乎完全一样。

桥接模式下的虚拟机和网内真实计算机所处的位置是一样的。

在 Bridged 模式下,电脑设备创建的虚拟机就像一台真正的计算机一样,它会直接连接到实际的网络上,逻辑上网与宿主机(电脑设备)没有联系。

Bridged网络类型适合的场景:
\begin{dinglist}{118}
\item 特别适合于局域网环境,优势是虚拟机像一台真正的主机一样
\item 缺点是可能会和其他物理主机 IP 冲突,并且在和宿主机交换数据时,都会经过实际的路由器,当不考虑 NAT 模式的时候,就选这个桥接模式,桥接模式下换了网络环境后所有虚拟机的 IP 都会受影响。
\end{dinglist}
\subsection{Host-only(仅主机)}
在 Host-only 模式下,虚拟机的网卡会连接到宿主的 VMnet1上,但宿主系统并不为虚拟机提供任何路由服务,因此虚拟机只能和宿主机进行通信,不能连接到实际网络上,即无法上网。

\subsection{磁盘分区}
常规分区方案
\begin{ascboxD}{企业生产场景中 Linux 系统的分区方案}
如果根据Red Hat的建议,他们建议是分配RAM 20\%的换空间,也就是RAM是8GB,分配1.6GB交换空间。

CentOS建议

如果RAM小于2GB,就分配和RAM同等大小的Swap交换空间。

如果RAM大于2GB,就分配2GB交换空间

Ubuntu考虑到系统需要休眠,

如果RAM小于1GB,Swap空间至少要和RAM一样大,甚至是要为RAM的两倍大小

如果RAM大于1GB,Swap交换空间应该至少等于RAM大小的平方根,并且最多为RAM大小的两倍

如果要休眠,Swap交换大小应该等于RAM的大小加上RAM大小的平方根
\end{ascboxD}

\begin{ascboxD}{常见网络集群架构中的节点服务器(多个功能一样的服务器,服务器数据有多份)分区方案}
	\begin{itemize}
		\item /boot分区:存放引导程序,centos-6 给200M
		\item swap:虚拟内存
	\begin{itemize}
		\item 物理内存 < 8G ,swap分配 内存*1.5数量
		
		\item 物理内存 > 8G,swap就给8G
		
	\end{itemize}
		\item 
		/ 根目录,存放所有数据,剩余空间都给根目录(/usr,/home,/var等分区共用/目录,如同c盘下的系统文件夹)
	\end{itemize}
\end{ascboxD}

\begin{ascboxD}{数据库角色的服务器,有大量数据需要访问(重要数据单独分区,便于备份和管理)}
	\begin{itemize}
		\item /boot :存放引导程序,CentOS6分配200M,centOS7分配200M
		\item swap:虚拟内存
		\begin{itemize}
			\item 物理内存 < 8G ,swap分配 8*1.5数量
			
			\item 物理内存 > 8G,swap就给8G
			
		\end{itemize}
		\item /:根目录,50-200G,只存放系统相关文件,不存放数据文件
		\item /data:剩余硬盘空间全部给/data
	\end{itemize}
\end{ascboxD}

\begin{ascboxD}{大型门户网站,大型企业分区思路}
	\begin{itemize}
		\item /boot:存放引导程序,CentOS6 给 200M,CentOS7 给 200M
		\item swap:虚拟内存
		\begin{itemize}
			\item 物理内存 < 8G ,swap分配 8*1.5数量
			
			\item 物理内存 > 8G,swap就给8G
			
		\end{itemize}
		\item / 根目录,50-200G,放系统相关文件
		\item 剩余磁盘空间,保留,由业务需求决定分区
	\end{itemize}
\end{ascboxD}

LVM性能差

操作系统自带软RAID不用,性能差、没有冗余,生产环境用硬件raid

除了/boot、swap 和/三个分区外,还可以加/usr、/home、/var 等分区,具体要根据服务器的需求来决 定,一般情况下,只配置这三个分区足够了。

这种分区方案最大优点就是简单,使用方便,可批量安装部署,而且不会存在有的分区满了,有的 分区还剩余了很多空间又不能被利用的情况(LVM 的情况这里先不阐述)。

该分区方案的缺点是如果系统坏了,重新装系统时,因为数据都在/(根)分区,导致数据备份很麻 烦,如果设置了/usr、/home、/var 等分区,即使系统出了故障,也可以直接在/(根)分区装系统, 这样并不会破坏其他分区的数据。

当然,刚才也说了,如果是不存在备份数据的集群节点,那采用 这种分区方案是很明智的,不需要特别担心某个分区暴满的问题。




