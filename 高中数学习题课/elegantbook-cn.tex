\documentclass[lang=cn, zihao=4.5]{elegantbook}
\usepackage{hyperref}

% font settings


% watermark settings
%\usepackage{ctex, draftwatermark, everypage}
%	\SetWatermarkText{DEEP Team 讲义模版}
%	\SetWatermarkLightness{0.95}
%	\SetWatermarkScale{0.3}

% customised commands
\usepackage{ulem}
	\newcommand{\tk}{\uline{\hspace{4em}}}
	
\DeclareSymbolFont{yh}{OMX}{yhex}{m}{n}
\DeclareMathAccent{\hu}{\mathord}{yh}{"F3}

\newcommand{\xl}[1]{\overrightarrow{#1}}
\newcommand{\nd}[1]{〔#1〕}
\newcommand{\ssb}[1]{\left( #1 \right)}
\newcommand{\sw}[1]{\boxed{\text{解法 #1}} \ }
\newcommand{\buzhou}[1]{$#1^{\circ} \ $}

% cover settings

\title{高中数学习题册}
\subtitle

\author{Johnny Tang}
\institute{DEEP Team}
\date{January 21, 2022}

\extrainfo{请:相信时间的力量,敬畏概率的准则}


\cover{cover.png}

% 本文档命令


% 修改标题页的橙色带
% \definecolor{customcolor}{RGB}{32,178,170}
% \colorlet{coverlinecolor}{customcolor}


\begin{document}

\maketitle

\frontmatter

\mainmatter

\tableofcontents

\newpage

\part{一试与强基部分}

\chapter{集合}

\section{集合及其运算}

\subsection*{填空题}

\begin{example} % 启东中学奥赛教程 p6
	设集合$M=\{ -1,0,1 \} , ~N=\{ 2,3,4,5,6 \}$,映射$f:M \to N$,则对任意的$x \in M$,使得$x + f(x) +xf(x)$恒为奇数的映射$f$的个数为\tk .
\end{example}
\begin{hint}
	分类讨论.
\end{hint}

\begin{example}
	称有限集$S$的所有元素的乘积为$S$的“积数”,给定数集$M= \{ \dfrac{1}{2},\dfrac{1}{3}, \cdots ,\dfrac{1}{100} \}$,则集合$M$的所有含偶数个元素的子集的“积数”之和为\tk .
\end{example}
\begin{hint}
	举例分析.
\end{hint}

\subsection*{解答题}

\begin{example} % 启东中学奥赛教程 p4
	(2015高联)设$a_1,a_2,a_3,a_4$是$4$个有理数,使得$\{ a_ia_j | 1 \leq i < j \leq 4 \} = \{ -24,-2,-\dfrac{3}{2},-\dfrac{1}{8},1,3 \}$.求$a_1+a_2+a_3+a_4$的值.
\end{example}
\begin{hint}
	通过大小关系将$a_1a_2,a_1a_3,a_1a_4,a_2a_3,a_2a_4,a_3a_4$与这六个数字对应.
\end{hint}

\begin{example} % 启东中学奥赛教程 p4
	(2017清华THUSSAT)已知集合$A= \{ a_1,a_2,a_3,a_4 \}$,且$a_1 < a_2 < a_3 < a_4$,$a_i \in \mathbb{N} ^* ~(i=1,2,3,4)$.记$a_1+a_2+a_3+a_4=S$,集合$B = \{ (a_i,a_j) : (a_i+a_j) | S, a_i,a_j \in A, i<j \}$中的元素个数为$4$个,求$a_1$的值.
\end{example}
\begin{hint}
	通过大小关系得出不能被$S$整除的两项.
\end{hint}

\begin{example} % 启东中学奥赛教程 p5
	$X$是非空的正整数集合,满足下列条件:(i)若$x \in X$,则$4x \in X$;(ii)若$x \in X$,则$[\sqrt{x}] \in X$.求证:$X$是全体正整数的集合.
\end{example}
\begin{hint}
	将两种关于$X$的性质结合起来看.
\end{hint}

\begin{example} % 启东中学奥赛教程 p5
	设$S$为非空数集,且满足:(i)$2 \notin S$;(ii)若$a \in S$,则$\dfrac{1}{2-a} \in S$.证明: \\
	(1)对一切$n \in \mathbb{N} ^{*} ~, n \geq 3$,有$\dfrac{n}{n-1} \notin S$;(2)$S$或者是单元素集,或者是无限集.
\end{example}
\begin{hint}
	数学归纳法.
\end{hint}

\begin{example} % 启东中学奥赛教程 p6
	以某些整数为元素的集合$P$具有下列性质:(i)$P$中的元素有正数,有负数;(ii)$P$中的元素有奇数,有偶数;(iii)$-1 \notin P$;(iv)若$x,y \in P$,则$x+y \in P$.试证明: \\
	(1)$0 \in P$;(2)$2 \notin P$.
\end{example}
\begin{hint}
	第一问:构造;第二问:反证法.
\end{hint}

\begin{example} % 启东中学奥赛教程 p6
	已知数集$A$具有以下性质:(i)$0 \in A,1 \in A$;(ii)若$x,y \in A$,则$x-y \in A$;(iii)若$x \in A, ~x \neq 0$,则$\dfrac{1}{x} \in A$. \\
	求证:当$x,y \in A$时,则$xy \in A$.
\end{example}
\begin{hint}
	只需证明$\dfrac{1}{xy} \in A$,然后构造.
\end{hint}

\newpage
\section{集合元素的个数}

\begin{theorem}{容斥原理1——容斥公式}
	设$A_i$($i=1,2, \cdots ,n$)为有限集,则
	$$|\bigcup_{i=1}^{n} A_i| = \sum_{i=1}^{n} |A_i|-\sum_{1 \leq i < j \leq n}|A_i \cap A_j| + \cdots + (-1)^{n-1} |\bigcap_{i=1}^{n} A_i|$$
	可以使用数学归纳法证明.
\end{theorem}

\begin{theorem}{容斥原理2——筛法公式}
	设$A_i$($i=1,2, \cdots ,n$)为全集$I$的子集,则
	$$|\bigcap_{i=1}^{n} \complement _{I} A_i| = |I| - \sum_{i=1}^{n} |A_i|+\sum_{1 \leq i < j \leq n}|A_i \cap A_j| - \cdots + (-1)^{n} |\bigcap_{i=1}^{n} A_i|$$
	可以通过摩根律证明.这个公式常常用来计算不满足任意给定性质的子集个数.
\end{theorem}

\subsection*{填空题}

\begin{example} % 启东中学奥赛教程 p8
	设$\{ b_n \}$是集合$\{ 2^t+2^s+2^r | 0 \leq r < s < t, ~r,s,t \in \mathbb{Z} \}$中所有的数从小到大排列成的数列,已知$b_k = 1160$,则$k$的值为\tk .
\end{example}
\begin{hint}
	分段考虑.
\end{hint}

\begin{example} % 启东中学奥赛教程 p12
	$A=\{ z|z^{18}=1 \},~ B=\{ w|w^{48}=1 \}$都是$1$的复单位根的集合,$C=\{ zw|z \in A,~ w \in B \}$也是$1$的复单位根的集合.则集合$C$中含有元素的个数为\tk .
\end{example}
\begin{hint}
	复数的三角表示.
\end{hint}

\begin{example} % 启东中学奥赛教程 p12
	已知集合$\{ 1,2, \cdots ,3n \}$可以分为$n$个互不相交的三元组$\{ x,y,z \}$,其中$x+y=3z$,则满足上述要求的两个最小的正整数$n$是\tk .
\end{example}
\begin{hint}
	从条件$x+y=3z$入手变形消元.
\end{hint}

\begin{example} % 启东中学奥赛教程 p12
	集合$M= \{ x|\cos{x} + \lg \sin{x} = 1 \}$中元素的个数是\tk .
\end{example}
\begin{hint}
	有没有可能无解?
\end{hint}

\subsection*{解答题}

\begin{example} % 启东中学奥赛教程 p12
	设集合$M = \{ 1,2, \cdots ,1995 \}$,$A$是$M$的子集且满足条件:当$x \in A$时,$15x \notin A$,求$A$中元素个数的最大值.
\end{example}
\begin{hint}
	先构造最大值情况,再证明这是最大值.
\end{hint}

\begin{example} % 启东中学奥赛教程 p12
	求最大的正整数$n$,使得$n$元集合$S$同时满足:(i)$S$中的每个数均为不超过$2002$的正整数;(ii)对于$S$的两个元素$a$和$b$(可以相同),它们的乘积$ab$不属于$S$.
\end{example}
\begin{hint}
	先构造最大值情况,再证明这是最大值.
\end{hint}

\begin{example} % 启东中学奥赛教程 p12
	我们称一个正整数的集合$A$是“一致”的,是指:删除$A$中任何一个元素之后,剩余的元素可以分成两个不相交的子集,而且这两个子集的元素之和相等.求最小的正整数$n$($n>1$),使得可以找到一个具有$n$的元素的“一致”集合$A$.
\end{example}
\begin{hint}
	将$A$中元素分奇偶讨论.
\end{hint}

\begin{example} % 启东中学奥赛教程 p12
	设$n$是正整数,我们说集合$\{ 1,2, \cdots ,2n \}$的一个排列$(x_1,x_2, \cdots ,x_{2n})$具有性质$P$,是指在$\{ 1,2, \cdots ,2n-1 \}$中至少有一个$i$,使得$|x_i-x_{i+1}|=n$,求证:对于任何$n$,具有性质$P$的排列比不具有性质$P$的排列的个数多.
\end{example}
\begin{hint}
	只需证明具有性质$P$的排列个数大于全部排列数的一半.利用容斥原理放缩.
\end{hint}

\begin{example} % 启东中学奥赛教程 p12
	设$S \subsetneqq \mathbb{R}$是一个非空的有限实数集,定义$|S|$为$S$中的元素个数,$$m(S) = \frac{\sum_{x \in S} x}{|S|}$$
	已知$S$的任意两个非空子集的元素的算术平均值都不相同.定义$$\dot{S} = \{ m(A) | A \subseteq S, ~A \neq \varnothing \}$$
	证明:$m(\dot{S}) = m(S)$.
\end{example}
\begin{hint}
	贡献法.
\end{hint}

\newpage
\section{子集的性质}



\chapter{函数}

\chapter{三角函数}

\chapter{平面向量}

\chapter{复数}

\chapter{数列}

\chapter{极限与导数}

\chapter{不等式}

\chapter{概率统计与计数}

\chapter{解析几何}

\chapter{立体几何}








\setcounter{chapter}{0}
\part{二试与冬令营部分}

\chapter{代数部分}

\section{不等式问题}

\subsection{切线放缩}



\chapter{几何部分}

\chapter{组合部分}

\chapter{数论部分}


\end{document}





















