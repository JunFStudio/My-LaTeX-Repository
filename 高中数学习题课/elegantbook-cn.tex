\documentclass[lang=cn, zihao=4.5]{elegantbook}
\usepackage{hyperref}

% font settings


% watermark settings
%\usepackage{ctex, draftwatermark, everypage}
%	\SetWatermarkText{DEEP Team 讲义模版}
%	\SetWatermarkLightness{0.95}
%	\SetWatermarkScale{0.3}

% customised commands
\usepackage{ulem}
	\newcommand{\tk}{\uline{\hspace{4em}}}
	
\DeclareSymbolFont{yh}{OMX}{yhex}{m}{n}
\DeclareMathAccent{\hu}{\mathord}{yh}{"F3}

\newcommand{\xl}[1]{\overrightarrow{#1}}
\newcommand{\nd}[1]{〔#1〕}
\newcommand{\ssb}[1]{\left( #1 \right)}
\newcommand{\sw}[1]{\boxed{\text{解法 #1}} \ }
\newcommand{\buzhou}[1]{$#1^{\circ} \ $}

% cover settings

\title{高中数学习题册}
\subtitle

\author{Johnny Tang}
\institute{DEEP Team}
\date{January 21, 2022}

\extrainfo{请:相信时间的力量,敬畏概率的准则}


\cover{cover.png}

% 本文档命令


% 修改标题页的橙色带
% \definecolor{customcolor}{RGB}{32,178,170}
% \colorlet{coverlinecolor}{customcolor}


\begin{document}

\maketitle

\frontmatter

\mainmatter

\tableofcontents

\newpage

\part{一试与强基部分}

\chapter{集合}

\section{集合及其运算}

\begin{example} % 启东中学奥赛教程 p4
	(2015高联)设$a_1,a_2,a_3,a_4$是$4$个有理数,使得$\{ a_ia_j | 1 \leq i < j \leq 4 \} = \{ -24,-2,-\dfrac{3}{2},-\dfrac{1}{8},1,3 \}$.求$a_1+a_2+a_3+a_4$的值.
\end{example}

\section{集合元素的数目}

\chapter{函数}

\chapter{三角函数}

\chapter{平面向量}

\chapter{复数}

\chapter{数列}

\chapter{极限与导数}

\chapter{不等式}

\chapter{概率统计与计数}

\chapter{解析几何}

\chapter{立体几何}








\part{二试与冬令营部分}

\chapter{代数部分}

\chapter{几何部分}

\chapter{组合部分}

\chapter{数论部分}


\end{document}





















