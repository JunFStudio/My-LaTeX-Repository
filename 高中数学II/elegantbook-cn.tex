\documentclass[lang=cn, zihao=4.5]{elegantbook}
\usepackage{hyperref}

% font settings
\definecolor{mgreen}{RGB}{0,166,82}
\definecolor{guess}{RGB}{37,55,105}

% watermark settings
%\usepackage{ctex, draftwatermark, everypage}
%	\SetWatermarkText{DEEP Team 讲义模版}
%	\SetWatermarkLightness{0.95}
%	\SetWatermarkScale{0.3}

% customised commands
\usepackage{ulem}
	\newcommand{\tk}{\uline{\hspace{4em}}}
	
\DeclareSymbolFont{yh}{OMX}{yhex}{m}{n}
\DeclareMathAccent{\hu}{\mathord}{yh}{"F3}

\newcommand{\xl}[1]{\overrightarrow{#1}}
\newcommand{\nd}[1]{〔#1〕}
\newcommand{\ssb}[1]{\left( #1 \right)}
\newcommand{\sw}[1]{\boxed{\text{解法 #1}} \ }
\newcommand{\buzhou}[1]{$#1^{\circ} \ $}
\newcommand{\R}{\mathbb{R}}

\DeclareMathOperator{\card}{card}

\newcommand{\examplefont}[1]{\color{mgreen} \textbf{#1}}

% cover settings

\title{高中数学·二阶}
\subtitle{适用于联赛二试与冬令营}

\author{Johnny Tang}
\institute{DEEP Team}
\date{January 21, 2022}

\extrainfo{请:相信时间的力量,敬畏概率的准则}


\cover{cover.png}

% 本文档命令


% 修改标题页的橙色带
% \definecolor{customcolor}{RGB}{32,178,170}
% \colorlet{coverlinecolor}{customcolor}


\begin{document}

\maketitle

\frontmatter

\mainmatter

\tableofcontents

\newpage

\part{代数部分}

\part{几何部分}

\part{组合部分}

\chapter{常见结论}

\section{抽屉原理}

\begin{theorem}{抽屉原理}
	有$m$个小球,$n$个抽屉,那么存在一个抽屉放了至少$\left[ \dfrac{m-1}{n} \right]+1$个、至多$\left[ \dfrac{m}{n} \right]$个小球.
\end{theorem}
\begin{proof}
	(1)假设所有抽屉最多有$\left[ \dfrac{m-1}{n} \right]$个小球,则总小球数目至多为$$\left[ \frac{m-1}{n} \right] \times n \leq \frac{m-1}{n} \times n = m-1 < m$$
	这与条件矛盾. \\
	(2)假设所有抽屉至少有$\left[ \dfrac{m}{n} \right] + 1$个小球,则总小球数目至少为$$\ssb{\left[ \dfrac{m}{n} \right] + 1} \times n > \frac{m}{n} \times n = m$$
	这与条件矛盾.
\end{proof}

\begin{corollary}{平均值原理}
	对于给定的实数$a_1, \cdots ,a_n$,存在$a_i,a_j$使得$$a_i \geq \dfrac{1}{n}(a_1+ \cdots +a_n), \quad a_j \leq \dfrac{1}{n}(a_1+ \cdots +a_n)$$
\end{corollary}

\begin{example} % 兴趣二阶组合1.1
	\nd{1}证明: \\
	(1)从前$100$个正整数中任意取出$51$个数,都可以找到两个数,使得它们中的一个是另一个的整数倍. \\
	(2)从前$91$个正整数中任意取出$10$个数,则一定有两个数,使得这两个数中较大数不超过较小数的$1.5$倍. \\
	(3)若$a_1,a_2,\cdots ,a_{100}$都是实数,且在集合$\{ a_1, \dfrac{a_1+a_2}{2}, \cdots ,\dfrac{a_1+a_2+\cdots +a_{100}}{100} \}$中至少有$51$个元素的值相等,则$a_1,a_2, \cdots ,a_{100}$中有两个数相等.
\end{example}
\begin{proof}
	(1)构造:$$\{ 1 \times 2^0, \cdots , 1 \times 2^6 \},\{ 3 \times 2^0, \cdots , 3 \times 2^5 \}, \cdots ,\{99 \times 2^0 \}$$共$50$个抽屉.由抽屉原理,在前$100$个正整数中必有两个数在同一个抽屉中,即它们有倍数关系. \\
	(2)构造:$$\{ 1 \},\{ 2,3 \},\{ 4,5,6 \},\{ 7,8,9,10 \},\{ 11,12,13,14,15,16 \},\{ 17,\cdots 25 \},\{ 26,\cdots ,39 \},\{ 40, \cdots ,60 \},\{ 61, \cdots ,91 \}$$共$9$个抽屉.由抽屉原理,前$91$个正整数中必有两个在同一抽屉中,即满足较大数不超过较小数的$1.5$倍. \\
	(3)记$b_i=\dfrac{a_1+ \cdots + a_i}{i}$,注意到,若$b_i=b_{i+1}=p$,则$$\frac{a_1+ \cdots + a_i}{i} = \frac{a_1+ \cdots + a_i + a_{i+1}}{i+1}$$可得$a_{i+1} = \dfrac{a_1 + \cdots + a_i}{i} = b_i = p$.设$b_1, \cdots ,b_{100}$中相等的$51$个数均等于$p$. \\
	\buzhou{1} 当$a_1 \neq p$时:构造$$\{ b_1,b_2 \}, \cdots ,\{ b_{99},b_{100} \}$$共$50$个抽屉,同理存在$a_{2k+1}=p$;构造$$\{ b_2,b_3 \}, \cdots ,\{ b_{98},b_{99} \},\{ b_{100} \}$$共$50$个抽屉,同理存在$a_{2l}=p$.故$a_{2k+1}=a_{2l}$. \\
	\buzhou{2} 当$a_1=p$时:构造$$\{ b_2,b_3 \}, \cdots ,\{ b_{98},b_{99} \},\{ b_{100} \}$$共$50$个抽屉.由抽屉原理,必存在$b_{2k}=b_{2k+1}=p$,因而$a_{2k}=p=a_1$.
\end{proof}

\begin{example} % 兴趣二阶组合1.2
	\nd{1}证明: \\
	(1)平面上任作$8$条互不平行的直线,其中必有两条直线的夹角小于$23$度. \\
	(2)给定一个由$10$个互不相等的两位十进制正整数组成的集合,则这个集合必有两个无公共元素的非空子集合,它们的元素和相等. \\
	(3)$100$个孩子围成一圈,其中$41$个男孩,$59$个女孩.则一定有$2$个男孩,他们中间的孩子个数恰为$19$的整数倍.
\end{example}
\begin{proof}
	(1)由于平面上两直线的夹角不会随平移而改变,不妨平移这$8$条线使得它们交于同一点.由平均值原理,必有两条直线的夹角小于等于$22.5$度,即小于$23$度. \\
	(2)由于所有可能的非空子集个数为$2^{10}-1$,而子集和的所有可能情况只有在$[10,945]$中(共$936$种),由抽屉原理,必有两个子集的和相同. \\
	若这两个子集交集为空,则符合题意;若交集不空,则分别去掉交集中的元素,构成两个新的元素和相等且交集为空的集合. \\
	(3)假设这$100$个孩子的编号分别为$1, \cdots ,100$,则构造$$\{ 1,21,41,61,81 \}, \{ 2,22,42,62,82 \}, \cdots ,\{ 20,40,60,80,100 \}$$
	共$20$个集合.由抽屉原理,至少有一个集合中同时有三个男孩,即满足题意.
\end{proof}

\begin{example} % 兴趣二阶组合1.3
	\nd{1}证明: \\
	(1)已知$a_1, \cdots a_{21}$是区间$(0,400)$内的$21$个实数,总可以找到两个数$a_i,a_j(1 \leq i < j \leq 21)$,满足$a_i+a_j < 1+2\sqrt{a_ia_j}$. \\
	(2)已知实数$0<a_1 < \cdots < a_{2011}$,则存在两个数$a_i,a_j(1 \leq i < j \leq 2011)$,满足$a_j-a_i < \dfrac{(1+a_i)(1+a_j)}{2010}$.
\end{example}
\begin{proof}
	(1)只需证明$\sqrt{a_i}-\sqrt{a_j}<\sqrt{2}$即可.实际上,不妨设$a_1<\cdots <a_{21}$,那么在$\sqrt{a_{21}}-\sqrt{a_{20}}, \cdots ,\sqrt{a_2}-\sqrt{a_1}$中,由于它们的和为$\sqrt{a_{21}}-\sqrt{a_1} < 20$,由平均值原理可知其中必有一个$<1<\sqrt{2}$. \\
	(2)只需证明$\dfrac{1}{1+a_i}-\dfrac{1}{1+a_j} < \dfrac{1}{2010}$.与(1)同理可知.
\end{proof}

\begin{example}
	从$4$个同心圆的圆心出发的$100$条射线等分各圆周,分别与$4$个圆各有$100$个交点.任意给每个圆上的点染上黑、白两色之一,使每个圆上都恰有$50$个黑点和$50$个白点.证明:可将此$4$个圆适当旋转,使这$100$条射线中至少存在$13$条射线,它们中的每一条穿过的$4$个点颜色都相同.
\end{example}

\part{数论部分}


\end{document}





















