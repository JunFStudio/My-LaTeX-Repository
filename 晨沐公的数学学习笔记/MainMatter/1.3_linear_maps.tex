\chapter{线性映射与矩阵理论}


\section{基本概念}

\subsection{向量空间的线性映射}

在描述一种颜色时, 我们会给出几个不同方面的数据进行叠加. 以RGB颜色为例, 其通过对红(R), 绿(G), 蓝(B)三个通道分别赋予$0\sim 255$的值再进行组合而产生出$256^3$种颜色. 设颜色向量$c_1,c_2,c_3$. 如果想将$c_1,c_2$叠加, 我们会考虑一种加权平均, 也即新的颜色$c_4=tc_1+(1-t)c_2, 0\leq t \leq 1$. 如果考虑$c_5$为另一种方式的叠加, 即$c_5=sc_1+(1-s)c_2, 0 \leq s \leq 1$, 再定义两种颜色的直接和$\tilde{+}$为其算术平均, 那么会有$c_4 \tilde{+} c_5= (t \tilde{+} s) c_1 + ((1-t) \tilde{+} (1-s)) c_2$. 将$t$与$s$视作映射$C$的自变量, 即可得到$C(t \tilde{+} s) = C(t) \tilde{+} C(s)$. 同样地, 如果定义标量乘法$\tilde{\times}$表示将白色$(0,0,0)$与该颜色按量$\lambda$混合, 可以得到$C(\lambda \tilde{\times} t) = \lambda \tilde{\times} C(t)$. 

上方的例子解释了所谓\textit{叠加原理}: 即和的输入所得输出等于输入所得输出的和, 标量倍输入所得输出等于输入所得输出的标量倍. 由此引入线性映射的概念. 

\begin{definition}{线性映射}
	设函数$T:V \to W$,若该函数满足下列性质:
	\begin{enumerate}
		\item \textit{加性}:$$\forall u,v \in V,~T(u+v)=Tu+Tv.$$
		\item \textit{齐性}:$$\forall \lambda \in \F,~v \in V,~T(\lambda v)=\lambda Tv.$$
	\end{enumerate}
	则称$T$是\textit{线性映射}.
\end{definition}
\begin{remark}
	为了让你想起线性映射可以直接写成矩阵乘法(见3.1.3节), 通常用$Tv$代替$T(v)$. 
\end{remark}

规定记号$\lmap (V,W)$,表示所有从$V$到$W$的线性映射构成的集合.

另一种理解线性映射的方式: 记$T$的图为$\{ (v,Tv)\in V \times W : v \in V \}$, 则$T$是线性映射当且仅当$T$的图是$V \times W$的子空间. (这里, $\times$表示笛卡尔积)

\begin{example}{\examplefont{线性映射的例子}}
	(1)零函数:定义$0 \in \lmap (V,W)$如下$$\forall v \in V, 0v=0.$$
	(2)恒等映射:定义$I \in \lmap (V,V)$如下$$\forall v \in V, Iv=v.$$
	(3)微分:定义$D \in \lmap (\mathcal{P}(\R),\mathcal{P}(\R))$如下$$Dp=p'.$$
	(4)乘以$x^2$:定义$T \in \lmap (\mathcal{P}(\R),\mathcal{P}(\R))$满足$$\forall x \in \R ,~(Tp)(x)=x^2p(x).$$
	(5)向后移位:定义$T \in \lmap (\F ^{\infty},\F ^{\infty})$如下$$T(x_1,x_2,x_3, \cdots ) = (x_2,x_3, \cdots).$$
	(6)从$\R ^3$到$\R ^2$:定义$T \in \lmap (\R ^{3},\R ^{2})$如下$$T(x,y,z)=(2x-y+3z,7x+5y-6z).$$
	(7)从$\F ^n$到$\F ^m$:定义$T \in \lmap (\F ^{n},\F ^{m})$如下$$T(x_1,\cdots ,x_n) = (A_{1,1}x_1 + \cdots + A_{1,n}x_n, \cdots , A_{m,1}x_1+\cdots +A_{m,n}x_n).$$
	其中,$A_{j,k} \in \F ,~j=1, \cdots ,m,~ k = 1, \cdots ,n$.事实上,从$\F ^n$到$\F ^m$的每个线性映射都是这种形式的.我们稍后会进行证明.
\end{example}

观察线性映射所具有的加性与齐性,似乎可以将其与线性组合联系起来.例如,对于$T \in \lmap (V,W)$,若给定$v_1, \cdots ,v_m$是$V$的基,设$c_1, \cdots ,c_m$是$\F$中的任意元素,则$$T(c_1v_1 + \cdots + c_mv_m) = c_1Tv_1 + \cdots + c_mTv_m.$$
由这种想法,以下的命题不难证明:

\begin{proposition}{线性映射与定义域的基} \label{pro:xmxkykuedkyiyu}
	对于$T \in \lmap (V,W)$,设$v_1, \cdots ,v_m$是$V$的基,$w_1, \cdots ,w_m \in W$,则存在唯一一个线性映射$T \in \lmap (V,W)$使得对任意$j=1, \cdots ,m$都有$$Tv_j = w_j.$$
\end{proposition}
\begin{proof}
	\buzhou{1}存在性:利用先前得到的想法,构造$T:V \to W$如下:$$\forall c_j \in \F~(j=1, \cdots,m),~T(c_1v_1 + \cdots + c_mv_m) = c_1w_1 + \cdots + c_mw_m.$$
	由于$v_1, \cdots ,v_m$是$V$的基,$T$的定义域是$V$.下证$T \in \lmap (V,W)$. \\
	取$u=a_1v_1 + \cdots + a_mv_m,~v=c_1v_1 + \cdots + a_mv_m,~ \lambda \in \F$,因为
	\begin{align*}
		T(u+v) &= T((a_1+c_1)v_1 + \cdots + (a_m+c_m)v_m) \\
		&= (a_1+c_1)w_1 + \cdots + (a_m+c_m)w_m, \\
		Tu+Tv &= a_1w_1 + \cdots + a_mv_m + c_1v_1 + \cdots + c_mv_m.
	\end{align*}
	所以$T(u+v)=Tu+Tv$,即$T$满足加性.因为
	\begin{align*}
		T(\lambda u) &= T((\lambda a_1)v_1 + \cdots + (\lambda a_m)v_m) \\
		&= \lambda a_1 w_1 + \cdots + \lambda a_m w_m, \\
		\lambda T(u) &= \lambda (a_1w_1 + \cdots a_mw_m).
	\end{align*}
	所以$T(\lambda u)=\lambda T(u)$,即$T$满足齐性.综上,这样的$T \in \lmap (V,W)$. \\
	\buzhou{2}唯一性:若$T \in \lmap (V,W)$,记$T v_j = w_j~(j=1,\cdots ,m)$,设$c_1, \cdots ,c_m$是$\F$中的任意元素,有$$T(c_1v_1 + \cdots + c_mv_m) = c_1w_1 + \cdots + c_mw_m.$$
	这表明任何一个满足题目要求的映射$T$都满足上述形式要求.结合\buzhou{1},可知映射$T$的唯一形式即为该形式.
\end{proof}

对上方的命题稍加延伸, 不难证明: 

\begin{proposition}{}
	设$v_1,\cdots ,v_n \in V, T \in \lmap (V,W)$. \\
	(1) $v_1,\cdots ,v_n$线性相关, 则$Tv_1,\cdots ,Tv_n$线性相关; \\
	(2) $v_1,\cdots ,v_n$线性无关且满足$Tv=0$的$v$只有$0$(即$T$的零空间为$\{ 0 \}$), 则$Tv_1,\cdots ,Tv_n$线性无关. 
\end{proposition}

\begin{example}
	设$T \in \lmap (\F ^n,\F ^m)$.证明存在标量$A_{j,k} \in \F$~(其中$j=1, \cdots ,m,~k=1,\cdots ,n$)使得对任意$(x_1, \cdots ,x_n) \in \F ^n$都有$$T(x_1, \cdots ,x_n) = (A_{1,1}x_1+ \cdots +A_{1,n}x_n, \cdots ,A_{m,1}x_1+\cdots +A_{m,n}x_n).$$
\end{example}
\begin{proof}
	记$T(e_i)=(A_{1i},\cdots A_{mi})$, 则$T(x_1,\cdots ,x_n)=x_1Te_1+\cdots + x_nTe_n = (A_{1,1}x_1+ \cdots +A_{1,n}x_n, \cdots ,A_{m,1}x_1+\cdots +A_{m,n}x_n)$.
\end{proof}

\begin{example}
	将二维平面$\R ^2$上的绕原点逆时针旋转$\theta$看做映射$\varphi _{\theta} : (x,y) \mapsto (x_1,y_1)$. 容易验证$\varphi _{\theta}$是一个线性映射. 由于$$\varphi _{\theta} (1,0) = (\cos \theta , \sin \theta) , \varphi _{\theta} (1,0) = (\cos \left( \theta + \frac{\pi}{2} \right) , \sin \left( \theta + \frac{\pi}{2} \right)),$$
	我们可以得到$\varphi _{\theta} (x,y) = (x\cos \theta - y\sin \theta ,x\sin \theta + y\cos \theta)$.
\end{example}

我们接着完善线性映射的定义.先在$\lmap (V,W)$上定义加法和标量乘法:

\begin{definition}{$\lmap (V,W)$上的加法和标量乘法}
	设$S,T \in \lmap (V,W),~\lambda \in \F$. \\
	(1)定义\textit{和}$S+T$是$V$到$W$的映射,满足$$\forall u \in V,~(S+T)(u)=Su+Tu.$$
	(2)定义\textit{积}$\lambda T$是$V$到$W$的映射,满足$$\forall u \in V,~(\lambda T)(u) = \lambda (Tu).$$
\end{definition}
\begin{remark}
	实际上,定义中的$S+T,\lambda T$均为从$V$到$W$的线性映射,也即上述定义的加法、标量乘法是封闭的.更进一步,$\lmap (V,W)$就是一个向量空间.这个命题易于证明.
\end{remark}

一般来说,向量空间中元素之间的乘法是没有意义的.然而对于线性映射,我们倾向于将一种特殊的运算视作乘积:

\begin{definition}{线性映射的乘积}
	设$T \in \lmap (U,V),~S \in \lmap (V,W)$,定义\textit{乘积}$ST$是$U$到$W$的映射,满足$$\forall u \in U,~(ST)(u)=S(Tu).$$
\end{definition}
\begin{remark}
	同样的,这里的$ST$是从$U$到$W$的线性映射.
\end{remark}
\begin{remark}
	此处线性映射的乘积就是所谓“函数复合”$S \circ T$.
\end{remark}

\begin{proposition}{线性映射乘积的代数性质} \label{pro:xmxkykueigji}
	(1)结合性:设线性映射$T_1,T_2,T_3$,在运算有意义的情况下,有$$(T_1T_2)T_3=T_1(T_2T_3).$$
	(2)单位元:设$T \in \lmap (V,W)$,$I$是$W$上的恒等映射,则$$TI=IT=T.$$
	(3)分配性质:设$T,T_1,T_2 \in \lmap (U,V),~S,S_1,S_2 \in \lmap (V,W)$,则$$(S_1+S_2)T=S_1T+S_2T,\quad S(T_1+T_2)=ST_1+ST_2.$$
\end{proposition}

请注意,线性映射的乘法不满足交换性,即$ST=TS$不一定成立.

\subsection*{习题 ~~\small 对应原书3.A习题}

\begin{exercise}
	给出一个函数$\varphi :\R ^2 \to \R$,使得对所有$a \in \R$和所有$v \in \R ^2$有$\varphi (av)=a\varphi (v)$
	但$\varphi$不是线性的.
\end{exercise}
\begin{solution}
	$\varphi (x,y)= \sqrt[3]{x^3+y^3}$.
\end{solution}

\begin{exercise}
	给出一个函数$\varphi :\C \to \C$,使得对所有$w,z \in \C$都有$\varphi (w+z) = \varphi (w) + \varphi (z)$
	但$\varphi$不是线性的.这里$\C$视为一个复向量空间.
\end{exercise}
\begin{solution}
	$\varphi (a+bi)=a$. 注: $\varphi :\R \to \R$的情况即为Cauchy方程问题. 
\end{solution}

\begin{exercise}
	设$V$是有限维的.证明$V$的子空间上的线性映射可以扩张成$V$上的线性映射.也就是说,证明:如果$U$是$V$的子空间,$S \in \lmap (U,W)$,那么存在$T \in \lmap (V,W)$使得对所有$u \in U$都有$Tu=Su$.
\end{exercise}
\begin{proof}
	取$U$的基$u_1,\cdots ,u_m$并将其扩张为$V$的基$u_1,\cdots ,u_m,v_1,\cdots ,v_n$, 定义$Tu_i=Su_i$和$Tv_j=0$, 容易验证对于$u \in U$有$Tu=Su$. 
\end{proof}

\begin{exercise}
	设$V$是有限维的且$\dim V>0$,再设$W$是无限维的.证明$\lmap (V,W)$是无限维的.
\end{exercise}
\begin{proof}
	由$W$是无限维的, 其中一定存在某个无穷向量序列$w_1,\cdots ,w_m,\cdots$满足任意前$m$位都线性无关. 取定$v \in V$, 构造$T_iv=w_i$, 那么$$a_1T_1 + \cdots + a_mT_m = 0 \Rightarrow (a_1T_1 + \cdots + a_mT_m)(v)=0 \Leftrightarrow a_1w_1 + \cdots + a_mw_m = 0.$$
	从而$T_1,\cdots ,T_m$线性无关. 这就证明了$\lmap (V,W)$是无限维的. 
\end{proof}

\begin{exercise}
	设$v_1, \cdots ,v_m$是$V$中的一个线性相关的向量组,并设$W \neq \{ 0 \}$.证明存在$w_1, \cdots ,w_m \in W$使得没有$T \in \lmap (V,W)$能满足$Tv_k=w_k,~k=1,\cdots ,m$.
\end{exercise}
\begin{proof}
	考虑$a_1v_1+\cdots +a_mv_m=0$, 其中不妨设$a_1 \neq 0$. 先待定$w_k$的取法, 假设满足$Tv_k=w_k$, 则有$0=T(a_1v_1+\cdots + a_mv_m)=a_1w_1+\cdots + a_mv_m$. 只需要取$w_1 \neq 0$且其余$w_k$均为$0$即可. 
\end{proof}

\begin{exercise}
	设$V$是有限维的且$\dim V \geq 2$.证明存在$S,T \in \lmap (V,V)$使得$ST \neq TS$.
\end{exercise}
\begin{proof}
	设$V$的一组基为$v_1,\cdots ,v_n$. 取$S$满足$Sv_1=v_2,Sv_2=v_1,Sv_i=v_i(i\geq 3)$, $T$满足$Tv_1=a_1v_1+b_1v_2,Tv_2=a_2v_1+b_2v_2,Tv_i=v_i(i\geq 3)$, 其中$a,b$待定. 如下构造即可$$STv_1=a_1v_2+b_1v_1 \neq a_2v_1+b_2v_2 = TSv_1. $$
\end{proof}

\subsection{线性映射的表示矩阵}

我们知道,对于线性映射$T:V \to W$,通过$V$的基的象$Tv_1, \cdots ,Tv_n$可以确定任意$V$中元素的象. 稍后我们会利用$W$的基在矩阵上记录这些$Tv_j$的值.

\begin{definition}{矩阵}
	设正整数$m,n$,$m \times n$\textit{矩阵}$A$是由$\F$的元素构成的$m$行$n$列的矩形阵列:$$A = 
	\begin{pmatrix}
		A_{1,1} & \cdots & A_{1,n} \\
		\vdots &  & \vdots \\
		A_{m,1} & \cdots & A_{m,n}
	\end{pmatrix}.$$
	其中,记号$A_{j,k}$表示$A$的第$j$行第$k$列的元素.
\end{definition}

\begin{definition}{线性映射的矩阵}
	设$T \in \lmap (V,W)$,并设$v_1, \cdots ,v_n$是$V$的基,$w_1, \cdots ,w_n$是$W$的基.规定$T$\textit{关于这些基的矩阵}为$m \times n$矩阵$\mathcal{M}(T)$,其中$A_{j,k}$满足$$Tv_k = A_{1,k}w_1 + \cdots + A_{m,k}w_m.$$
	如果这些基不是上下文自明的,则采用记号$\mathcal{M}(T,(v_1, \cdots ,v_n),(w_1, \cdots ,w_m))$.
\end{definition}

构造$\mathcal{M}(T)$的方法如下图所示:把$Tv_k$写成$w_1, \cdots ,w_m$的线性组合形式$A_{1,k} w_1 + \cdots + A_{m,k} w_m$,那么所有系数自上而下构成的矩阵的第$k$列.
	$$\mathcal{M}(T) = \begin{matrix}
  	& Tv_1~~ \cdots ~~Tv_k~~ \cdots ~~Tv_n\\
	\begin{matrix} w_1 \\ \vdots \\ w_m \end{matrix}  
	&\begin{pmatrix} ~~~~~ & ~~~~~ & A_{1,k} & ~~~~~ & ~~~~~\\  &  & \vdots &  & \\  &  & A_{m,k} &  & \end{pmatrix}
	\end{matrix}.$$

例如,对于线性映射$T:\F ^2 \to \F ^3$定义为$T(x,y)=(x+3y,2x+5y,7x+9y)$,则$T$关于$\F ^2$与$\F ^3$的标准基的矩阵为$$\mathcal{M}(T)= \begin{pmatrix}
	1 & 3 \\ 2 & 5 \\ 7 & 9
\end{pmatrix}.$$
这是因为$T(1,0)=(1,2,7)=1(1,0,0)+2(0,1,0)+7(0,0,1),~T(0,1)=(3,5,9)=3(1,0,0)+5(0,1,0)+9(0,0,1)$.

对于微分映射$D:\mathcal{P}_3(\R) \to \mathcal{P}_2(\R)$满足$Dp=p'$,它关于$\mathcal{P}_3(\R)$和$\mathcal{P}_2(\R)$的标准基的矩阵为$$\begin{pmatrix}
	0 & 1 & 0 & 0 \\ 0 & 0 & 2 & 0 \\ 0 & 0 & 0 & 3
\end{pmatrix}.$$

如果将向量表示为如下形式: 

\begin{definition}{向量的矩阵}
	设$v \in V$,并设$v_1, \cdots ,v_n$是$V$的基.若$v=c_1v_1 + \cdots + c_nv_n$,规定$v$关于这个基的矩阵是一个$n \times 1$矩阵$$\mmatrix(v) = \begin{pmatrix}
		c_1 \\ \vdots \\ c_n
	\end{pmatrix}.$$
\end{definition}

那么对于具有表示矩阵$A$的线性映射$T \in \lmap (V,W)$, $Tv$可以表示为: 
$$\begin{pmatrix}
		A_{11}c_1+\cdots +A_{1n}c_n \\ \vdots \\ A_{m1}c_1+\cdots +A_{mn}c_n
	\end{pmatrix} := \begin{pmatrix}
		A_{11} & \cdots & A_{1n} \\
		\vdots &  & \vdots \\
		A_{m1} & \cdots & A_{mn}
	\end{pmatrix} \begin{pmatrix}
		c_1 \\ \vdots \\ c_n
	\end{pmatrix}.$$
	
我们将其定义为矩阵与向量的乘积. 如此, $Tv$与$Av$的结果就相等了, 因而常将线性映射的表示矩阵写作其本身的符号. 当然, 这样的运算满足线性分配律$A(c_1v_1 + \cdots + c_nv_n) = c_1(Av_1) + \cdots c_n(Av_n)$. 

前文所提到的旋转变换就可以这样写: $$\begin{pmatrix}
		x_1 \\ y_1
	\end{pmatrix} = \begin{pmatrix}
		\cos \theta & -\sin \theta \\ \sin \theta & \cos \theta
	\end{pmatrix} \cdot \begin{pmatrix}
		x \\ y
	\end{pmatrix}.$$



\subsection{矩阵的运算}

矩阵的加法与标量乘法定义很符合直觉:

\begin{definition}{矩阵的加法与标量乘法}
	(1) 规定两个同样大小的矩阵的\textit{和}是将对应元素相加得到的矩阵:$$\begin{pmatrix}
		A_{1,1} & \cdots & A_{1,n} \\
		\vdots &  & \vdots \\
		A_{m,1} & \cdots & A_{m,n}
	\end{pmatrix} + \begin{pmatrix}
		C_{1,1} & \cdots & C_{1,n} \\
		\vdots &  & \vdots \\
		C_{m,1} & \cdots & C_{m,n}
	\end{pmatrix} = \begin{pmatrix}
		A_{1,1}+C_{1,1} & \cdots & A_{1,n}+C_{1,n} \\
		\vdots &  & \vdots \\
		A_{m,1}+C_{m,1} & \cdots & A_{m,n}+C_{m,n}
	\end{pmatrix}.$$
	(2) 规定标量与矩阵的\textit{乘积}是将标量乘以每个元素得到的矩阵:$$\lambda \begin{pmatrix}
		A_{1,1} & \cdots & A_{1,n} \\
		\vdots &  & \vdots \\
		A_{m,1} & \cdots & A_{m,n}
	\end{pmatrix} = \begin{pmatrix}
		\lambda A_{1,1} & \cdots & \lambda A_{1,n} \\
		\vdots &  & \vdots \\
		\lambda A_{m,1} & \cdots & \lambda A_{m,n}
	\end{pmatrix}.$$
\end{definition}

因而,我们有

\begin{proposition}{线性映射与矩阵运算}\label{pro:xmxkykueyysrjuvf}
	(1) 设$S,T \in \lmap (V,W)$,则$\mmatrix (S+T)=\mmatrix (S) + \mmatrix (T)$; \\
	(2) 设$\lambda \in \F ,~T \in \lmap (V,W)$,则$\mmatrix (\lambda T) = \lambda \mmatrix (T)$. 
\end{proposition}

\begin{proposition}{$\F ^{m,n}$是向量空间}
	对于正整数$m,n$,元素取自$\F$的所有$m \times n$矩阵的集合记为$\F ^{m,n}$.按照矩阵运算的定义,$\F ^{m,n}$是$mn$维向量空间.
\end{proposition}

上述命题的证明是显然的.

我们注意到,线性映射不止有加法和标量乘法,还有元素之间的乘法. 猜测是否会有$\mmatrix (ST)= \mmatrix (S) \mmatrix (T)$?为了得到这个结果,尝试倒推矩阵乘法的定义:

考虑$T \in \lmap (U,V),~S \in \lmap (V,W)$,并设$u_1, \cdots ,u_p$是$U$的基,$v_1, \cdots ,v_n$是$V$的基,$w_1, \cdots ,w_m$是$W$的基.记$\mmatrix (S)=A,~\mmatrix (T)=C$.那么对于任意的$1 \leq k \leq p$,有$$(ST)u_k = S\ssb{\sum_{r=1}^{n} C_{r,k}v_r} = \sum_{r=1}^{n} C_{r,k}Sv_r = \sum_{r=1}^{n} C_{r,k} \sum_{j=1}^{m} A_{j,r}w_j = \sum_{j=1}^{m} \ssb{\sum_{r=1}^{n} A_{j,r} C_{r,k}} w_j.$$
因此$\mmatrix (ST)$是$m \times p$矩阵,且满足$$\mmatrix (ST)_{j,k} = \sum_{r=1}^{n} A_{j,r} C_{r,k}.$$
于是可以定义:

\begin{definition}{矩阵乘法}
	设$A$是$m \times n$矩阵,$C$是$n \times p$矩阵.$AC$定义为$m \times p$矩阵,满足$$(AC)_{j,k} = \sum_{r=1}^{n} A_{j,r} C_{r,k}.$$
	也即,将$A$的第$j$行与$C$的第$k$列元素对应相乘再相加.
\end{definition}

这样的矩阵乘法脱胎于线性映射的乘法,因此其代数性质也类似线性映射乘法的结合性、单位元、分配性质,且不满足交换性.

例如,将一个$3 \times 2$矩阵与一个$2 \times 4$矩阵相乘,得到一个$3 \times 4$矩阵:$$\begin{pmatrix}
	1 & 2 \\ 3 & 4 \\ 5 & 6
\end{pmatrix} \begin{pmatrix}
	6 & 5 & 4 & 3 \\ 2 & 1 & 0 & -1
\end{pmatrix} = \begin{pmatrix}
	10 & 7 & 4 & 1 \\ 26 & 19 & 12 & 5 \\ 42 & 31 & 20 & 9
\end{pmatrix}.$$

若将第二个矩阵写作行向量的列向量形式, 可以视作矩阵对向量的乘法: $$\begin{pmatrix}
	1 & 2 \\ 3 & 4 \\ 5 & 6
\end{pmatrix} \begin{pmatrix}
	A_1^{\T} \\ A_2^{\T}
\end{pmatrix} = \begin{pmatrix}
	A_1^{\T}+2A_2^{\T} \\ 3A_1^{\T}+4A_2^{\T} \\ 5A_1^{\T}+6A_2^{\T}
\end{pmatrix}.$$

实际上是将每一列分开乘再进行特殊的相加(这里记作$\tilde{+}$): $$\begin{pmatrix}
	1 & 2 \\ 3 & 4 \\ 5 & 6
\end{pmatrix} \ssb{\begin{pmatrix} 6 \\ 2 \end{pmatrix} \tilde{+} \begin{pmatrix} 5 \\ 1 \end{pmatrix} \tilde{+} \begin{pmatrix} 4 \\ 0 \end{pmatrix} \tilde{+} \begin{pmatrix} 3 \\ -1 \end{pmatrix}}
 = \begin{pmatrix} 10 \\ 26 \\ 42 \end{pmatrix} \tilde{+} \begin{pmatrix} 7 \\ 19 \\ 31 \end{pmatrix} \tilde{+} \begin{pmatrix} 4 \\ 12 \\ 20 \end{pmatrix} \tilde{+} \begin{pmatrix} 1 \\ 5 \\ 9 \end{pmatrix}.$$

例如将列向量转为行向量, 有时需要考虑矩阵的转置: 

\begin{definition}{矩阵的转置}
	$m\times n$矩阵$A$的\textit{转置}定义为将其行列交换得到的$n\times m$矩阵, 记作$A^{\T}$. 
\end{definition}

容易验证如下命题: 

\begin{proposition}{转置的运算}
	设$m\times n$矩阵$A$, $n\times p$矩阵$B$, $\lambda ,\mu \in \mathbb{F}$. 我们有$$1.~~(\lambda A + \mu B)^{\T} = \lambda A^{\T} + \mu B^{\T} ,\qquad 2.~~(AB)^{\T}=B^{\T}A^{\T}.$$
\end{proposition}

\newpage
\section{零空间与值域}

\subsection{解线性方程组问题}


考虑解一个非齐次线性方程组(即右侧常数不全为$0$): $$\begin{cases}
	4x_1+3x_2-6x_3=-17 \\ x_1+2x_2+3x_3=22 \\ 2x_1+3x_2 = 11
\end{cases}.$$

我们当然可以使用一些奇技淫巧, 例如通过后两个方程直接得到$x_1,x_2,x_3$的倍数关系, 但这毕竟不是通用方法. 最经典的方法应该是消元: 不断地利用倍乘变换(将某个方程左右同乘常数$k$)、倍加变换(将某个方程的$k$倍加到另一个方程上)使得最后每个方程都变成左边只含一个变量(如果将多余的变量视作常量的话). 我们可以将方程组的系数提取出来操作: $$\left(
\begin{array}{ccc|c}
  4 & 3 & -6 & -17 \\
  1 & 2 & 3 & 22 \\
  2 & 3 & 0 & 11
\end{array}
\right) \quad \stackrel{\textit{化简}}{\longrightarrow} \quad \left(
\begin{array}{ccc|c}
  1 & 0 & 0 & 1 \\
  0 & 1 & 0 & 3 \\
  0 & 0 & 1 & 5
\end{array}
\right).$$
左侧矩阵被称作方程组的\textit{增广矩阵}. 从化简的结果来看, 我们得到了$x_1=1,x_2=3,x_3=5$. 需要注意的是, 为了保证$x_1,x_2,x_3$顺次排列, 可能还会用到对换变换(将矩阵的两行调换位置). 

有些时候不一定能够得到每个变量的固定解, 而是需要用一部分变量来表示另一部分变量. 例如: $$\begin{cases}
	2x_1+3x_2=5 \\ x_1+x_4=8 \\ x_1+x_2+x_3=4
\end{cases}  \stackrel{\textit{表示}}{\longrightarrow} \quad \left(
\begin{array}{cccc|c}
  2 & 3 & 0 & 0 & 5 \\
  1 & 0 & 0 & 1 & 8 \\
  1 & 1 & 1 & 0 & 4
\end{array}
\right) \quad \stackrel{\textit{化简}}{\longrightarrow} \quad \left(
\begin{array}{cccc|c}
  1 & 0 & 0 & \frac{1}{2} & \frac{65}{8} \\
  0 & 1 & 0 & -\frac{1}{3} & -\frac{15}{4} \\
  0 & 0 & 1 & -\frac{2}{3} & -\frac{1}{3}
\end{array}
\right).$$
从而$x_1 = -\frac{1}{2}x_4+\frac{65}{8}, x_2=\frac{1}{3}x_4-\frac{15}{4},x_3=\frac{2}{3}x_4-\frac{1}{3}$. 

从上述例子中, 尝试提取一些共性: 通过三种变换(倍乘变换, 倍加变换, 对换变换, 统称为\textit{基本变换}), 我们总是能将一个增广矩阵化简为一部分列恰存在一个$1$元素的矩阵.(利用归纳法容易证明) 这样的列对应主变量, 称作\textit{主列}; 而剩下的列对应自由变量, 称作\textit{自由列}. 最后化简出来的矩阵称作原矩阵$A$的\textit{行简化阶梯型矩阵}, 记作$\textrm{rref}(A)$. 

容易验证, 一个齐次线性方程组只有零解当且仅当其系数矩阵的行简化阶梯型矩阵的阶梯数等于未知数个数; 一个非齐次线性方程组有解当且仅当增广矩阵和系数矩阵的行简化阶梯型矩阵的阶梯数相等. 

既然是变换, 能否找到其对应的(线性)映射呢? 在$\F ^n$上考虑, 我们从最简单的恒等变换开始: 

设$T \in \lmap (\F ^n,\F ^n)$满足$T:v \mapsto v$, 由$Te_i=e_i$可知(其中空白部分为$0$)$$T=\left( \begin{smallmatrix}
	 1 &   &   &   \\
	   & 1 &   &   \\
	   &   &\ddots &   \\
	   &   &   & 1     
\end{smallmatrix} \right) =: I_n.$$
将列向量$v$替换为行向量的列向量即可验证对任意矩阵$A$都有$I_nA=AI_n=A$. 

在$I_n$的基础上, 可以分别得到倍乘变换、倍加变换、对换变换的表示矩阵:$$\left( \begin{smallmatrix}
	\ddots &   &   &  &   \\
	       & 1 &   &  &   \\
	       &   &k  &  &   \\
	       &   &   & 1&   \\
	       &   &   &  &\ddots  
\end{smallmatrix} \right),\quad \left( \begin{smallmatrix}
	\ddots &   &       &  &   \\
	       & 1 &       &  &   \\
	       &   &\ddots &  &   \\
	       & k &       & 1&   \\
	       &   &       &  &\ddots  
\end{smallmatrix} \right),\quad \left( \begin{smallmatrix}
	\ddots &   &   &   &   &   &   &   &   \\
	       & 1 &   &   &   &   &   &   &   \\
	       &   & 0 &   &   &   & 1 &   &   \\
	       &   &   & 1 &   &   &   &   &   \\
	       &   &   &   & \ddots  &   &   &   &   \\
	       &   &   &   &   & 1 &   &   &   \\
	       &   & 1 &   &   &   & 0 &   &   \\
	       &   &   &   &   &   &   & 1 &   \\
	       &   &   &   &   &   &   &   & \ddots
\end{smallmatrix} \right).$$实际上这些矩阵都是对$I_n$做一次变换得到的. 

从线性映射的角度, 设初等变换的表示矩阵$R$, 则$RA$表示将$A$进行初等\textbf{行}变换. 相对应地, $AR$则表示将$A$进行初等\textbf{列}变换, 其法则类似于行变换. 

\subsection{可逆矩阵}

\begin{definition}{矩阵的逆}
	设$A$是$n$阶方阵, 若存在$n$阶方阵$B$满足$AB=BA=I_n$, 则称$A$是\textit{可逆的}, $B$为其\textit{逆}. 
\end{definition}
\begin{remark}
	广义的矩阵逆定义如上. 然而, 研究矩阵的“左逆”和“右逆”往往更有趣(这是因为, $AB=I_n$足以推出$BA=I_n$了, 这一点将在后文说明), 只不过原书中没有涉及这一部分内容, 只好忍痛删去. 
\end{remark}

容易验证, 一个矩阵若可逆, 则其逆一定唯一. 因此, 我们得以用$A^{-1}$来指代$A$的逆. 

举一些可逆矩阵的例子: $I_n$的逆是它本身, 旋转$\theta$变换的逆是旋转$-\theta$, 初等变换的逆就是对应操作的逆. 一个重要的例子是所谓对角矩阵: $$\textrm{diag} (a_1, \cdots ,a_n) = \left( \begin{smallmatrix}
	 a_1 &   &   &   \\
	   & a_2 &   &   \\
	   &   &\ddots &   \\
	   &   &   & a_n     
\end{smallmatrix} \right)=:D,~\textit{若}~a_1\cdots a_n \neq 0~\textit{则$D$的逆存在且} ~D^{-1} = \textrm{diag} (a_1^{-1},\cdots ,a_n^{-1}). $$

容易验证如下命题. 

\begin{proposition}{矩阵的逆的运算}
	设$A,B$是可逆矩阵, 则$$1.~~(A^{\T})^{-1} = \left(A^{-1}\right)^{\T} ,\qquad 2.~~(AB)^{-1}=B^{-1}A^{-1}.$$
\end{proposition}

\begin{proposition}{矩阵可逆的条件}
	设$n$阶方阵$A$, 则以下说法等价:
	\begin{center}
		a) $A$可逆. \qquad b) 齐次线性方程组$Ax=0$的解唯一. \qquad c) $\textrm{rref}~ A=I_n$. \\ d) $A$是有限个初等矩阵的乘积. 
	\end{center}
\end{proposition}
\begin{proof}
	采用轮换证法. 
	a) $\Rightarrow $ b): $A$可逆即其对应的线性映射可逆, 亦等价于该线性映射为双射. 满射性可以导出$Ax=0$存在解, 单射性可以导出$Ax=0$解唯一. \\
	b) $\Rightarrow $ c): 显然. \\
	c) $\Rightarrow $ d): 在将$A$化简为$\textrm{rref}~A$即$I_n$时, 会应用有限个可逆的初等行变换$E_1,\cdots E_n$, 即$A=E_1\cdots E_nI_n$. \\
	d) $\Rightarrow $ a): $A^{-1} = E_n^{-1}\cdots E_1^{-1}$. 
\end{proof}

由该命题可以得到一种计算矩阵的逆的方法. 对于$n$阶矩阵$A$, 将矩阵$\begin{pmatrix}
	A & I_n
\end{pmatrix}$变换为$\begin{pmatrix}
	I_n & E
\end{pmatrix}$, 所需初等行变换的乘积为$E$, 那么$EA=I_n$, 进一步$A^{-1} = EAA^{-1}=E$. 

有一些特殊的矩阵的逆需要知晓. 

定义\textit{上三角矩阵}为对角线左下方的元素均为$0$的矩阵. 那么其可逆当且仅当其对角元素均不为$0$. 逆矩阵也是上三角矩阵, 且对角线元素为原矩阵对角线元素的倒数. \textit{下三角矩阵}类似. 

定义\textit{(行)对角占优矩阵}$A=\begin{pmatrix}
	a_{ij}
\end{pmatrix}_{n\times n}$满足对所有$i=1,\cdots ,n$有$|a_{ii}|>\sum_{j\neq i}a_{ij}$. 对角占优矩阵可逆. 

\begin{proof}
	即证$Ax=0$只有$0$解. 不然, 设$x_i$为$x$的分量中绝对值最大的. 考虑第$i$个方程$$a_{i1}x_1 + \cdots + a_{ii}x_i + \cdots + a_{in}x_n = 0.$$
	显然$x_i \neq 0$, 则$$|a_{ii}||x_i| = \left| \sum_{j\neq i} a_{ij}x_i \right| \leq \sum_{j\neq i} |a_{ij}||x_j| \leq |x_i|\sum_{j\neq i} |a_{ij}| < |a_{ii}||x_i|.$$
	矛盾! 
\end{proof}

\subsection{零空间与值域}

\begin{definition}{零空间}
	对于$T \in \lmap (V,W)$,$T$的\textit{零空间}(或称为“核”)定义如下:$$\nul T = \{ v \in V:Tv=0 \}$$
\end{definition}

\begin{example}{\examplefont{零空间的例子}}
	(1)若$T$是$V$到$W$的零映射,则$\nul T=V$. \\
	(2)设$\varphi \in \lmap (\C ^3,\F)$定义为$\varphi (z_1,z_2,z_3)=z_1+2z_2+3z_3$.则$$\nul \varphi = \{ (z_1,z_2,z_3) \in \C ^3 : z_1+2z_2+3z_3=0 \}$$
	并且$\nul \varphi$的一个基为$(-2,1,0),(-3,0,1)$. \\
	(3)设$D \in \lmap (\mathcal{P}(\R),\mathcal{P}(\R))$是微分映射$Dp=p'$.只有常函数的导数才能等于零函数,故$T$的零空间是常函数组成的集合. \\
	(4)设$T \in \lmap (\F ^{\infty},\F ^{\infty})$是向后移位映射$$T(x_1,x_2,x_3, \cdots )=(x_2,x_3,\cdots )$$
	为让$Tv=0$,要求$x_2=x_3=\cdots =0$,故$\nul T = \{ (a,0,0,\cdots ) :a \in \F \}$.
\end{example}

自然地,零空间是向量空间.

\begin{proposition}{零空间是子空间}
	设$T \in \lmap (V,W)$,则$\nul T$是$V$的子空间.
\end{proposition}
\begin{proof}
	略.为证明上述命题,只需注意到$T(0)=0$~(因为$T(0+0)=T(0)+T(0)$).
\end{proof}

\begin{definition}{单射}
	如果当$Tu=Tv$时必有$u=v$,则称映射$T:V \to W$是单射.
\end{definition}

\begin{proposition}{单射性的判定}
	设$T \in \lmap (V,W)$,则$T$是单射当且仅当$\nul T=\{ 0 \}$.
\end{proposition}
\begin{proof}
	\buzhou{1} 充分性:当$\nul T = \{ 0 \}$时,设$Tu=Tv$,则$Tu-Tv=T(u-v)=0$,于是$u-v=0$,即$u=v$.这表明$T$是单射. \\
	\buzhou{2} 必要性:任取$\nul T$中的元素$v$,则$Tv=0$.因为$T0=0$且$T$是单射,所以必有$v=0$,即$\nul T = \{ 0 \}$.
\end{proof}

\begin{definition}{值域}
	对于$T \in \lmap (V,W)$,$T$的\textit{值域}(或称为“像”)定义如下:$$\rge T = \{ Tv : v \in V \}$$
\end{definition}

\begin{example}{\examplefont{值域的例子}}
	(1)若$T$是$V$到$W$的零映射,则$\rge T=\{ 0 \}$. \\
	(2)设$T \in \lmap (\R ^2,\R ^3)$定义为$T(x,y)=(2x,5y,x+y)$,则$$\rge T = \{ (2x,5y,x+y):x,y \in \R \}$$
	并且$\rge T$的一个基为$(2,0,1),(0,5,1)$. \\
	(3)设$D \in \lmap (\mathcal{P}(\R ),\mathcal{P}(\R ))$是微分映射$Dp=p'$.由于对每个多项式$q \in \mathcal{P}(\R )$均存在多项式$p$使得$p'=q$,故$D$的值域为$\mathcal{P}(\R )$.
\end{example}

自然地,值域是向量空间.

\begin{proposition}{值域是子空间}
	设$T \in \lmap (V,W)$,则$\rge T$是$V$的子空间.
\end{proposition}
\begin{proof}
	略.
\end{proof}

\begin{definition}{满射}
	如果函数$T:V \to W$的值域等于$W$,则称$T$为\textit{满射}.
\end{definition}

上述例子中只有微分映射是满的.

\subsection{线性映射基本定理}

\begin{proposition}{线性映射基本定理}
	设$V$是有限维的,$T \in \lmap (V,W)$.则$\rge T$是有限维的并且$$\dim V = \dim \nul T + \dim \rge T.$$
\end{proposition}
\begin{proof}
	设$v_1, \cdots ,v_m$是$\nul T$的基.将其扩展为$V$的一个基$v_1, \cdots ,v_m ,u_1, \cdots ,u_n$.注意到原命题等价于证明$\dim \rge T = n$,于是下证$Tu_1, \cdots ,Tu_n$为$T$的基: \\
	首先,任取$v \in V$,记$v=a_1v_1 + \cdots + a_mv_m + b_1u_1 + \cdots + b_nu_n$,则$$Tv = a_1Tv_1 + \cdots + a_mTv_m + b_1Tu_1 + \cdots + b_nTu_n = b_1Tu_1 + \cdots + b_nTu_n$$
	故$Tu_1, \cdots ,Tu_n$张成$\rge T$. \\
	另外,若$b_1Tu_1 + \cdots + b_nTu_n=0$,则$$T(b_1u_1 + \cdots + b_nu_n)=0$$
	这表明$b_1u_1 + \cdots + b_nu_n \in \nul T$.记$b_1u_1 + \cdots + b_nu_n = c_1v_1 + \cdots + c_mv_m$,则由$v_1, \cdots ,v_m,u_1, \cdots ,u_n$线性无关,可得$b_1= \cdots = b_n = c_1 = \cdots = c_m$,于是$Tu_1, \cdots ,Tu_n$线性无关.
\end{proof}
\begin{remark}
	需要注意的是, 证明线性映射基本定理的前提是$V$为有限维.
\end{remark}

利用线性映射基本定理,可以快速证伪某些命题.

\begin{proposition}{}
	设有限维向量空间$V,W$.若$\dim V > \dim W$,那么$V$到$W$的线性映射$T$一定不是单射;相反地,若$\dim V < \dim W$,那么$V$到$W$的线性映射$T$一定不是满射.
\end{proposition}
\begin{proof}
	只证明第一部分.由$\dim V > \dim W \geq \dim \rge T$,可知$\dim \nul T = \dim V - \dim \rge T > 0$,于是$T$不是单射.
\end{proof}

\begin{example}
	用线性映射重述齐次线性方程组是否有非零解的问题.即,对给定的正整数$m,n$,设$A_{j,k} \in \F ~(j=1,\cdots ,m,~k=1,\cdots ,n)$,考虑齐次线性方程组$$\begin{cases}
		\sum_{k=1}^{n} A_{1,k}x_k = 0 \\
		\cdots \cdots \\
		\sum_{k=1}^{n} A_{m,k}x_k = 0
	\end{cases}$$是否有不全为$0$的解.
\end{example}
\begin{solution}
	构造$T:\F ^n \to \F ^m$满足:$$T(x_1, \cdots ,x_n) = \ssb{\sum_{k=1}^{n} A_{1,k}x_k, \cdots , \sum_{k=1}^{n} A_{m,k}x_k}$$
	易于证明$T$是线性映射.则原方程有不全为$0$的解等价于$T$不是单射.由上述命题可知,若$n>m$则$T$一定不是单射.故当$n > m$时原方程组有不全为$0$的解.(即变量个数大于方程个数时)
\end{solution}

\begin{example}
	用线性映射重述是否可以选取常数项使得非齐次线性方程组无解的问题.即,对给定的正整数$m,n$,设$A_{j,k} \in \F ~(j=1,\cdots ,m,~k=1,\cdots ,n)$及$c_1, \cdots ,c_m \in \F$,考虑线性方程组$$\begin{cases}
		\sum_{k=1}^{n} A_{1,k}x_k = c_1 \\
		\cdots \cdots \\
		\sum_{k=1}^{n} A_{m,k}x_k = c_m
	\end{cases}$$
	是否存在某些常数$c_1, \cdots ,c_m$使得上述方程组无解.
\end{example}
\begin{solution}
	构造$T:\F ^n \to \F ^m$满足:$$T(x_1, \cdots ,x_n) = \ssb{\sum_{k=1}^{n} A_{1,k}x_k, \cdots , \sum_{k=1}^{n} A_{m,k}x_k}$$
	易于证明$T$是线性映射.则存在这样的一组常数等价于$T$不是满的.由上述命题可知,若$n<m$则$T$一定不是满射.故当$n < m$存在这样一组常数.(即变量个数小于方程个数时)
\end{solution}



\subsection*{习题  ~~\small 对应原书3.B习题}

\begin{exercise} %LADR 练习3B.4
	证明$\{ T \in \lmap (\R ^5,\R ^4):\dim \nul T>2 \}$不是$\lmap (\R ^5,\R ^4)$的子空间.
\end{exercise}
\begin{proof}
	设$\R ^5$的基$v_1,\cdots ,v_5$, $\R ^4$的基$w_1, \cdots ,w_4$. 构造$T_1,T_2$满足$$T_1v_1=w_1,T_1v_2=w_2,T_1v_i=0(i=3,4,5), \quad T_2v_3=w_3,T_2v_4=w_4,T_2v_i=0(i=1,2,5).$$
	容易验证$T_1,T_2 \in \{ T \in \lmap (\R ^5,\R ^4):\dim \nul T>2 \}$, 但是$\dim \nul (T_1+T_2) = 1$. 
\end{proof}

\begin{exercise} %LADR 练习3B.7
	设$V,W$都是有限维的, 且$2 \leq \dim V \leq \dim W$. 证明$\{ T \in \lmap (V,W) : T~\textit{不是单的}~ \}$不是$\lmap (V,W)$的子空间. 
\end{exercise}
\begin{proof}
	设$V$的基$v_1,\cdots ,v_m$, $U$的基$u_1,\cdots ,u_n$, 满足$2 \leq m \leq n$. 构造$T_1,T_2$满足$$T_1v_1=u_1,T_2v_2=u_2,T_1v_2=0,T_2v_1=0,\quad T_1v_i=T_2v_i=u_i(i=3,\cdots m).$$
	容易验证, $T_1,T_2$均不为单射, 但$T_1+T_2$为单射. 
\end{proof}

\begin{exercise} %LADR 练习3B.8
	设$V,W$都是有限维的, 且$2 \leq \dim W \leq \dim V$. 证明$\{ T \in \lmap (V,W) : T~\textit{不是满的}~ \}$不是$\lmap (V,W)$的子空间. 
\end{exercise}
\begin{proof}
	设$V$的基$v_1,\cdots ,v_m$, $U$的基$u_1,\cdots ,u_n$, 满足$2 \leq n \leq m$. 构造$T_1,T_2$满足$$T_1v_1=u_1,T_2v_2=u_2,T_1v_2=0,T_2v_1=0,\quad T_1v_i=T_2v_i=u_i(i=3,\cdots m),T_1v_j=T_2v_j=0(j=m+1,\cdots ,n).$$
	容易验证, $T_1,T_2$均不为满射, 但$T_1+T_2$为满射. 
\end{proof}

\begin{exercise} %LADR 练习3B.12
	设$V$是有限维的, $T \in \lmap (V,W)$. 证明$V$有一个子空间$U$使得$U \cap \nul T = \{ 0 \}$且$\rge T = \{ Tu:u\in U \}$. 
\end{exercise}
\begin{proof}
	取$U$使得$U \oplus \nul T = V$. 任取$v \in V$, 记$v=u+w$, 其中$u \in U$而$w \in \nul T$, 那么$Tv=Tu+Tw=Tu$. 由$v$的任意性知原命题成立. 
\end{proof}

\begin{exercise} %LADR 练习3B.16
	假设在$V$上存在一个线性映射, 其零空间和值域都是有限维的. 证明$V$是有限维的. 
\end{exercise}
\begin{proof}
	设$\nul T$的一组基为$u_1,\cdots ,u_m$, $\rge T$的一组基为$w_1,\cdots w_n$. 设$Tv_i=w_i$, 下面证明一个加强命题: $V$由$v_1,\cdots ,v_n,u_1,\cdots ,u_m$张成. 任取$v \in V$, 那么$$Tv=c_1w_1 + \cdots + c_nw_n=T(c_1v_1 + \cdots + c_nv_n),$$从而$$v-c_1v_1 + \cdots + c_nv_n = a_1u_1 + \cdots + a_mu_m.$$ 
\end{proof}

\begin{exercise} %LADR 练习3B.22,23
	设$U,V$都是有限维的, 并设$S \in \lmap (V,W), T \in \lmap (U,V)$. 证明: $$\dim \nul ST \leq \dim \nul S + \dim \nul T,\qquad \dim \rge ST \leq \min \{ \dim \rge S,\dim \rge T \}.$$
\end{exercise}
\begin{proof}
	(1) 构造映射$T'=T|_{\nul ST}$. 那么$$\dim \nul ST = \dim \nul T' + \dim \rge T' \leq \dim \nul T + \dim \nul S.$$
	(2) 显然$\dim \rge ST \leq \dim \rge S$. 另一方面, 设$\rge T$的一组基$v_1,\cdots ,v_m$, 则容易验证$Sv_1,\cdots Sv_m$张成$\rge ST$, 从而$\dim \rge ST \leq \dim \rge T$.
\end{proof}

\begin{exercise} %LADR 练习3B.24
	设$W$是有限维的, 并设$T_1,T_2 \in \lmap (V,W)$. 证明$\nul T_1 \subset \nul T_2$当且仅当存在$S \in \lmap (W,W)$使得$T_2=ST_1$. 
\end{exercise}
\begin{proof}
	充分性显然, 下证必要性: 假设$\nul T_1 \subset \nul T_2$. 按照练习3.10的方法将$V$分解为$\nul T_1 \oplus U$. 由于$\rge T_1=\{ Tu:u \in U \}$为有限维的, $U$也应是有限维的. 设$U$的一组基为$u_1, \cdots ,u_n$. 那么对于任意的$v \in V$, 可以写出$$v=c_1u_1+ \cdots + c_nu_n+ \tilde{v}, \quad \textit{其中}~\tilde{v} \in \nul T_1.$$
	从而$T_2v=c_1T_2u_1 + \cdots + c_nT_2u_n$. 只需取$S$使得$ST_1v_i=T_2v_i$即可, 容易验证$S$是线性映射. 
\end{proof}

\begin{exercise} %LADR 练习3B.25
	设$V$是有限维的, 并设$T_1,T_2 \in \lmap (V,W)$. 证明$\rge T_1 \subset \rge T_2$当且仅当存在$S \in \lmap (V,V)$使得$T_1=ST_2$. 
\end{exercise}
\begin{proof}
	充分性显然. 必要性: 设$V$的一组基为$v_1, \cdots ,v_m$, 则存在$u_1, \cdots ,u_m \in V$使得$T_1v_i=T_2u_i$. 令$S$满足$Sv_i=u_i$即可. 
\end{proof}

\begin{exercise} %LADR 练习3B.26
	设$D \in \lmap (\mathcal{P}(\R),\mathcal{P}(\R))$使得对每个非常数多项式$p \in \mathcal{P}(\R)$均有$\deg Dp=\deg p -1$. 证明$D$是满的. 
\end{exercise}
\begin{proof}
	对于$x^n$有$\deg Dx^n=n-1$. 注意到$\mathcal{P}(\R) = \spn (Dx,Dx^2,\cdots) \subset \rge D \subset \mathcal{P}(\R)$, 所以$D$是满的. 
\end{proof}

\begin{exercise} %LADR 练习3B.29
	设$\varphi \in \lmap (V,\F)$. 假定$u \in V$不属于$\nul \varphi$, 证明$V=\nul \varphi \oplus \{ au:a \in \F \}$. 
\end{exercise}
\begin{proof}
	将$v\in V$分解为某个$\nul \varphi$里的元素与$au$的和. 我们考虑$T(v-au)=Tv-aTu=0$, 即取$a=Tv/Tu$, 这样就证明了$V=\nul \varphi + \{ au:a \in \F \}$. 另一方面, 假设$au \in \nul \varphi$, 那么$T(au)=aTu=0$但是$Tu \neq 0$, 只有$a=0$, 说明$V=\nul \varphi \oplus \{ au:a \in \F \}$.
\end{proof}

\begin{exercise} %LADR 练习3B.30
	设$\varphi _1,\varphi _2 \in \lmap (V,\F)$, 且具有相同的零空间. 证明存在常数$c \in \F$使得$\varphi _1 = c\varphi _2$. 
\end{exercise}
\begin{hint}
	利用上一题的结论. 
\end{hint}
\begin{proof}
	若$\nul \varphi _1=\nul \varphi _2=V$, 则$\varphi _1 = \varphi _2=0$. 否则考虑$u\in V$且$u \notin \nul \varphi _1$. 由上一题可知$V = \nul \varphi _1 \oplus \{ au:a \in \F \}$. 任取$v \in V$, 将其表示为$v = w + au$, 那么$\frac{\varphi _1(v)}{\varphi _2(v)} = \frac{\varphi _1(u)}{\varphi _2(u)}$, 为定值. 
\end{proof}

\section{可逆性与同构的向量空间}

\subsection{线性映射的可逆性}

类似于一般的函数,我们可以定义线性映射的可逆性:

\begin{definition}{线性映射的可逆性}
	线性映射$T \in \lmap (V,W)$称为\textit{可逆的},如果存在线性映射$S \in \lmap (W,V)$使得$ST$等于$V$上的恒等映射且$TS$等于$W$上的恒等映射.这样的$S$称作$T$的\textit{逆},记为$T^{-1}$.
\end{definition}

这里的“逆”,在线性映射的乘法意义下,即为其乘法逆元.自然它是唯一的.

\begin{proposition}{}
	可逆的线性映射有唯一的逆.
\end{proposition}
\begin{proof}
	设$T \in \lmap (V,W)$可逆,且$S_1,S_2$均为$T$的不同的逆.由于$$S_1 = S_1I = S_1(TS_2) = (S_1T)S_2 = IS_2 = S_2$$
	这与假设矛盾.故$T$的逆是唯一的.
\end{proof}

以映射的观点来看,一个函数可逆当且仅当它是双射.这一点对于线性映射也成立.

\begin{proposition}{线性映射可逆性的判定}
	一个线性映射是可逆的当且仅当它既是单的又是满的.
\end{proposition}
\begin{proof}
	\buzhou{1} 必要性:设$T \in \lmap (V,W)$是可逆的.设$u_1,u_2 \in V$使得$Tu_1 = Tu_2$,那么$$u_1 = T^{-1} T u_1 = T^{-1} T u_2 = u_2.$$
	于是$T$是单的.另一方面,设$w \in W$,则由$w = T(T^{-1}w)$可知$W \subseteq \rge T$,又$\rge \subseteq W$,则$W = \rge T$,即$T$是满的. \\
	\buzhou{2} 充分性:设$T$既是单的又是满的,构造映射$S$满足:对于每个$w \in W$,$Sw$表示使得$T(Sw)=w$成立的$V$中的唯一元素(这里的存在与唯一可以由$T$的单射与满射得到).我们证明$S$是线性映射且$ST$是$V$上的恒等映射. \\
	首先,设$w_1,w_2 \in W$,由于$$T(Sw_1+Sw_2)=TSw_1 + TSw_2 = w_1 + w_2,$$
	$$T(S(w_1+w_2)) = w_1+w_2,$$
	所以$S(w_1+w_2)=Sw_1 + Sw_2$.类似地可得$S$的齐性.故$S$是线性映射. \\
	接着,任取$v \in V$,由于$$T(STv) = (TS)Tv=ITv=Tv.$$
	所以$STv=v$,即$ST$是$V$上的恒等映射.
\end{proof}

其实我们可以做出更为细致的分析:

\begin{proposition}{} %LADR 练习3.20,21
	(1) 设$W$是有限维的, $T \in \lmap (V,W)$, 则$T$是单的当且仅当存在$S \in \lmap (W,V)$使得$ST=I$. \\
	(2) 设$V$是有限维的, $T \in \lmap (V,W)$, 则$T$是满的当且仅当存在$S \in \lmap (W,V)$使得$TS=I$.
\end{proposition}
\begin{proof}
	(1) "$\Rightarrow$": 假设$T$是单的. 记$S':\rge T \to V$使得$S'(Tv)=v$(由于$Tv$一定对应一个$v$, 这是良定义的). 将$S'$延伸到$S:W \to V$即可. \\
	"$\Leftarrow$": 设$u,v$使得$Tu=Tv$, 则$ST(u)=ST(v)$, 从而$u=v$. \\
	(2) "$\Rightarrow$": 假设$T$是满的. 设$Tv_1,\cdots ,Tv_m$是$W$的一组基(由于任意$W$中元素都能表示为$Tv$的形式, 这是良定义的). 记$S:W \to V$使得$S(c_1Tv_1+\cdots + c_mTv_m)=c_1v_1+\cdots + c_mv_m$. 容易验证$S$是线性的. \\
	"$\Leftarrow$": 对于任意$w \in W$, $TS(w)=w$, 即$T$是满的.
\end{proof}

一般来说, 单射性和满射性并不相互蕴含. 然而对于从向量空间映射到自身的线性映射, 称为\textit{算子}, 满射性和单射性是等价的(如果有限维). 设$V$是有限维的, $T \in \lmap (V)$, 则$T$是单的等价于$\dim \nul T = 0$, $T$是满的等价于$\dim \rge T = \dim V$, 由线性映射基本定理就可以得到结论. 利用这一点我们可以得到: 

\begin{proposition}{}
	若$V,W$都是有限维向量空间, $S \in \lmap (V,W), T \in \lmap (W,V)$, 那么$ST=I$等价于$TS=I$. 
\end{proposition}
\begin{proof}
	只证明必要性. 由上一个命题我们知道$T$是单的, 则$T$是满的, 进而$T$可逆. 所以$ST=I$就得到$S=T^{-1}$, 进一步有$TS = TT^{-1} = I$. 
\end{proof}

\subsection{同构的向量空间}

在高中数学中,“同构”这个词被大量滥用,但其也能为我们揭示同构的内涵.例如,我们说等式$$x(\ln x+1) = ye^{y-1}$$关于$x$和$y$是同构的,是因为若作换元$y=\ln t+1$,可得$x(\ln x +1) = t(\ln t +1)$.

为什么$x$与$y$“同构”呢?因为$y$和$x$可以通过一个映射联系起来\footnote{请注意, 这是直观而不严格的说法. } .类似地,我们正式给出两个向量空间的同构定义:

\begin{definition}{向量空间的同构}
	\textit{同构}就是可逆的线性映射.若两个向量空间之间存在一个同构,则称这两个向量空间是\textit{同构的}.
\end{definition}

同构$T:V \to W$做了一步操作,将$v \in V$重新标记为$Tv \in W$;$T$的逆$T^{-1}$同等地将每个$Tv \in W$重新标记为$v \in V$.于是$V$与$W$中的元素只是形式不一样,其性质是一样的.

回想之前提到的“矩阵乘法和线性映射乘法的代数性质一致”这件事,本质上是因为$\lmap (V,W)$与$\F ^{m,n}$同构.

\begin{proposition}{$\lmap (V,W)$与$\F ^{m,n}$同构}
	设$v_1, \cdots ,v_n$是$V$的基,$w_1, \cdots ,w_m$是$W$的基,则$\mmatrix$是$\lmap (V,W)$与$\F ^{m,n}$之间的一个同构.
\end{proposition}
\begin{proof}
	将$\mmatrix$视作一个映射,那么由命题\ref{pro:xmxkykueyysrjuvf}可知它是线性的.现在只需证明它可逆. 
	
	一方面,若对于$T \in \lmap (V,W)$,$\mmatrix (T)=0$,则由定义可得$Tv_k=0,~k=1,\cdots ,n$,那么$T(c_1v_1 + \cdots + c_nv_n)=c_1Tv_1 + \cdots + c_nTv_n =0$,即$\nul \mmatrix = \{ 0 \}$,于是$\mmatrix$是单的. 
	
	另一方面,任取$A \in \F ^{m,n}$,构作线性映射$T:V \to W$满足$$Tv_k = \sum_{j=1}^{m} A_{j,k} w_j$$
	则$\mmatrix (T) =A$.这表明$\mmatrix$是满射.
\end{proof}

作为一个例子, 我们注意到为证明$\dim \lmap (V,W) = \dim V \cdot \dim W$需要一些构造, 但是将其转化为矩阵空间则是显然的. 这启示我们可以从与一个向量空间同构的另一个更简单(或已知)的向量空间来考察该向量空间.

注意到一个问题:类似于高中数学中“集合的对应原理”:若两个有限集合$A,B$之间存在一个双射$f$,则$|A|=|B|$.两个向量空间同构,它们的维数应当相同.而更进一步,由于“维数相同”这一概念比“集合元素个数相等”更强,上面的说法反过来也可以是对的.

\begin{proposition}{向量空间同构的判定}
	$\F$上两个有限维向量空间同构当且仅当其维数相同.
\end{proposition}
\begin{proof}
	\buzhou{1} 必要性:设$V$和$W$是同构的有限维向量空间,即存在可逆的线性映射$T:V \to W$.于是$\dim \nul T = 0,~\dim \rge T =\dim W$.又由线性映射基本定理可知$$\dim V = \dim \nul T + \dim \rge T = \dim W$$
	\buzhou{2} 充分性:设$V$和$W$维数相同,$v_1, \cdots ,v_n$是$V$的基,$w_1, \cdots ,w_n$是$W$的基.由命题\ref{pro:xmxkykuedkyiyu}可知存在一个线性映射$T:V \to W$满足$$T(c_1v_1 + \cdots + c_nv_n)=c_1w_1 + \cdots + c_nw_n$$
	只需证明这个$T$是可逆的.实际上,若$T(c_1v_1 + \cdots + c_nv_n)=0$,由于$w_1, \cdots ,w_n$是线性无关的,必有$c_1= \cdots = c_n=0$,即$\nul T = \{ 0 \}$,即$T$是单的;另一方面,等式右侧是$w_1, \cdots ,w_n$的线性组合形式,于是$\rge T = W$,即$T$是满的.
\end{proof}

\subsection{基的变换}

在开始下方的命题之前, 我们先来看一个显然而有用的等式: 设$T \in \lmap (V,W)$, $V$和$W$的一组基分别为$v_1,\cdots ,v_n, w_1,\cdots ,w_m$, 记$A = \mathcal{M} (T, (v_1,\cdots ,v_n),( w_1,\cdots ,w_m))$, 那么$$\begin{pmatrix}
 w_1 & \cdots & w_m
\end{pmatrix} A = \begin{pmatrix}
 Tv_1 & \cdots & Tv_n
\end{pmatrix}.$$注意, 由于我们不在乎左右两侧的矩阵对应的线性映射, 请不要认为它们和基的选取相关. 

\begin{proposition}{线性映射之积的矩阵}
	设$T \in \lmap (U,V), S \in \lmap (V,W)$. 我们有
	\begin{align*}
		\mathcal{M} (ST, &(u_1,\cdots ,u_m), (w_1,\cdots ,w_p)) = \\
		&\mathcal{M} (S,(v_1,\cdots ,v_n),(w_1,\cdots ,w_p)) \mathcal{M} (T,(u_1,\cdots ,u_m),(v_1,\cdots ,v_n)).
	\end{align*}
\end{proposition}
\begin{proof}
	我们定义矩阵乘法就是为了这个命题. 
\end{proof}

\begin{example}
	矩阵$$\mathcal{M} (I,(u_1,\cdots ,u_n),(v_1,\cdots ,v_n)),\qquad \mathcal{M} (I,(v_1,\cdots ,v_n),(u_1,\cdots ,u_n))$$
	是可逆的, 且互为逆矩阵. 
\end{example}

\begin{proposition}{换基公式}
	设$T\in \lmap (V)$, $V$有两组基$v_1,\cdots ,v_n$和$u_1,\cdots ,u_n$. 记$$A = \mathcal{M}(T,(u_1,\cdots ,u_n)),\qquad B=\mathcal{M}(T,(v_1,\cdots ,v_n)),\qquad C=\mathcal{M}(I,(u_1,\cdots ,u_n),(v_1,\cdots ,v_n)).$$
	那么我们有$$A=C^{-1}BC.$$
\end{proposition}

形式化的证明直接来源于前面的命题, 这里省略. 上述命题的另一种理解方式是$$\begin{pmatrix}
 u_1 & \cdots & u_n
\end{pmatrix} A = \begin{pmatrix}
 Tu_1 & \cdots & Tu_n
\end{pmatrix},\qquad \begin{pmatrix}
 v_1 & \cdots & v_n
\end{pmatrix} B = \begin{pmatrix}
 Tv_1 & \cdots & Tv_n
\end{pmatrix}, $$
$$\begin{pmatrix}
 v_1 & \cdots & v_n
\end{pmatrix} C = \begin{pmatrix}
 u_1 & \cdots & u_n
\end{pmatrix}. $$
只需注意到$$\begin{pmatrix}
 Tv_1 & \cdots & Tv_n
\end{pmatrix} = T\begin{pmatrix}
 v_1 & \cdots & v_n
\end{pmatrix}, $$
即可完成同样的证明. 

\newpage
\subsection*{习题  ~~\small 对应原书3.D习题}

\begin{exercise} %(3e) 4
	设$W$是有限维的, $S,T \in \lmap (V,W)$. 证明: $\nul S = \nul T$当且仅当存在可逆的算子$E \in \lmap (W)$使得$S=ET$. 
\end{exercise}
\vspace{1em}

\begin{exercise} %(3e) 5
	设$V$是有限维的, $S,T \in \lmap (V,W)$. 证明: $\rge S = \rge T$当且仅当存在可逆的算子$E \in \lmap (V)$使得$S=TE$. 
\end{exercise}
\vspace{1em}

\begin{exercise} %(3e) 6
	设$V,W$都是有限维的, $S,T \in \lmap (V,W)$. 证明$\dim \nul S = \dim \nul T$当且仅当存在可逆的$E_1 \in \lmap (V)$和$E_2 \in \lmap (W)$使得$S=E_2TE_1$. 
\end{exercise}
\vspace{1em}

\begin{exercise} %(3e) 16
	设$V$是有限维的, $T \in \lmap (V)$. 证明: $T$是标量乘以恒等映射当且仅当对每个$S \in \lmap (V)$都有$ST=TS$. 
\end{exercise}
\vspace{1em}

\begin{exercise} %(3e) 17
	设$V$是有限维的, $\mathcal{E}$是$\lmap (V)$的子空间使得对所有$S \in \lmap (V)$和$T \in \mathcal{E}$都有$ST \in \mathcal{E}$和$TS \in \mathcal{E}$. 证明$\mathcal{E}=\{ 0 \}$或$\mathcal{E}=\lmap (V)$. 
\end{exercise}
\vspace{1em}

\begin{exercise} %(4e) 10
	设$V,W$都是有限维的, $U$是$V$的一个子空间. 记$\mathcal{E} = \{ T \in \lmap (V,W):U \subseteq \nul T \}$. 
	
	(a) 证明$\mathcal{E}$是$\lmap (V,W)$的一个子空间. 
	
	(b) 求$\dim \mathcal{E}$. 
\end{exercise}
\vspace{1em}

\begin{exercise} %(4e) 19
	设$V$是有限维的, $T \in \lmap (V)$. 证明: $T$对$V$的任何一组基有相同的表示矩阵, 当且仅当$T$是标量乘以恒等映射. 
\end{exercise}
\vspace{1em}

\begin{exercise} %(3e) 19
	设$T \in \lmap (\mathcal{P} (\R))$是单的, 且对每个非零多项式$p \in \mathcal{P}(\R)$都有$\deg Tp \leq \deg p$. 
	
	(a) 证明$T$是满的. 
	
	(b) 证明对每个非零多项式$p \in \mathcal{P}(\R)$都有$\deg Tp = \deg p$. 
\end{exercise}


\newpage
\section{向量空间的积与商}



\section{对偶与矩阵的秩}


% LADR 新增内容: Column–Row Factorization and Rank of a Matrix (3C)




