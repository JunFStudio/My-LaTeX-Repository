\usepackage{fontawesome}
\usepackage[many]{tcolorbox}
\usetikzlibrary{arrows,positioning,calc,fadings,shapes,decorations.markings}
\usepackage{circuitikz}
\usepackage{float}
\definecolor{lsp}{RGB}{0,174,247}
\usepackage{tikz}
\usepackage{mathtools}
\usepackage{mdframed}
\BeforeBeginEnvironment{minted}{\begin{mdframed}[backgroundcolor=gray!5,hidealllines=true,innerrightmargin=0pt,linecolor=gray!70,skipabove=2pt,everyline=true,leftmargin=3mm,innerbottommargin=0pt]}
\AfterEndEnvironment{minted}{\end{mdframed}}
\usetikzlibrary{backgrounds,calc,shadows}
\usepackage[object=vectorian]{pgfornament} %% 
\usepackage{fontspec}
\setmonofont{CMU Typewriter Text}
\usepackage{xcolor}
\tcbuselibrary{minted}
\tcbuselibrary{listings}
\definecolor{mlblue3}{RGB}{44, 49, 51} % mycmd颜色
\definecolor{mlblue2}{RGB}{35, 41, 55}
\definecolor{bg}{rgb}{0.99,0.99,0.99}
\definecolor{Green}{rgb}{199, 231, 228}
\definecolor{mlblue}{RGB}{236,243,255}
\definecolor{mblue}{RGB}{0,123,255}
\newcommand{\md}[1]{{\color{purple}#1}}
\newcommand{\keypoint}[1]{\textbf{\textcolor{red}{#1}}}
\newcommand{\cmt}{\noindent\hspace{-0.25em}\textcolor{Green}{\ding{226}} \hspace{0.2em}}
\newcommand{\sol}{\noindent\hspace{-0.12em}\textcolor{cyan}{\ding{45}} \hspace{0.2em}}
\newcommand{\solc}{\noindent\hspace{-0.12em}\textcolor{cyan}{\ding{45}} \hspace{0.2em} \vspace*{-\baselineskip}}
\definecolor{mlred}{RGB}{253,243,242}
\definecolor{mred}{RGB}{220,53,69}
% highlight environment
\newenvironment{ilight}
{\centering
	\vspace*{6pt}
	\begin{tcolorbox}[colframe=Gray,colback=LightGrey!15]
		\setlength{\baselineskip}{\baselineskip}%
	}
	{\end{tcolorbox}\vspace*{-4pt}}

%%%%%%%%%%%%%%%%%%%%%%%%%%%%%%%
%mycmd定制
%%%%%%%%%%%%%%%%%%%%%%%%%%%%%%%
\newtcblisting{mycmd}[2]{colback=black,
	colupper=white,
	colback=mlblue3,
	frame style={opacity=0.25},
	colframe=mlblue3,
	center title,
	left = 2pt,
	width=\linewidth,
	listing only,
	drop shadow,
	right = 2pt,
	top = 2mm,
	boxsep =2pt,
	arc=2pt,
	fonttitle=\small\ttfamily\bfseries,
	title ={\faCode \hspace*{\fill} #1 \hspace*{\fill}\faRemove\\
		文件(F)\quad 动作(A)\quad 编辑(E)\quad 查看(V)\quad 帮助(H)\hspace*{\fill}},
	listing options={style=tcblatex,language=sh},
	every listing line={\textcolor{red}{\small\ttfamily\bfseries #2 \$> }}
}

%%%%%%%%%%%%%%%%%%%%%%%%%%%%%%%
%mycmd2定制
%%%%%%%%%%%%%%%%%%%%%%%%%%%%%%%
\newtcolorbox{mycmd2}[1]{colback=black,
	colupper=white,
	colback=mlblue3,
	frame style={opacity=0.25},
	colframe=mlblue3,
	center title,
	left = 2pt,
	width=\linewidth,
	listing only,
	drop shadow,
	right = 2pt,
	top = 1mm,
	boxsep =1pt,
	arc=2pt,
	fonttitle=\small\ttfamily\bfseries,
	title ={\faCode \hspace*{\fill} #1 \hspace*{\fill}\faRemove\\文件(F)\quad 动作(A)\quad 编辑(E)\quad 查看(V)\quad 帮助(H)\hspace*{\fill}},
	listing options={style=tcblatex,language=sh},}

\lstdefinestyle{HTML}{
	language =  bash, % 语言选Python
	frame=none,
	mathescape,
	basicstyle = {\small\ttfamily\bfseries\color{white}},
	emphstyle={ \small\ttfamily\bfseries\color{red}},
	numbers=none,
	stepnumber=2,
	numbersep=3em,
	numberstyle= \small\ttfamily\bfseries,
	keywordstyle    = \small\ttfamily\bfseries\color{red!60},
	stringstyle     =   \small\ttfamily\bfseries\color{white},
	breaklines      =   true,   % 自动换行,
	columns         =   fixed,  %字间距就不固定很丑,必须加
	basewidth       =  .5em,
	commentstyle=\small\ttfamily\bfseries\color{white},
	backgroundcolor=\color[RGB]{44, 49, 51},
	tabsize=4,
	showspaces=false,
	showstringspaces=false,
	morekeywords={ls,cd},
}
\lstdefinestyle{text}{
	language =   HTML, % 语言选Python
	frame=topline,
	mathescape,
	emphstyle={\color{frenchplum}},
	numbers=none,
	stepnumber=2,
	numbersep=3em,
	numberstyle=\tiny,
	rulecolor = \color{black},
	keywordstyle    =   \color{blue},
	stringstyle     =   \color{magenta},
	commentstyle    =   \color{red},
	breaklines      =   true,   % 自动换行,
	columns         =   fixed,  %字间距就不固定很丑,必须加
	basewidth       =   .5em,
	keywordstyle=\color{red},
	commentstyle=\color{gray},
	backgroundcolor=\color[RGB]{218, 218, 218},
	tabsize=4,
	showspaces=false,
	showstringspaces=false,
	columns=fixed,
	morekeywords={maketitle,pip,matplotlib,pandas,numpy,np,dtype,In [],Out []},
}
%%%%%%%%%%%%%%%%%%%%%%%%%%%%%%%%%
\newfontfamily\bfont{Source Code Pro}
\newminted{vim}{gobble=2,linenos}

\setminted[vim]{breaklines,
breakanywhere,
escapeinside=||,
highlightcolor=orange!10,
linenos=true,
framesep=3pt,
frame=bottomline,
framerule=2pt,
rulecolor=\color{black!35},
style=colorful,
breaklines=true,
}
\setminted[nginx]{breaklines,
breakanywhere,
escapeinside=||,
highlightcolor=green!30,
linenos=true,
framesep=3pt,
frame=bottomline,
framerule=2pt,
rulecolor=\color{black!35},
style=colorful,
breaklines=true,
}

\setminted[shell]{breaklines,
linenos=true,
bgcolor=blue!10,
breakautoindent=false,
breaksymbolleft=\raisebox{0.8ex}{
\small\reflectbox{\carriagereturn}},
breaksymbolindentleft=0pt,
frame=bottomline,
framesep=1pt,
framerule=1pt,
rulecolor=\color{black!35},
breaksymbolsepleft=0pt,
breaksymbolright=\small\carriagereturn,
breaksymbolindentright=0pt,
}
\lstdefinestyle{linux}{
	language = sh , % 语言选Python
	frame=b,
	aboveskip=3mm,
	belowskip=3mm,
	showstringspaces=false,
	columns=flexible,
	framerule=1pt,
	rulecolor=\color{black!35},
	backgroundcolor=\color{gray!5},
	basicstyle={\small\ttfamily},
	numbers=left,
	numberstyle=\tiny\color{black},
	comment=[is]{!/*}
	commentstyle=\color{green!60},
	stringstyle=\color{blue},
	breaklines=true,
	breakatwhitespace=true,
	tabsize=3,
	morekeywords={bind-utils,}
	classoffset=0,
	morekeywords={root@localhost,},keywordstyle=\color{orange},
	classoffset=1,
	morekeywords={yum,bind-utils},keywordstyle=\color{blue},
	classoffset=0,
}

\lstdefinestyle{linux2}{
	language =  bash , % 语言选Python
	frame=none,
%	aboveskip=4mm,
%	belowskip=4mm,
	showstringspaces=false,
	framesep=1pt,
	columns=flexible,
	framerule=2pt,
	aboveskip=1em,
	numbersep=12pt,
	rulecolor=\color{black},
	backgroundcolor=\color{gray!5},
	basicstyle={\small\bfont},
	numbers=left,
	numberstyle=\tiny\color{black!80},
	keywordstyle=\color{blue},
	comment=[is]{!/*}
	commentstyle=\color{green!60},
	stringstyle=\color{blue},
	breaklines=true,
	breakatwhitespace=true,
	tabsize=3,
	morekeywords={ls,mkdir,-,|},
}

\lstdefinestyle{python}{
	language =   python, % 语言选Python
	frame=b,
	mathescape,
	emphstyle={\small\ttfamily\bfseries\color{frenchplum}},
	numbers=left,
	stepnumber=1,
	numbersep=3em,
	numberstyle= \small\ttfamily\bfseries\tiny,
	keywordstyle    =  \small\ttfamily\bfseries\color{orange},
	stringstyle     =  \small\ttfamily\bfseries\color{red},
	breaklines      =   true,   % 自动换行,
	columns         =   fixed,  %字间距就不固定很丑,必须加
	basewidth       =   .5em,
	commentstyle=\small\ttfamily\bfseries\color{cyan},
	backgroundcolor=\color[RGB]{250, 250, 250},
	tabsize=4,
	showspaces=false,
	showstringspaces=false,
	morekeywords={},
}
\lstdefinestyle{python2}{
	language =   Python, % 语言选Python
	frame=none,
	basicstyle={\small\ttfamily\bfseries\color{white}},
	mathescape,
	emphstyle={\small\ttfamily\bfseries\color{white}},
	numbers=none,
	stepnumber=2,
	numbersep=3em,
	numberstyle= \small\ttfamily\bfseries\tiny,
	keywordstyle    =  \small\ttfamily\bfseries\color{orange},
	stringstyle     =  \small\ttfamily\bfseries\color{white},
	breaklines      =   true,   % 自动换行,
	columns         =   fixed,  %字间距就不固定很丑,必须加
	basewidth       =   .5em,
	commentstyle=\small\ttfamily\bfseries\color{cyan!30},
	backgroundcolor=\color{mlblue3},
	tabsize=4,
	showspaces=false,
	showstringspaces=false,
	morekeywords={Out,In,maketitle,pip,matplotlib,pandas,numpy,np,dtype},
}
\lstdefinestyle{python3}{
	language =   Python, % 语言选Python
	frame=none,
	basicstyle={\small\ttfamily\bfseries\color{white}},
	mathescape,
	emphstyle={\small\ttfamily\bfseries\color{white}},
	numbers=none,
	stepnumber=2,
	numbersep=3em,
	numberstyle= \small\ttfamily\bfseries\tiny,
	keywordstyle    =  \small\ttfamily\bfseries\color{orange},
	stringstyle     =  \small\ttfamily\bfseries\color{white},
	breaklines      =   true,   % 自动换行,
	columns         =   fixed,  %字间距就不固定很丑,必须加
	basewidth       =   .5em,
	commentstyle=\small\ttfamily\bfseries\color{cyan!30},
	backgroundcolor=\color{macosbox@bgdark},
	tabsize=4,
	showspaces=false,
	showstringspaces=false,
	morekeywords={Out,In,maketitle,pip,matplotlib,pandas,numpy,np,dtype},
}
\lstdefinestyle{python4}{
	language =   Python, % 语言选Python
	frame=none,
	mathescape,
	emphstyle={\small\ttfamily\bfseries\color{frenchplum}},
	numbers=none,
	stepnumber=2,
	numbersep=3em,
	numberstyle= \small\ttfamily\bfseries\tiny,
	keywordstyle    =  \small\ttfamily\bfseries\color{orange},
	stringstyle     =  \small\ttfamily\bfseries\color{magenta},
	breaklines      =   true,   % 自动换行,
	columns         =   fixed,  %字间距就不固定很丑,必须加
	basewidth       =   .5em,
	commentstyle=\small\ttfamily\bfseries\color{cyan},
	backgroundcolor=\color{macosbox@bg},
	tabsize=4,
	showspaces=false,
	showstringspaces=false,
	morekeywords={Out,In,maketitle,pip,matplotlib,pandas,numpy,np,dtype},
}
%%%%%%%%%%%%%%%%%%%%%%%
\newtcolorbox{code}[1][]{%
	enhanced jigsaw,
	breakable,
	borderline west={1.5pt}{0pt}{mblue},
	sharp corners,
	boxrule=0pt,
	fonttitle={\small\ttfamily\bfseries},
	coltitle={black},
	title={{\color{mblue} \faCode} \quad Python Code:\ },
	colbacktitle=mlblue,
	colback=bg,
	left=2mm,
	right=2mm,
	#1}
%%%%%%%%%%%%%%%%%%%%%%%%%%%%%
\tcbset{
	myexample/.style={
		enhanced,
		width=\linewidth,
		colback=white, % 背景颜色 red!5!white
		colframe=gray!20, % 外框的颜色
		fonttitle=\bfseries,
		breakable,
		arc=2pt,
		drop shadow={gray!15,opacity=1},
		titlerule=0pt,
		title style={fill=white},
		coltitle=gray,
		drop shadow,
		highlight math style={reset,colback=white,colframe=black}
	}
}
\newtcolorbox{sBox}{myexample}
%%%%%%%%%%%%%%%%%%%%%%%%%%%%%%%%%%%%%%%%%
\newtcolorbox{notes}[1][]{%
	enhanced jigsaw,
	borderline west={1.5pt}{0pt}{mblue},
	sharp corners,
	boxrule=0pt,
	fonttitle={\small\bfseries},
	coltitle={black},
	title={{\color{mblue} \faInfoCircle} Note:\ },
	colbacktitle=mlblue,
	colback=bg,
	left=2mm,
	right=2mm,
	#1}

%%%%%%%%%%%%%%%%%%%%%%%%%%%
\newtcolorbox{warning}[1][]{%
	enhanced jigsaw,
	breakable,
	borderline west={1.5pt}{0pt}{mred},
	sharp corners,
	boxrule=0pt,
	fonttitle={\small\bfseries},
	coltitle={black},
	title={{\color{mred} \faExclamationTriangle} Warning:\ },
	colbacktitle=mlred,
	colback=bg,
	left=2mm,
	right=2mm,
	#1}
%%%%%%%%%%%%%%%%%%%%%%%%%%%%%
\newenvironment{macosbox}[2][]{
	\begin{tcolorbox}[enhanced,
		coltitle=black,
		colback=macosbox@bg,
		boxrule=0mm,
		frame style={draw=macosbox@bord,fill=macosbox@bord},
		title style={top color=macosbox@top,bottom color=macosbox@bot},
		drop fuzzy shadow=black,
		#1, %hbox
		title=\hspace*{-3mm}%
		\macosbox@dot{macosbox@red} %
		\macosbox@dot{macosbox@yellow} %
		\macosbox@dot{macosbox@green}%
		\hspace*{\fill}\hspace*{-10mm}#2\hspace*{\fill}]
		\ifthenelse{\isundefined{\usemintedstyle}}{}{\usemintedstyle{\macosbox@mintedstyle}}
		\color{macosbox@textcol}
	}{
	\end{tcolorbox}
}
%%%%%%%%%%%%%%%%%%%%%%%%%%%%%%%%%%%%%%%%%%%%%%%%%%%%%%%%%%%%%%
\newtcolorbox{macbox}[2][]{%
	enhanced,
	coltitle=black,
	colback=macosbox@bg,
	boxrule=0mm,
	frame style={draw=macosbox@bord,fill=macosbox@bord},
	title style={top color=macosbox@top,bottom color=macosbox@bot},
	drop fuzzy shadow=black,
	title={{\textcolor[RGB]{236, 96, 92}{\faCircle}
			\textcolor[RGB]{247, 188, 44}{\faCircle}
			\textcolor[RGB]{88, 204, 65}{\faCircle}
			\hspace*{\fill}\hspace*{-10mm}\texttt{#2}\hspace*{\fill}}},#1
}
\newtcolorbox{macboxd}[2][]{%
	enhanced,
	colupper=white,
	coltitle=white,
	colback=macosbox@bgdark,
	boxrule=0mm,
	frame style={draw=macosbox@borddark,fill=macosbox@borddark},
	title style={top color=macosbox@topdark,bottom color=macosbox@botdark},
	drop fuzzy shadow=black,
	title={{\textcolor[RGB]{236, 96, 92}{\faCircle}
			\textcolor[RGB]{247, 188, 44}{\faCircle}
			\textcolor[RGB]{88, 204, 65}{\faCircle}\hspace*{\fill}\hspace*{-10mm}\texttt{#2}\hspace*{\fill}}},
	#1
}
%%%%%%%%%%%%%%%%%%%%%%%%%%%%%
\definecolor{macosbox@red}{RGB}{236,96,92}
\definecolor{macosbox@yellow}{RGB}{247,188,44}
\definecolor{macosbox@green}{RGB}{88,204,65}

\definecolor{macosbox@top}{RGB}{237,237,237}
\definecolor{macosbox@bot}{RGB}{189,189,189}
\definecolor{macosbox@bord}{RGB}{182,176,176}
\definecolor{macosbox@bg}{RGB}{240,240,240}
\definecolor{macosbox@textcol}{RGB}{50,50,50}

\definecolor{macosbox@topdark}{RGB}{96,96,98}
\definecolor{macosbox@botdark}{RGB}{44,45,47}
\definecolor{macosbox@borddark}{RGB}{13,13,15}
\definecolor{macosbox@bgdark}{RGB}{55,56,58}
\definecolor{macosbox@textcoldark}{RGB}{223,224,226}

\lstdefinestyle{text}{
	language =   HTML, % 语言选Python
	frame=topline,
	mathescape,
	emphstyle={\color{frenchplum}},
	numbers=none,
	stepnumber=2,
	numbersep=3em,
	numberstyle=\tiny,
	rulecolor = \color{black},
	keywordstyle    =   \color{blue},
	stringstyle     =   \color{magenta},
	commentstyle    =   \color{red},
	breaklines      =   true,   % 自动换行,
	columns         =   fixed,  %字间距就不固定很丑,必须加
	basewidth       =   .5em,
	keywordstyle=\color{red},
	commentstyle=\color{gray},
	backgroundcolor=\color[RGB]{218, 218, 218},
	tabsize=4,
	showspaces=false,
	showstringspaces=false,
	columns=fixed,
	morekeywords={maketitle,pip,matplotlib,pandas,numpy,np,dtype,In [],Out []},
}



\usepackage{minitoc}
\usepackage{wallpaper}
\definecolor{ocre}{RGB}{146, 218, 243}
\makeatletter
\newcommand{\@mypartnumtocformat}[2]{%
	\setlength\fboxsep{0pt}%
	\noindent\colorbox{orange!60}{\strut\parbox[c][.7cm]{\ecart}{\color{ocre!70}\Large\sffamily\bfseries\centering#1}}\hskip\esp\colorbox{ocre!40}{\strut\parbox[c][.7cm]{\linewidth+0.2cm-\ecart-\esp}{\partname\Large\sffamily\centering 第#1部分 * #2}}}%
%%%%%%%%%%%%%%%%%%%%%%%%%%%%%%%%%%
% unnumbered part in the table of contents
\newcommand{\@myparttocformat}[1]{%
	\setlength\fboxsep{0pt}%
	\noindent\colorbox{ocre!40}{\strut\parbox[c][.7cm]{\linewidth}{\Large\sffamily\centering#1}}}%
%%%%%%%%%%%%%%%%%%%%%%%%%%%%%%%%%%
\newlength\esp
\setlength\esp{4pt}
\newlength\ecart
\setlength\ecart{1.2cm-\esp}
\newcommand{\thepartimage}{}%
\newcommand{\partimage}[1]{\renewcommand{\thepartimage}{#1}}%

%----------------------------------------------------------------------------------------
%	MINI TABLE OF CONTENTS IN PART HEADS
%----------------------------------------------------------------------------------------

% Chapter text styling

\newcommand{\thepartimages}{}%
\newcommand{\partimages}[1]{\ifusepartimage\renewcommand{\thepartimages}{#1}\fi}%
\def\@part[#1]#2{%
	\ifnum \c@secnumdepth >-2\relax
	\refstepcounter{part}%
	\addcontentsline{toc}{part}{\texorpdfstring{\protect\@mypartnumtocformat{\arabic{part}}{#1}}{\partname~\thepart\ ---\ #1}}
	\else
	\addcontentsline{toc}{part}{\texorpdfstring{\protect\@myparttocformat{#1}}{#1}}%
	\fi
	\markboth{}{}%
	{\ThisCenterWallPaper{1.1}{12.png}
		\centering
		\vspace{-6cm}
		\interlinepenalty \@M
		\normalfont
		\ifnum \c@secnumdepth >-2\relax
		\huge\bfseries \partname  \thepart  * 
%		\vskip 20\p@
		\fi
		\huge \bfseries #2\par
		\hbox{%
	\vbox{%
		\vspace{-4cm}
		\hsize=7mm%
		\begin{tabular}{@{}p{7mm}@{}}
			\makebox[7mm]{\scshape\strut\small}\\
			\makebox[7mm]{\cellcolor{black}\Huge\color{white}\bfseries\strut\rule[-4cm]{0pt}{4cm}}%
		\end{tabular}%
		\makebox(0,0){\put(-10,-100){\fbox{\phantom{\rule[-4cm]{7mm}{4cm}}}}}
	}%
	\kern-2pt
	\vbox to 0pt{%
		\tabular[t]{@{}p{1cm}p{\dimexpr\hsize-2.1cm}@{}}\hline
		& \Huge\itshape\rule{0pt}{1.5\ht\strutbox}\endtabular}}%	
}%
	\@endpart}
\def\@spart#1{%
	{\centering
		\interlinepenalty \@M
		\normalfont
		\Huge \bfseries #1\par}%
	\@endpart}
\def\@endpart{\vfil\newpage
	\if@twoside
	\if@openright
	\null
	\thispagestyle{empty}%
	\newpage
	\fi
	\fi
	\if@tempswa
	\twocolumn
	\fi}

%----------------------------------------------------------------------------------------
%	CHAPTER HEADINGS
%----------------------------------------------------------------------------------------

% A switch to conditionally include a picture, implemented by Christian Hupfer
\newif\ifusechapterimage
\usechapterimagetrue
\newcommand{\thechapterimage}{}%
\newcommand{\chapterimage}[1]{\ifusechapterimage\renewcommand{\thechapterimage}{#1}\fi}%
\newcommand{\autodot}{{\color{structurecolor}\faModx}}
\def\@makechapterhead#1{%
	{\parindent \z@ \raggedright \normalfont
		\ifnum \c@secnumdepth >\m@ne
		\if@mainmatter
		\begin{tikzpicture}[remember picture,overlay]
			\node at (current page.north west)
			{\begin{tikzpicture}[remember picture,overlay]
					\node[anchor=north west,inner sep=0pt] at (0,0) {\ifusechapterimage\includegraphics[width=\paperwidth]{\thechapterimage}\fi};
					\draw[anchor=west] (\Gm@lmargin+3.3cm,-9cm) node [line width=2pt,rounded corners=15pt,draw=ocre,fill=white,fill opacity=0.5,inner sep=15pt]{\strut\makebox[22cm]{}};
					\draw[anchor=west] (\Gm@lmargin+3.9cm,-9cm) node {\huge\kaishu\bfseries\color{black} 第 \thechapter 章  \autodot\color{black}~#1\strut};
			\end{tikzpicture}};
		\end{tikzpicture}
		\else
		\begin{tikzpicture}[remember picture,overlay]
			\node at (current page.north west)
			{\begin{tikzpicture}[remember picture,overlay]
					\node[anchor=north west,inner sep=0pt] at (0,0) {\ifusechapterimage\includegraphics[width=\paperwidth]{\thechapterimage}\fi};
					\draw[anchor=west] (\Gm@lmargin+3.3cm,-9cm) node [line width=2pt,rounded corners=15pt,draw=ocre,fill=white,fill opacity=0.5,inner sep=15pt]{\strut\makebox[22cm]{}};
					\draw[anchor=west] (\Gm@lmargin+3.9cm,-9cm) node {\huge\sffamily\bfseries\color{black}#1\strut};
			\end{tikzpicture}};
		\end{tikzpicture}
		\fi\fi\par\vspace*{270\p@}}}

%-------------------------------------------

\def\@makeschapterhead#1{%
	\begin{tikzpicture}[remember picture,overlay]
		\node at (current page.north west)
		{\begin{tikzpicture}[remember picture,overlay]
				\node[anchor=north west,inner sep=0pt] at (0,0) {\ifusechapterimage\includegraphics[width=\paperwidth]{\thechapterimage}\fi};
				\draw[anchor=west] (\Gm@lmargin+3.3cm,-9cm) node [line width=2pt,rounded corners=15pt,draw=ocre,fill=white,fill opacity=0.5,inner sep=15pt]{\strut\makebox[22cm]{}};
				\draw[anchor=west] (\Gm@lmargin+3.9cm,-9cm) node {\huge\sffamily\bfseries\color{black}#1\strut};
		\end{tikzpicture}};
	\end{tikzpicture}
	\par\vspace*{270\p@}}

\pagenumbering{arabic}

\usepackage{titletoc} % Required for manipulating the table of contents

\contentsmargin{0cm} % Removes the default margin
\setcounter{secnumdepth}{5}
% Part text styling (this is mostly taken care of in the PART HEADINGS section of this file)
\titlecontents{part}
[0cm] % Left indentation
{\addvspace{20pt}\bfseries} % Spacing and font options for parts
{}
{}
{}

\titlecontents{chapter}
[16mm] % Left indentation
{\bfseries} % Spacing and font options for parts
{\contentslabel[\heiti 第~\thecontentslabel 章]{16mm}}
{}
%{[0.5pc]{$\cdot$}\contentspage[{\makebox[0pt][r]{\thecontentspage}}]}
{\titlerule*[0.5pc]{~ }\contentspage}
\makeatother

\usepackage{graphicx}
\usepackage{xhfill}
\newcommand{\fancyline}[2][\ding{36}]{\noindent\ #1\xdotfill{1pt}[ocre]#1\xdotfill{1pt}[ocre]#1\
}

%%%%%%%%%%%%%%%%%%%%%%%%%%%%%%%%%%%%%%%%
\usepackage{ifthen}
\usetikzlibrary{shadows}
\usepackage{afterpage}
\newcommand\myemptypage{
	\null
	\thispagestyle{empty}
	\addtocounter{page}{-1}
	\newpage
}
\usepackage{booktabs,colortbl}
\colorlet{tableheadcolor}{gray!25} % Table header colour = 25% gray
\newcommand{\headcol}{\rowcolor{tableheadcolor}} %
\newcommand{\hdrule}[1]{\begin{ascolorbox101}{#1}\end{ascolorbox101}}
\newcommand{\btrule}[1]{\begin{ascolorbox102}{#1}\end{ascolorbox102}}