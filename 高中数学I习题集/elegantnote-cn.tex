%!TEX program = xelatex
\documentclass[cn,hazy,black,10pt,normal]{elegantnote}
\usepackage{hyperref}
\usepackage{amssymb}
\usepackage[version=3]{mhchem}

% font settings
\definecolor{mgreen}{RGB}{0,166,82}
\definecolor{guess}{RGB}{47,79,79}
\newenvironment{guess}{
  \color{guess}}{\newline \color{black}}

% cover settings
\title{高中数学I习题集}

\author{Johnny Tang}
\institute{DEEP Team}

\date{\zhtoday}

% customised commands
\usepackage{ulem}
	\newcommand{\tk}{\uline{\hspace{4em}}}
\DeclareSymbolFont{yh}{OMX}{yhex}{m}{n}
\DeclareMathAccent{\hu}{\mathord}{yh}{"F3}
\newcommand{\xl}[1]{\overrightarrow{#1}}
\newcommand{\nd}[1]{〔#1〕}
\newcommand{\ssb}[1]{\left( #1 \right)}
\newcommand{\sw}[1]{\boxed{\text{解法 #1}} \ }
\newcommand{\buzhou}[1]{$#1^{\circ} \ $}
\newcommand{\R}{\mathbb{R}}
\newcommand{\hlt}[1]{\color{red} #1 \color{black}}
\DeclareMathOperator{\card}{card}

% 行距设置
\setlength{\lineskiplimit}{5pt} %至少宽度
\setlength{\lineskip}{4pt} %正常宽度
\setlength{\normallineskiplimit}{5pt} %正常宽度
\setlength{\normallineskip}{5pt} %正常宽度


\begin{document}

\maketitle

\chapter{集合}

\section{集合及其运算}

\subsection*{填空题}

\begin{problem} % 启东中学奥赛教程 p6
	设集合$M=\{ -1,0,1 \} , ~N=\{ 2,3,4,5,6 \}$,映射$f:M \to N$,则对任意的$x \in M$,使得$x + f(x) +xf(x)$恒为奇数的映射$f$的个数为\tk .
\end{problem}
\begin{hint}
	分类讨论.
\end{hint}

\begin{problem}
	称有限集$S$的所有元素的乘积为$S$的“积数”,给定数集$M= \{ \dfrac{1}{2},\dfrac{1}{3}, \cdots ,\dfrac{1}{100} \}$,则集合$M$的所有含偶数个元素的子集的“积数”之和为\tk .
\end{problem}
\begin{hint}
	举例分析.
\end{hint}

\subsection*{解答题}

\begin{problem} % 启东中学奥赛教程 p4
	(2015高联)设$a_1,a_2,a_3,a_4$是$4$个有理数,使得$\{ a_ia_j | 1 \leq i < j \leq 4 \} = \{ -24,-2,-\dfrac{3}{2},-\dfrac{1}{8},1,3 \}$.求$a_1+a_2+a_3+a_4$的值.
\end{problem}
\begin{hint}
	通过大小关系将$a_1a_2,a_1a_3,a_1a_4,a_2a_3,a_2a_4,a_3a_4$与这六个数字对应.
\end{hint}

\begin{problem} % 启东中学奥赛教程 p4
	(2017清华THUSSAT)已知集合$A= \{ a_1,a_2,a_3,a_4 \}$,且$a_1 < a_2 < a_3 < a_4$,$a_i \in \mathbb{N} ^* ~(i=1,2,3,4)$.记$a_1+a_2+a_3+a_4=S$,集合$B = \{ (a_i,a_j) : (a_i+a_j) | S, a_i,a_j \in A, i<j \}$中的元素个数为$4$个,求$a_1$的值.
\end{problem}
\begin{hint}
	通过大小关系得出不能被$S$整除的两项.
\end{hint}

\begin{problem} % 启东中学奥赛教程 p5
	$X$是非空的正整数集合,满足下列条件:(i)若$x \in X$,则$4x \in X$;(ii)若$x \in X$,则$[\sqrt{x}] \in X$.求证:$X$是全体正整数的集合.
\end{problem}
\begin{hint}
	将两种关于$X$的性质结合起来看.
\end{hint}

\begin{problem} % 启东中学奥赛教程 p5
	设$S$为非空数集,且满足:(i)$2 \notin S$;(ii)若$a \in S$,则$\dfrac{1}{2-a} \in S$.证明: \\
	(1)对一切$n \in \mathbb{N} ^{*} ~, n \geq 3$,有$\dfrac{n}{n-1} \notin S$;(2)$S$或者是单元素集,或者是无限集.
\end{problem}
\begin{hint}
	数学归纳法.
\end{hint}

\begin{problem} % 启东中学奥赛教程 p6
	以某些整数为元素的集合$P$具有下列性质:(i)$P$中的元素有正数,有负数;(ii)$P$中的元素有奇数,有偶数;(iii)$-1 \notin P$;(iv)若$x,y \in P$,则$x+y \in P$.试证明: \\
	(1)$0 \in P$;(2)$2 \notin P$.
\end{problem}
\begin{hint}
	第一问:构造;第二问:反证法.
\end{hint}

\begin{problem} % 启东中学奥赛教程 p6
	已知数集$A$具有以下性质:(i)$0 \in A,1 \in A$;(ii)若$x,y \in A$,则$x-y \in A$;(iii)若$x \in A, ~x \neq 0$,则$\dfrac{1}{x} \in A$. \\
	求证:当$x,y \in A$时,则$xy \in A$.
\end{problem}
\begin{hint}
	只需证明$\dfrac{1}{xy} \in A$,然后构造.
\end{hint}

\newpage
\section{集合元素的个数}

\begin{theorem}{容斥原理1——容斥公式}
	设$A_i$($i=1,2, \cdots ,n$)为有限集,则
	$$|\bigcup_{i=1}^{n} A_i| = \sum_{i=1}^{n} |A_i|-\sum_{1 \leq i < j \leq n}|A_i \cap A_j| + \cdots + (-1)^{n-1} |\bigcap_{i=1}^{n} A_i|$$
	可以使用数学归纳法证明.
\end{theorem}

\begin{theorem}{容斥原理2——筛法公式}
	设$A_i$($i=1,2, \cdots ,n$)为全集$I$的子集,则
	$$|\bigcap_{i=1}^{n} \complement _{I} A_i| = |I| - \sum_{i=1}^{n} |A_i|+\sum_{1 \leq i < j \leq n}|A_i \cap A_j| - \cdots + (-1)^{n} |\bigcap_{i=1}^{n} A_i|$$
	可以通过摩根律证明.这个公式常常用来计算不满足任意给定性质的子集个数.
\end{theorem}

\subsection*{填空题}

\begin{problem} % 启东中学奥赛教程 p8
	设$\{ b_n \}$是集合$\{ 2^t+2^s+2^r | 0 \leq r < s < t, ~r,s,t \in \mathbb{Z} \}$中所有的数从小到大排列成的数列,已知$b_k = 1160$,则$k$的值为\tk .
\end{problem}
\begin{hint}
	分段考虑.
\end{hint}

\begin{problem} % 启东中学奥赛教程 p12
	$A=\{ z|z^{18}=1 \},~ B=\{ w|w^{48}=1 \}$都是$1$的复单位根的集合,$C=\{ zw|z \in A,~ w \in B \}$也是$1$的复单位根的集合.则集合$C$中含有元素的个数为\tk .
\end{problem}
\begin{hint}
	复数的三角表示.
\end{hint}

\begin{problem} % 启东中学奥赛教程 p12
	已知集合$\{ 1,2, \cdots ,3n \}$可以分为$n$个互不相交的三元组$\{ x,y,z \}$,其中$x+y=3z$,则满足上述要求的两个最小的正整数$n$是\tk .
\end{problem}
\begin{hint}
	从条件$x+y=3z$入手变形消元.
\end{hint}

\begin{problem} % 启东中学奥赛教程 p12
	集合$M= \{ x|\cos{x} + \lg \sin{x} = 1 \}$中元素的个数是\tk .
\end{problem}
\begin{hint}
	有没有可能无解?
\end{hint}

\subsection*{解答题}

\begin{problem} % 启东中学奥赛教程 p12
	设集合$M = \{ 1,2, \cdots ,1995 \}$,$A$是$M$的子集且满足条件:当$x \in A$时,$15x \notin A$,求$A$中元素个数的最大值.
\end{problem}
\begin{hint}
	先构造最大值情况,再证明这是最大值.
\end{hint}

\begin{problem} % 启东中学奥赛教程 p12
	求最大的正整数$n$,使得$n$元集合$S$同时满足:(i)$S$中的每个数均为不超过$2002$的正整数;(ii)对于$S$的两个元素$a$和$b$(可以相同),它们的乘积$ab$不属于$S$.
\end{problem}
\begin{hint}
	先构造最大值情况,再证明这是最大值.
\end{hint}

\begin{problem} % 启东中学奥赛教程 p12
	我们称一个正整数的集合$A$是“一致”的,是指:删除$A$中任何一个元素之后,剩余的元素可以分成两个不相交的子集,而且这两个子集的元素之和相等.求最小的正整数$n$($n>1$),使得可以找到一个具有$n$的元素的“一致”集合$A$.
\end{problem}
\begin{hint}
	将$A$中元素分奇偶讨论.
\end{hint}

\begin{problem} % 启东中学奥赛教程 p12
	设$n$是正整数,我们说集合$\{ 1,2, \cdots ,2n \}$的一个排列$(x_1,x_2, \cdots ,x_{2n})$具有性质$P$,是指在$\{ 1,2, \cdots ,2n-1 \}$中至少有一个$i$,使得$|x_i-x_{i+1}|=n$,求证:对于任何$n$,具有性质$P$的排列比不具有性质$P$的排列的个数多.
\end{problem}
\begin{hint}
	只需证明具有性质$P$的排列个数大于全部排列数的一半.利用容斥原理放缩.
\end{hint}

\begin{problem} % 启东中学奥赛教程 p12
	设$S \subsetneqq \mathbb{R}$是一个非空的有限实数集,定义$|S|$为$S$中的元素个数,$$m(S) = \frac{\sum_{x \in S} x}{|S|}$$
	已知$S$的任意两个非空子集的元素的算术平均值都不相同.定义$$\dot{S} = \{ m(A) | A \subseteq S, ~A \neq \varnothing \}$$
	证明:$m(\dot{S}) = m(S)$.
\end{problem}
\begin{hint}
	贡献法.
\end{hint}

\newpage
\section{子集的性质}

\subsection*{填空题}

\begin{problem} % 启东中学奥赛教程 p19
	设$S=\{ (x,y)|x^2-y^2 \text{为奇数},~x,y \in \R \},~ T=\{ (x,y)|\sin ^2 (2\pi x^2) - \sin ^2 (2\pi y^2) = \cos ^2 (2\pi x^2) - \cos ^2 (2\pi y^2),~ x,y \in \R \}$,则$S$与$T$的关系为\tk .
\end{problem}
\begin{hint}
	变形.
\end{hint}

\subsection*{解答题}

\begin{problem} % 启东中学奥赛教程 p14
	设$S$是集合$\{ 1,2,3,\cdots ,50 \}$的非空子集,$S$中任何两个数之和不能被$7$整除.求$\card (S)$的最大值.
\end{problem}
\begin{hint}
	列举.
\end{hint}

\begin{problem} % 启东中学奥赛教程 p14
	已知集合$A = \{ 1,2, \cdots ,10 \}$.求集合$A$的具有下列性质的子集个数:每个子集至少含有$2$个元素,且每个子集中任何两个元素的差的绝对值大于$1$.
\end{problem}
\begin{hint}
	递推思想.
\end{hint}

\begin{problem} % 启东中学奥赛教程 p15
	证明:任何一个有限集的全部子集可以这样地排列顺序,使任意两个相邻的集相差一个元素.
\end{problem}
\begin{hint}
	举例或递推.
\end{hint}

\begin{problem} % 启东中学奥赛教程 p15
	对于整数$n~(n \geq 2)$,如果存在集合$\{ 1,2, \cdots ,n \}$的子集族$A_1,A_2, \cdots ,A_n$满足: \\
	(a)$~i \notin A_i,~i=1,2,\cdots ,n$; \\
	(b)若$i \neq j,~ i,j \in \{ 1,2, \cdots ,n \}$,则$i \in A_i$当且仅当$j \notin A_i$; \\
	(c)$~\forall i,j \in \{ 1,2,\cdots ,n \},~ A_i \cap A_j \neq \varnothing$. \\
	则称$n$是“好数”. 证明:(1)$7$是“好数”;(2)当且仅当$n \geq 7$时,$n$是“好数”.
\end{problem}
\begin{hint}
	举例与构造.
\end{hint}

\begin{problem} % 启东中学奥赛教程 p16
	设$S$是一个有$6$个元素的集合,能有多少种方法选取$S$的两个(不必不相同)子集,使得这两个子集的并是$S$?选取的次序无关紧要,例如,一对子集$\{ a,c \},\{ b,c,d,e,f \}$与一对子集$\{ b,c,d,e,f \},\{ a,c \}$表示同一种取法.
\end{problem}
\begin{hint}
	对$\card{(A \cap B)}$进行讨论.
\end{hint}

\begin{problem} % 启东中学奥赛教程 p16
	(2018山东预赛)设集合$A,B$满足:$A \cup B = \{ 1,2,\cdots ,10 \}, ~A \cap B = \varnothing$.若集合$A$中的元素个数不是$A$中的元素,集合$B$中的元素个数不是$B$中的元素,求满足条件的所有不同的集合$A$的个数.
\end{problem}
\begin{hint}
	对$|A|,|B|$进行讨论.
\end{hint}

\begin{problem} % 启东中学奥赛教程 p17
	设$k,n$为给定的整数,$n>k \geq 2$,对任意$n$元的数集$P$,作$P$的所有$k$元子集的元素和,记这些和组成的集合为$Q$,集合$Q$中元素个数是$C_Q$.求$C_Q$的最大值和最小值.
\end{problem}
\begin{hint}
	数学归纳法.
\end{hint}

\begin{problem} % 启东中学奥赛教程 p17
	设集合$S_n=\{1,2, \cdots ,n\}$. 若$X$是$S_n$的子集,把$X$中所有数的和为$X$的“容量”(规定空集的容量为$0$),若$X$的容量为奇(偶)数,则称$X$为$S_n$的奇(偶)子集. \\
    (1)证明:$S_n$的奇子集与偶子集的个数相等;\\
    (2)证明:当$n>2$时,$S_n$的所有奇子集的容量之和等于所有偶子集的容量之和;\\
    (3)当$n>2$时,求$S_n$的所有奇子集的容量之和.
\end{problem}
\begin{hint}
	贡献法.
\end{hint}
 
\chapter{函数}

\section{函数的极值}

\subsection{恒等变形}

\begin{problem} %奥数教程高一p54
	求函数$f(x)=\sqrt{8x-x^2}-\sqrt{14x-x^2-48}$的最值.
\end{problem}
\begin{solution}
	先求定义域:因为$$x(8-x) \geq 0, \quad (8-x)(x-6) \geq 0$$
	所以$6 \leq x \leq 8$.再看最值:因为$$f(x)=\sqrt{8-x} \cdot (\sqrt{x} - \sqrt{x-6}) = \frac{6\sqrt{8-x}}{\sqrt{x} + \sqrt{x-6}}$$
	这个函数在$[6,8]$上显然单调递减,则$f(x) \in [0,2\sqrt{3}]$,取等条件分别为$x=8,6$.
\end{solution}

\begin{problem} %奥数教程高一p58
	若$x \neq 0$,求$$f(x)= \frac{\sqrt{x^4+x^2+1} - \sqrt{x^4+1}}{x}$$的最大值.
\end{problem}
\begin{solution}
	容易发现,$f(x)$是一个奇函数,故最大值只需考虑$x>0$的情况. \\
	因为$$f(x) = \sqrt{x^2+\frac{1}{x^2} + 1} - \sqrt{x^2 + \frac{1}{x^2}}$$
	令$t=x + \dfrac{1}{x} \geq 2$,故$x^2 + \dfrac{1}{x^2} = t^2 - 2 \geq 2$.为了将不等号反向,作变换$$f(x) = \frac{1}{\sqrt{x^2+\dfrac{1}{x^2} + 1} + \sqrt{x^2 + \dfrac{1}{x^2}}} \leq \frac{1}{\sqrt{2+1} + \sqrt{2}} = \sqrt{3} - \sqrt{2}$$
	取等条件为$x=1$.
\end{solution}

\subsection{先猜后凑与配方估计}

\begin{problem} %奥数教程高一p50
	设$x \in \R _{+}$,求$y=x^2+x+\dfrac{3}{x}$的最小值.
\end{problem}
\begin{solution}
	先估计函数值的下界:因为$$y = (x^2 - 2x + 1) + \ssb{3x + \frac{3}{x} - 2} + 5$$
	所以$$y = (x-1)^2 + 3\ssb{\sqrt{x} - \frac{1}{\sqrt{x}}}^2 + 5 \geq 5$$
	因为在$x=1$时这个下界可以取到,可知$y \geq 5$.
\end{solution}
\begin{remark}
	在对$y$进行配凑时,需要让最后的常数项尽可能大.
\end{remark}

\begin{problem} %奥数教程高一p57
	已知$x,y \in \R$,则$f(x,y) = x^2+xy+y^2-x-y$的最小值为\tk .
\end{problem}
\begin{solution}
	\begin{guess}
		令$x=y$,则$f(x,y) = 3x^2-2x$,在$x=\dfrac{1}{3}$时取最小值.
	\end{guess}
	作如下变形:$$f(x,y) = \frac{3}{2}\ssb{x-\frac{1}{3}}^2 + \frac{3}{2}\ssb{y-\frac{1}{3}}^2 - \frac{1}{2}(x-y)^2 - \frac{1}{3} \geq -\frac{1}{3}$$
	取等条件为$x=y=\dfrac{1}{3}$.
\end{solution}

\begin{problem} %奥数教程高一p57
	设$x,y$是实数,且$x^2-2xy+y^2-\sqrt{2}x-\sqrt{2}y+6=0$,求$u=x+y$的最小值.
\end{problem}
\begin{solution}
	\begin{guess}
		令$x=y$,则$x=3\sqrt{2}$.由于取等条件是$x=y$,尝试配凑$(x-y)^2$.
	\end{guess}
	因为$$(x-y)^2 = \sqrt{2}u-6 \geq 0$$
	可得$u \geq 3\sqrt{2}$.取等条件为$x=y=\dfrac{3\sqrt{2}}{2}$.
\end{solution}

\begin{problem} %奥数教程高一p57
	已知$x,y$是正实数,求$$f(x,y) = \frac{x^4}{y^4} + \frac{y^4}{x^4} - \frac{x^2}{y^2} - \frac{y^2}{x^2} + \frac{x}{y} + \frac{y}{x}$$的最小值.
\end{problem}
\begin{solution}
	\begin{guess}
		令$x=y$,可知$f(x,y) \geq 2$.
	\end{guess}
	作如下变形:
	\begin{align*}
		f(x,y) &= \ssb{\frac{x^4}{y^4} - 2\frac{x^2}{y^2} +1} + \ssb{\frac{y^4}{x^4} - 2\frac{y^2}{x^2} + 1} + \ssb{\frac{x^2}{y^2} + \frac{y^2}{x^2} - 2} + \ssb{\frac{x}{y} + \frac{y}{x} - 2} + 2 \\
		&= \ssb{\frac{x^2}{y^2} - 1}^2 + \ssb{\frac{y^2}{x^2} - 1}^2 + \ssb{\frac{x}{y} - \frac{y}{x}}^2 + \ssb{\frac{\sqrt{x}}{\sqrt{y}} - \frac{\sqrt{y}}{\sqrt{x}}}^2 + 2 \\
		&\geq 2
	\end{align*}
	又因为当$x=y=1$时可以取等,于是$f(x,y)$的最小值为$2$.
\end{solution}

\begin{problem} %奥数教程高一p58
	设$x,y$是实数,求$u = x^2+xy+y^2-x-2y+3$的最小值.
\end{problem}
\begin{solution}
	以$x$为主元,因为
	\begin{align*}
		u &= x^2 + (y-1)x + y^2-2y+3 \\
		&= \ssb{x+\frac{y-1}{2}}^2 + \frac{3}{4}(y-1)^2 + 2
	\end{align*}
	于是$u \geq 2$,取等条件为$x=0,y=1$.
\end{solution}

\subsection{利用函数的性质}

\begin{problem} %奥数教程高一p57
	设函数$$f(x) = \frac{(x+\sqrt{2013})^2 + \sin 2013x}{x^2 + 2013}$$
	已知其最大值为$M$,最小值为$m$,则$M+m=$\tk .
\end{problem}
\begin{solution}
	可知$$f(x) = 1 + \frac{2\sqrt{2013}x+\sin 2013x}{x^2+2013}$$
	容易发现,$f(x)$的后一部分是一个奇函数,故$f(x)$关于$(1,0)$对称,即$M+m=2$.
\end{solution}

\begin{problem} %奥数教程高一p58
	求函数$f(x) = \sqrt{2x^2-3x+4} + \sqrt{x^2-2x}$的最小值.
\end{problem}
\begin{solution}
	先看定义域:因为$$2x^2-3x+4 \geq 0, \quad x^2-2x \geq 0$$
	可知$x \in (-\infty ,0] \cup [2,+\infty )$.由于$f(x)$的两个部分均在$(-\infty ,0]$上单调递减、在$[2,+\infty )$上单调递增,可知$f(x)$也在$(-\infty ,0]$上单调递减、在$[2,+\infty )$上单调递增.于是$$f(x) \geq \min \{ f(0),f(2) \} = f(0) = 2$$
\end{solution}

\subsection{综合练习}

\begin{problem} %奥数教程高一p58
	求函数$f(x)=\sqrt{x^4-3x^2-6x+13} - \sqrt{x^4 - x^2 + 1}$的最大值.
\end{problem}
\begin{solution}
	\begin{guess}
		看到两个如此不规整的根号相加,想到数形结合,然而根号下是四次多项式,不方便直接进行构造.不过,注意到$x^4-x^2+1 = (x^2-1)^2 + x^2$,可以从点$(x,x^2)$入手.
	\end{guess}
	因为$$f(x) = \sqrt{(x^2-2)^2 + (x-3)^2} - \sqrt{(x^2-1)^2 + x^2}$$
	这表示抛物线$y=x^2$上的一点$P(x,x^2)$到定点$A(3,2)$与$B(0,1)$的距离之差.因为$$|PA| - |PB| \leq |AB| = \sqrt{10}$$
	于是$f(x) \leq \sqrt{10}$,取等条件为$A,B,P$三点共线,即$x=\dfrac{1-\sqrt{37}}{6}$时.
\end{solution}

\section{二次函数相关问题}

\subsection{参变互换}

\begin{problem} % 启东中学奥赛教程p45
	设二次函数$f(x)=ax^2+bx+c~(a>0)$,方程$f(x)-x=0$的两个根$x_1,x_2$满足$0<x_1<x_2< \dfrac{1}{a}$. \\
	(1)当$x \in (0,x_1)$时,证明:$f(x)<x_1$; \\
	(2)设函数$f(x)$的图像关于直线$x=x_0$对称,证明:$x_0 < \dfrac{1}{2}x_1$.
\end{problem}
\begin{solution}
	(1)因为$f(x)-x=ax^2+(b-1)x+c=0$的两根$x_1,x_2$满足$$x_1+x_2=\frac{1-b}{a},\quad x_1x_2=\frac{c}{a}$$
	所以
	\begin{align*}
		f(x)-x_1 &= ax^2 + [1-a(x_1+x_2)]x+ax_1x_2-x_1 \\
		&= a[x^2-(x_1+x_2)x+x_1x_2]+x-x_1 \\
		&= (ax-ax_2+1)(x-x_1)
	\end{align*}
	其中,$x-x_1<0$,$ax-ax_2+1>0$,则$f(x)-x_1<0$,即$f(x)<x_1$. \\
	(2)由(1),此时$x_0 = \dfrac{b}{-2a} = \dfrac{x_1+x_2}{2}-\dfrac{1}{2a}$.由$x_2 < \dfrac{1}{a}$,显然有$x_0<\dfrac{1}{2}x_1$.
\end{solution}

\begin{problem} %奥数教程高一p190
	已知函数$f(x)=ax^2+bx+c$在$[0,1]$上的函数值的绝对值不超过$1$,求$|a|+|b|+|c|$的最大值.
\end{problem}
\begin{solution}
	因为$f(0),f\ssb{\dfrac{1}{2}},f(1) \in [-1,1]$,且$$a = 2f(1)+2f(0)-4f\ssb{\frac{1}{2}},\quad b = 4f\ssb{\frac{1}{2}}-3f(0)-f(1),\quad c=f(0)$$
	所以$$S_0 = |2f(1)+2f(0)-4f\ssb{\frac{1}{2}}| + |4f\ssb{\frac{1}{2}}-3f(0)-f(1)| + |f(0)|$$
	且这个式子在$f(0),f(1)$增大时增大,在$f\ssb{\dfrac{1}{2}}$增大时减小.故$$S_0 \leq 8+8+1=17$$
	取等条件为$f(1)=1,f(0)=1,f\ssb{\dfrac{1}{2}}=-1$,即$f(x)=8x^2-8x+1$.
\end{solution}

\subsection{调整}



\begin{problem} %奥数教程高一p193
	已知函数$f(x)=ax^2+bx+c~(a>b>c)$的图像上有两点$A(m_1,f(m_1)),~B(m_2,f(m_2))$,且满足$$f(1)=0,\quad a^2+[f(m_1)+f(m_2)]a+f(m_1)f(m_2)=0$$
	(1)求证:$b\geq 0$; \\
	(2)求证:$f(x)$的图像被$x$轴截得线段长的取值范围是$[2,3)$; \\
	(3)问能否得出$f(m_1+3),f(m_2+3)$中至少有一个为正?
\end{problem}
\begin{solution}
	(1)由第二个式子,$a=-f(m_1)~\textit{或}-f(m_2)$,也即$f(x)=-a$的两根为$m_1,m_2$.这告诉我们方程$$ax^2+bx+a+c=0$$有两个实数解,即$\Delta = b^2-4a(a+c) \geq 0$. \\
	由$f(1)=0$,有$b=-a-c$,故$(a+c)^2-4a(a+c) \geq 0$,即$(3a-c)(a+c) \leq 0$.又因为$a+b+c=0$中必有一正一负,且$a>b>c$,故$a>0>c$,于是必有$a+c \leq 0$,即$b \geq 0$. \\
	(2)由于$f(1)=0$,又$x_1x_2=\dfrac{c}{a}$,可知$f(x)$的零点为$1,\dfrac{c}{a}$.下证$1-\dfrac{c}{a} \in [2,3)$:\\
	因为$a>b=-a-c$,可知$\dfrac{c}{a} > -2$;因为$b=-a-c \geq 0$,可知$\dfrac{c}{a} \leq -1$.带入上式,即$1-\dfrac{c}{a} \in [2,3)$. \\
	(3)不妨设$a=-f(m_1)$,即$f(m_1) = a(m_1-1)(m_1-\dfrac{c}{a}) = -a$,可得$(m_1-1)(m_1-\dfrac{c}{a}) = -1 <0$,所以$\dfrac{c}{a} < m_1 < 1$,于是$\dfrac{c}{a}+3<m_1+3<4$,即$m_1+3>2$.又因为$f(x)$在$\ssb{\dfrac{1}{2}\ssb{\dfrac{c}{a}+1},+\infty}$上单调递增,即$f(x)$一定在$(0,+\infty)$上单调递增,于是$$f(m_1+3) > f(1) = 0$$
	同理,对于$a=-f(m_2)$的情况,可以得到$f(m_2+3)>0$.原题即证毕.
\end{solution}

\chapter{三角函数}

\chapter{平面向量}

\subsection*{填空题}

\begin{problem} %23春jx一试强化第一讲
	在边长为$8$的正方形$ABCD$中,$M$是$BC$的中点,$N$是$AD$边上的一点,且$DN=3NA$,若对于常数$m$,在正方形$ABCD$的边上恰有$6$个不同的点$P$,使$\xl{PM} \cdot \xl{PN} = m$,则实数$m$的取值范围为\tk .
\end{problem}

\begin{problem} %23春jx一试强化第一讲
	已知点$P,Q$在$\vartriangle ABC$内,且$\xl{PA}+2\xl{PB}+3\xl{PC}=2\xl{QA}+3\xl{QB}+5\xl{QC}=\xl{0}$,则$\dfrac{|\xl{PQ}|}{|\xl{AB}|}$的值为\tk .
\end{problem}

\begin{problem} %23春jx一试强化第一讲
	已知向量$\xl{OA} \bot \xl{OB}$,且$|\xl{OA}|=|\xl{OB}|=24$.若$t \in [0,1]$,则$|t\xl{AB}-\xl{AO}| + |\dfrac{5}{12} \xl{BO} - (1-t)\xl{BA}|$的最小值为\tk .
\end{problem}



\chapter{复数}

\chapter{数列}

\chapter{极限与导数}

\chapter{不等式}

\chapter{概率统计与计数}

\chapter{解析几何}

\chapter{立体几何}





\end{document}
