\chapter{本征值和本征向量}

\subsection*{引子一}

设$T \in \lmap (V)$, 设$V$的一个直和分解$V=V_1 \oplus \cdots \oplus V_m$, 则在研究$T$时我们只需要知道每个$T|_{V_k}$的行为即可. 但是, $T|_{V_k}$不一定是$V_k$上的算子, 这样很多有效的工具就没有作用了. 因此, 我们要考虑是否存在$V$的一种直和分解, 使得对所有$k$, $T|_{V_k} \in \lmap (V_k)$. 

一般地, 设$U$是$V$的子空间, $T\in \lmap (V)$. 称$U$是$T$下的一个\textit{不变子空间}, 如果$T(U) \subseteq U$. 显然不变子空间的交与和都是不变子空间, 因此可以考虑最简单的不变子空间, 即由向量$v$张成的一维子空间. 此时$Tv \in \spn (v)$说明存在$\lambda \in \F$使得$Tv=\lambda v$. 反过来, 若上式成立, 则$\spn (v)$自然是$T$的一个不变子空间. 

从矩阵的角度来看, 若想要让$V$分解为若干个一维不变子空间的直和, 就是要寻找一组基使得$T$的表示矩阵为对角矩阵. 

\subsection*{引子二\footnote{摘编自~梁鑫,田垠,杨一龙. \underline{线性代数入门}. 清华大学出版社, 2022}}

我们来看这样一个例子: 甲地和乙地之间每年都会有人口的流动. 设每年甲地向乙地转移其人口的$40 \%$, 乙地向甲地转移其人口的$10 \%$, 那么当足够久之后两地的人口比例是否会趋于稳定? 用矩阵变换来描述这个问题, 就是说对任意的$0<t<1$, 当$n\to \infty$时式子$$
\begin{pmatrix}
 0.6 & 0.1\\
 0.4 & 0.9
\end{pmatrix}^n \cdot \begin{pmatrix}
t \\
1-t
\end{pmatrix}$$
是否存在极限? 

一方面, 由压缩映射原理, 我们可以猜测稳定状态下的结果$x$就是$Ax=x$的解, 而后者解集为向量$\begin{pmatrix}
0.2 \\ 0.8
\end{pmatrix}$的张成空间. 另外不难验证对任意$n$, 最终$x$的两个坐标之和为$1$. 因此我们猜出了稳定的情况$x_0=\begin{pmatrix}
0.2 \\ 0.8
\end{pmatrix}$. 

另一方面, 设$A$是上方的矩阵, 我们要考虑怎样选取一组基才能将$A$变为对角矩阵(因为对角矩阵的乘积就是直接将对角线上元素相乘). 这就是说, 要寻找$x_1,x_2$使得$Ax_1=\lambda _1x_1$和$Ax_2=\lambda _2x_2$. 一般地, 考虑方程$(A-\lambda I)x=0$. 我们知道该方程有非零解当且仅当$\rank A=1$, 从而$$\rank A=1 ~~\Leftrightarrow ~~ \frac{0.6-\lambda}{0.4} = \frac{0.1}{0.9-\lambda} ~~\Leftrightarrow ~~ \lambda ^2-1.5\lambda +0.5=0 ~~\Leftrightarrow ~~ \lambda _1=1, \lambda _2=0.5. $$
计算可得, $\lambda _1,\lambda _2$对应的$x_1,x_2$分别为$\begin{pmatrix}
0.2 \\ 0.8
\end{pmatrix},\begin{pmatrix}
1 \\ -1
\end{pmatrix}$. 记$X=\begin{pmatrix}
	x_1 & x_2
\end{pmatrix}$. 那么$$A^n X = \begin{pmatrix}
	A^nx_1 & A^nx_2
\end{pmatrix} = \begin{pmatrix}
	x_1 & 0.5^nx_2
\end{pmatrix} = X\begin{pmatrix}
	1 & \\ & 0.5
\end{pmatrix}^n. $$
由$x_1,x_2$线性无关知$X$可逆, 那么$$A^n = \begin{pmatrix}
	0.2 & 1 \\ 0.8 & -1
\end{pmatrix} \cdot \begin{pmatrix}
	1 & \\ & 0.5^n
\end{pmatrix} \cdot \begin{pmatrix}
	0.2 & 1 \\ 0.8 & -1
\end{pmatrix}^{-1} = \begin{pmatrix}
	0.2+\frac{0.8}{2^n} & 0.2-\frac{0.2}{2^n} \\ 0.8-\frac{0.8}{2^n} & 0.8+\frac{0.2}{2^n}
\end{pmatrix}. $$
由数分的知识, 一个矩阵的极限就等于其所有分量极限的矩阵. 因此$A^n \to \begin{pmatrix}
	0.2 & 0.2 \\ 0.8 & 0.8
\end{pmatrix}$, 从而$A^n \begin{pmatrix}
t \\
1-t
\end{pmatrix} \to \begin{pmatrix}
0.2 \\ 0.8
\end{pmatrix}$. 这与我们的猜想是一致的. 

\section{不变子空间}

总结以上两个引子, 我们希望找到一组基使得$T \in \lmap (V)$的表示矩阵为对角矩阵, 这直接等价于寻找$\lambda$和$v$使得$Tv=\lambda v$. 于是引出如下定义: 

\begin{definition}{本征值, 本征向量}
	设$T \in \lmap (V)$. 称$\lambda \in \F$为$T$的一个\textit{本征值}, 如果存在$V \ni v \neq 0$使得$Tv = \lambda v$, 同时这样的$v$称作$\lambda $的一个\textit{本征向量}. 
\end{definition}

在$\R ^n$平面上, $Tv=\lambda v$的几何意义就是$Tv$和$v$共线. 

\begin{example}
	设旋转变换$R = \begin{pmatrix}
 \cos \theta & -\sin \theta \\
 \sin \theta & \cos \theta
\end{pmatrix}$. 若$\theta \neq k\pi$, 在$\R ^2$上显然不存在$v$使得$Rv$和$v$共线. 从矩阵的角度, $(R-\lambda I)v=0$有解等价于$$\frac{\cos \theta - \lambda}{\sin \theta} = \frac{-\sin \theta}{\cos \theta - \lambda} ~~\Leftrightarrow ~~\lambda ^2 - 2\cos \theta \lambda + 1 = 0~~\Leftrightarrow ~~\Delta = 4(\cos ^2 \theta - 1) \geq 0. $$
而这是不可能的. 但是另一方面, 若考虑$\C ^2$, 则方程有两解$\lambda _{1,2} = \cos \theta \pm \ic \sin \theta$. 
\end{example}

上面的例子说明, 在$\R ^n$和$\C ^n$上我们可能得到完全不同的结果. 

显然, 若$v$是$\lambda$的本征向量, 则$kv,k \in \F$也是. 因此不难得到下方结论: 

\begin{proposition}{}
	设$T \in \lmap (V)$, $\lambda _1, \cdots ,\lambda _m$是$T$的本征值, $v_j$分别是$\lambda _j$对应的本征向量$(j=1,\cdots ,m)$, 则$v_1, \cdots ,v_m$线性无关. 
\end{proposition}
\begin{proof}
	设若不然, 取$v_1,\cdots ,v_m$的一个极大线性无关组, 不妨设为$v_1,\cdots ,v_k$. 记$v_m = a_1v_1 + \cdots + a_kv_k$, 进而$$Tv_m = a_1Tv_1 + \cdots + a_kTv_k ~~\Leftrightarrow ~~\lambda _mv_m = a_1\lambda _1v_1 + \cdots + a_k\lambda _kv_k.$$
	消元可得$a_1(\lambda _1-\lambda _m)v_1 + \cdots + a_k(\lambda _k-\lambda _m)v_k = 0$, 显然矛盾. 
\end{proof}

该命题的一个直接推论是: 不同本征值的个数不能超过向量空间的维数. 

现在我们来看一些平凡的子空间. 设$T \in \lmap (V)$, 则$$\nul T,\qquad \rge T,\qquad \{ 0 \},\qquad V,$$
都是$T$下的不变子空间. 一般地, 设多项式$p(z)=a_0+a_1z+\cdots +a_nz^n$, 若定义关于算子$T \in \lmap (V)$的多项式$p(T):=a_0+a_1T+\cdots +a_nT^n$, 则有: 

\begin{proposition}{}
	设$T \in \lmap (V)$, $P \in \mathcal{P}(\F)$, 则$\nul p(T)$和$\rge p(T)$是$T$下的不变子空间. 
\end{proposition}
\begin{proof}
	(1) 设$p(T)u=0$, 则$p(T)(Tu)=T(p(T)u)=0$, 即是说$Tu \in \nul p(T)$. 
	
	(2) 设存在$v \in V$使得$u=p(T)v$, 则$Tu=T(p(T)v)=p(T)(Tv)$, 即是说$Tu \in \rge p(T)$. 
\end{proof}



\newpage
\section{极小多项式}

\subsection{本征值与极小多项式}

正如前文所述, 在$\R$和$\C$上的向量空间关于本征值的存在性可能不一样. 下文全部讨论$\C$上的向量空间. 

\begin{proposition}{复向量空间上的算子总存在本征值} \label{pro:bfvgvidecyzdxk}
	设$V$是$\C$上的有限维向量空间, 则任意的$T \in \lmap (V)$都存在本征值. 
\end{proposition}
\begin{proof}
	设$v \neq 0$, $n=\dim V$, 则$v,Tv,\cdots ,T^nv$线性相关, 从而存在非零多项式$p$使得$p(T)v=0$, 特别地我们要求$p$的次数是使得该式成立最小的(这句话是先验的). 由代数基本定理, 存在$\lambda \in \C$使得$p(\lambda)=0$, 进而存在多项式$q$使得$p(T)=(T-\lambda I)q(T)$, 这说明$(T-\lambda I)(q(T)v)=0$. 由$p$次数的最小性, $q(T)v \neq 0$, 于是$\lambda$是$T$的一个本征值, $q(T)v$是$\lambda$的本征向量. 
\end{proof}

去掉有限维的假设是不可行的, 反例如: 

\begin{example}
	设$T \in \lmap (\mathcal{P}(\C))$使得$(Tp)(z)=zp(z)$. 容易发现对于非零多项式$p$, $Tp$的次数会严格大于$p$的次数, 因此$Tp$不能表示为$p$的标量倍. 
\end{example}

在上方命题的证明中, 我们发现使得$p(T)v=0$的多项式$p$起到了关键作用. 实际上, 我们可将其扩展为更强的命题: 

\begin{proposition}{极小多项式的存在性和唯一性}
	设$V$是有限维向量空间, $T \in \lmap (V)$, 则存在首一多项式$p \in \mathcal{P}(\F)$使得$p(T)=0$. 特别地, 称满足该式的次数最小的多项式为\textit{极小多项式}, 那么极小多项式唯一且次数$\deg p \leq \dim V$. 
\end{proposition}
\begin{proof}
	(1) 用归纳法. 当$\dim V=0$时取$p(z)=1$即可; $\dim V=1$时, 任取$v \neq 0$, 由$v,Tv$线性相关可知存在首一多项式$p \in \mathcal{P}_1(\F)$使得$p(T)v=0$. 记$p(z)=z+a_0$, 那么$Tv+a_0 v=0$. 另一方面, 熟知存在$\lambda$使得$Tv=\lambda v$对所有$v \in V$成立, 因此$\lambda = -a_0$, 即是说$p(T)v=0$对所有$v \in V$成立. 
	
	设$\dim V=n+1$, 假设当$\dim V \leq n$时命题均成立. 任取$v \neq 0$, 由$v,Tv,\cdots ,T^{n+1}v$线性相关可知存在最小的$m$使得$T^mv$能被$v,Tv,\cdots ,T^{m-1}v$线性表出(也就是说$v,Tv,\cdots ,T^{m-1}v$是极大线性无关组). 由此构造首一$m$次多项式$p$使得$p(T)v=0$. 
	
	我们首先注意到对任意$k \geq 0$, $p(T)(T^kv) = T^k(p(T)v)=0$, 所以$\dim \nul p(T) \geq m$. 另外, 由之前的命题, $\rge p(T)$是$T$下的不变子空间, 故$T|_{\rge p(T)} \in \lmap (\rge p(T))$, 进而可以应用归纳假设, 也就是说存在首一多项式$q$使得$\deg q (\leq \dim \rge p(T) \leq n+1-m)$是满足$q(T|_{\rge p(T)}) = 0$的最小数. 
	
	最后只需验证, 对任意$v \in V$, $qp(T)(v) = q(T)(p(T)v) = 0$且$\deg qp = \deg q + \deg p \leq n+1$, 立得$qp$是满足要求的极小多项式. 
	
	(2) 假设存在不同的$n$次首一多项式$p,q$使得$p(T)=0,q(T)=0$且$n$是满足该式的最小正整数. 设$p-q$的首项系数为$\lambda \neq 0$, 那么$\deg (\frac{p-q}{\lambda})<n$和$(\frac{p-q}{\lambda})(T)=0$, 与$n$的最小性矛盾. 于是$p-q=0$. 
\end{proof}

在实际计算极小多项式时, 一般采用如下策略: 取$v \in V$, 考虑方程$c_0v + c_1Tv + \cdots + c_{n-1}T^{n-1}v + T^nv = 0$, 其中$n=\dim V$. 如果该方程存在唯一解$c_0,\cdots ,c_n$, 则极小多项式的系数就是这组解. 这对于表示矩阵中$0$的个数较多的线性映射是很方便的. 

\begin{example}
	设$T \in \lmap (\F ^5)$, 关于标准基的矩阵是$\mathcal{M}(T) =  \begin{pmatrix}
 0 & 0 & 0 & 0 & -3 \\
 1 & 0 & 0 & 0 & 6 \\
 0 & 1 & 0 & 0 & 0 \\
 0 & 0 & 1 & 0 & 0 \\
 0 & 0 & 0 & 1 & 0
\end{pmatrix}$. 求$T$的极小多项式. 
\end{example}
\begin{solution}
	考虑方程$c_0e_1+c_1Te_1 + \cdots + T^5e_1 = 0$, 注意到$$Te_1=e_2,\quad T^2e_1=Te_2=e_3,\quad T^3e_1=Te_3=e_4,\quad T^4e_1=Te_4=e_5,\quad T^5e_1=Te_5=-3e_1+6e_2, $$
	因此方程等价于$(c_0-3)e_1+(c_1+6)e_2+c_2e_3+c_3e_4+c_4e_5=0$. 由$e_1,\cdots ,e_5$线性无关可知方程的唯一解是$c_0=3,c_1=-6,c_2=c_3=c_4=0$. 因此$T$的极小多项式就是$p(z)=3-6z+z^5$. 
\end{solution}

根据命题\ref{pro:bfvgvidecyzdxk}的证明过程可以得到: 

\begin{proposition}{}
	设有限维复向量空间$V$上的算子$T \in \lmap (V)$, 设$T$的本征值为$\lambda _1,\cdots ,\lambda _m$, 则$T$的极小多项式是$p(z)=(z-\lambda _1) \cdots (z-\lambda _m)$. 
\end{proposition}
\begin{remark}
	通过合并一些因式, 可以将该命题推广到实数的情况. 
\end{remark}
\begin{remark}
	从这个命题可以看出, 一些矩阵的特征值是不可求(解析)解的, 例如上个例子的五次方程. 
\end{remark}

\begin{proposition}{}
	设有限维向量空间$V$上的算子$T \in \lmap (V)$, 则多项式$q$满足$q(T)=0$当且仅当它能被$T$的极小多项式乘除. 
\end{proposition}
\begin{proof}
	充分性是显然的. 必要性: 设$T$的极小多项式$p$, 令$q=ps+r$, 其中$\deg r < \deg p$, 那么$0=q(T)=p(T)s(T)+r(T)=r(T)$. 若$r \neq 0$, 不妨让$r$的首项系数为$1$, 即得到矛盾. 因此$p \mid q$. 
\end{proof}

\begin{corollary}{}
	设有限维向量空间$V$上的算子$T \in \lmap (V)$, $U \subseteq V$是$T$下的不变子空间. 那么$T$的极小多项式被$T|_U$的极小多项式整除. 
\end{corollary}

\begin{proposition}{}
	设有限维向量空间$V$上的算子$T \in \lmap (V)$, 则$T$不可逆当且仅当$T$的极小多项式常数项为$0$. 
\end{proposition}

\subsection{奇数维实向量空间上的本征值}

\begin{lemma}{}
	设有限维实向量空间$V$上的算子$T \in \lmap (V)$, $b,c \in \R$满足$b^2<4c$. 则$\dim \nul (T^2+bT+cI)$是偶数. 
\end{lemma}
\begin{proof}
	由于$\nul (T^2+bT+cI)$是$T$下的不变子空间, 不妨令$V=\nul (T^2+bT+cI)$. 设$T^2+bT+cI=0$, 假设$\lambda$是$T$的本征值, $v \neq 0$是$\lambda$的本征向量, 那么$$0 = (T^2+bT+cI)v=(\lambda ^2+b\lambda +c)v.$$
	从而$\lambda ^2+b\lambda +c=0$, 矛盾. 因此$T$不存在本征值. 
	
	设$U$是最大的$V$的偶数维数子空间, 使得$U$在$T$下不变. 不妨考虑$U \neq V$, 即存在$w \in V-U$. 取$W=\spn (w,Tw)$, 那么$T(Tw)=-bTw-cw \in W$, 即$W$在$T$下不变. 另外显然$\dim W = 2$. 注意到$U \cap W=\{ 0 \}$, 否则$U \cap W = \spn (Tw)$是$T$的不变子空间, 从而$T$存在本征值. 于是$U+W$是$T$下的偶数维不变子空间, 与$U$的最大性矛盾. 由此得到$U=V$. 
\end{proof}

下一个命题强化了命题\ref{pro:bfvgvidecyzdxk}的结果: 

\begin{proposition}{奇数维实向量空间上的算子总存在本征值}
	设$V$是$\R$上的奇数维向量空间, 则任意的$T \in \lmap (V)$都存在本征值. 
\end{proposition}
\begin{proof}
	对$V$的维数归纳证明. 当$\dim V=1$时显然成立. 假设$\dim V\leq n$时成立, 考虑$\dim V=n+2$: 设$p$是$T$的极小多项式, 不妨设$p$没有一次因式, 即存在$b,c \in \R$满足$b^2<4c$且$z^2+bz+c \mid p(z)$, 那么存在首一多项式$q$使得$$p(z)=q(z)(z^2+bz+c),\quad \Rightarrow \quad 0=p(T)=q(T)(T^2+bT+cI).$$
	这就是说, $q(T)$在$\rge (T^2+bT+cI)$上为$0$, 特别地有$\rge (T^2+bT+cI) \neq V$. 
	
	另一方面, 由引理可知$\dim \nul (T^2+bT+cI)$是偶数, 于是$\dim \rge T^2+bT+cI$是奇数, 应用归纳假设可得$T$在$\rge (T^2+bT+cI)$上存在本征值, 进一步$T$在$V$上有本征值. 
\end{proof}


























