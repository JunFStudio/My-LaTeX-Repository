\chapter{收敛性和连续性的一般化}

在第四章研究函数的连续性时, 我们已经“顺带”定义好了度量空间, 甚至定义了($\R$上)开集, 闭集, 连续映射等拓扑概念. 但是, 有一些微妙的事情我们还没有处理. 例如, 我们说闭区间上连续函数的性质可以推广到所谓“连通的”或者“紧的”集合上; 另外, 我们好奇拓扑空间的具体性质. 这些事情统统可以归在点集拓扑学里, 也就是本章会(部分地)研究的对象. 

在数分I就讲拓扑学是少见的安排, 事实上本章的内容应该对应Zorich的第七章和第九章. 这样做的原因主要在于, 连续函数(映射)自然会牵涉到很多拓扑的内容, 紧接着第四章讲拓扑学有助于贯通理解这两个概念的关系(而且拓扑和后面的内容没有必要的先后关系). 

本章的核心内容如下表所示. 

\begin{table}[h]
	\small 
	\centering
	\renewcommand\arraystretch{1.4}
	\begin{tabular}{|l|l|p{2cm}|l|p{2cm}|}
	\hline
\multicolumn{1}{|l|}{{ 分类}}       & { 性质}                    & { $\R ^n$或有限维赋范向量空间} & { (一般的)度量空间} & (一般的)拓扑空间        \\ \hline
{ }                             & { 邻域}                    & { 与度量相关}                                          & { 与度量相关}     & 与度量无关            \\ 
\multirow{-2}{*}{{ 度量相关}}       & { 有界性}                   & { 有}                                              & { 有}         & 无                \\ \hline
{ }                             & { 点列极限的性质}               & { 极限唯一; 点列有界}                                      & { 极限唯一; 点列有界} & Hausdorff空间中极限唯一 \\
{ }                             & { 点列极限线性运算法则}            & { 有}                                              & { 无}         & 无                \\
\multirow{-3}{*}{{ 点列的极限}}      & { 点列极限按分量收敛性质}           & { 有}                                              & { /}         & /                \\ \hline
{ }                             & { Cauchy收敛准则}            & { 有(利用上一条证明)}                                     & { 对应完备性}     & 无                \\
{ }                             & { 闭区间套定理}             & { {闭集套定理}}                                        & { 闭球套定理, 紧集套定理} & 紧集套定理            \\
{ }                             & { Bolzano-Weierstrass定理} & { 有}                                              & { 对应列紧性}     & 对应列紧性            \\ 
\multirow{-4}{*}{{ 实数完备性定理的推广}} & { Heine-Borel定理}         & { 有}                                              & { 对应紧致性}     & 对应紧致性            \\ \hline
{ }                             & { 函数极限的性质}             & { 极限唯一; 最终有界}                                              & { 极限唯一; 最终有界}         & /                \\ 
{ }                             & { Heine归结原理}             & { 有}                                              & { 有}         & /                \\ 
{ }                             & { 函数极限运算法则}              & { 有}                                              & { 有}         & /                \\
{ }                             & { 函数的Cauchy收敛准则}         & { 有}                                              & { 有}         & /                \\
{ }                             & { Cantor-Heine的一致连续性定理}  & { 有}                                              & { 如果定义域是紧集}  & /                \\
\multirow{-5}{*}{{ 函数/映射的极限}}   & Weierstrass最大值定理                             & 有                                                                     & 无                                & /  \\ \hline              
\end{tabular}
\caption{$\R ^n$, 度量空间, 拓扑空间有关收敛性和连续性的性质对比}
\end{table}

\section{$\R ^n$上的点列极限与连续映射}

在$\R ^n$中, 我们定义开球(邻域)$B(x_0,\varepsilon)$为$\{ x \in X:d(x,x_0)<\varepsilon \}$, 其中$d$是由范数$\| (x^1,\cdots ,x^n) \| = \sqrt{(x^1)^2+\cdots (x^n)^2}$引出的度量. 称点列$\{ x_k \}$是有界的, 如果存在$M>0$使得对任意$k$都有$x_k<M$. 

\subsection{点列的极限}

\begin{definition}{点列的极限}
	称点$l \in \R ^n$为点列$\{ x_k \}$的\textit{极限}(limit), 如果对任意的$l$的邻域$B(l,\varepsilon)$都存在$N$使得当$k \geq N$时$x_k \in B(l,\varepsilon)$. 记为$$l = \lim_{k \to \infty} x_k ~~ \text{或} ~~ x_k \to l,~k \to \infty$$且称$\{ x_k \}$\textit{收敛}(convergent)于$l$. 
\end{definition}

\begin{proposition}{点列极限的性质}
	(1) 收敛点列存在唯一的极限. \qquad (2) 收敛点列必有界. 
\end{proposition}

\begin{proposition}{点列极限线性运算法则}
	设$\R ^n$中收敛点列$\{ x_k \},\{ y_k \}$, 收敛的实数列$\{ l_n \}$, 则
	
	(1) $\lim_{k\to \infty} (x_k+y_k) = \lim_{k\to \infty} x_k + \lim_{k\to \infty} y_k$. 
	
	(2) $\lim_{k\to \infty} (l_kx_k) = \lim_{k\to \infty} l_k \lim_{k\to \infty} x_k$. 
\end{proposition}

由于$\R ^n$上没有序结构, 夹逼定理, 保序性等性质无法推广. 

\begin{lemma}{点列极限按分量收敛}
	设$\R ^n$中的点列$x_k = (x^1_k, \cdots ,x_k^n)$和点$l=(l^1,\cdots ,l^n)$, 则$$\lim_{k\to \infty} x_k = l \Longleftrightarrow \lim_{k\to \infty} x^i_k = l^i,~~i=1,\cdots ,n. $$
\end{lemma}
\begin{proof}
	利用下方不等式控制即可$$\max_{1 \leq i \leq n} |x^i_k-l^i| \leq \| x_k-l \| \leq \sum_{i=1}^{n} |x^i_k-l^i|. $$
\end{proof}

\subsection{实数完备性定理的推广}

\begin{definition}{Cauchy列}
	一个点列$\{ x_k \}$被称作\textit{Cauchy列}(Cauchy sequence),如果对于任意的$\varepsilon >0$都存在自然数$N$使得$\| x_k-x_p \|<\varepsilon$对$k,p>N$恒成立.
\end{definition}

\begin{theorem}{$\R ^n$中的Cauchy收敛准则}
	一个点列收敛当且仅当它是一个Cauchy列.
\end{theorem}
\begin{remark}
	这说明$\R ^n$是Banach空间. 
\end{remark}
\begin{proof}
	必要性显然. 充分性: 由于不等式$|x^i_k - x^i_p| \leq \| x_k - x_p \|(i=1,\cdots ,n)$, 可知$\{ x^1_k \},\cdots ,\{ x^n_k \}$都是Cauchy列, 由实数列的Cauchy收敛准则知它们都收敛, 从而$\lim_{k\to \infty} x_k = (\lim_{k\to \infty} x^1_k,\cdots , \lim_{k\to \infty} x^n_k)$. 
\end{proof}

我们定义集合$X \subseteq \R ^n$的直径如下: $$\diam X := \sup_{x,y \in X} \| x-y \|. $$

\begin{theorem}{闭集套定理}
	设$\R ^n$中的非空闭集列$\{ F_k \}$. 若$F_1 \supseteq \cdots \supseteq F_k \supseteq \cdots$且$\lim_{k\to \infty} \diam F_k=0$, 则存在唯一的$c \in \R ^n$使得$c \in \bigcap_{k\geq 1} F_k$. 
\end{theorem}
\begin{proof}
	存在性: 在每个$F_k$中取点$x_k$, 从而$\{ x_N,x_{N+1},\cdots \} \subseteq F_N$, 所以$\| x_{k}-x_p \| \leq \diam F_N \to 0, N\to \infty$对所有$k,p >N$成立. 进一步, $\{ x_k \}$是Cauchy列, 设其极限为$c$, 由于所有$F_k$都是闭集, 可知$c \in F_k,k\geq 1$. 
	
	唯一性: 假设存在$c_1,c_2 \in \bigcap_{k\geq 1} F_k$, 可知$\| c_1-c_2 \| \leq \diam F_k \to 0, k\to \infty$, 从而$c_1=c_2$. 
\end{proof}

\begin{theorem}{Bolzano-Weierstrass}
	$\R ^n$中的有界点列一定有收敛子列. 
\end{theorem}
\begin{proof}
	设点列$\{ x_k \}$有界, 则每个坐标组成的数列$\{ x^i_k \}$均有界. 归纳地操作: 在第$i+1$个数列中按照下标$m_{{i,k}}$选出有界数列, 再在其中选取收敛子列$\{ x^{i+1}_{m_{i+1,k}} \}$. 从而$\{ x_{m_{n,k}} \}$是$\{ x_k \}$的收敛子列. 
\end{proof}

\begin{theorem}{Heine-Borel}
	设$X \subseteq \R ^n$, 若$X$是有界闭集, 则任意$X$的开覆盖都存在有限子覆盖. 
\end{theorem}
\begin{proof}
	我们将证明留到拓扑不变量一节说明. 
\end{proof}

\subsection{多元函数的重极限}

对于一般定义的向量值函数$f:\R ^n \supseteq X \to \R ^n$, 我们可以将其分解为$f=(f^1,\cdots ,f^m)$, 因此只需研究值域为$\R$的函数, 称为多元函数. 

\begin{definition}{多元函数的重极限}
	设$f: \R ^n \supseteq X \to \R$, $\mathcal{B}$是$X$上的基. 称$l$为函数$f$\textit{在基$\mathcal{B}$上的极限}, 如果对于$l \in \R$的任何邻域$B(l,\varepsilon)$都存在$B \in \mathcal{B}$使得$f(B) \subseteq B(l,\varepsilon)$. 记作$$\lim_{\mathcal{B}}f(x)=l.$$
\end{definition}

\begin{proposition}{多元函数极限的性质}
	(1) 收敛的多元函数存在唯一的极限. \qquad (2) 在$\mathcal{B}$上收敛的多元函数在$\mathcal{B}$上最终有界. 
\end{proposition}

只要证明Heine归结原理, 剩下的函数极限性质都能自然推出. 

\begin{theorem}{多元函数的Heine归结原理}
	设$f: \R ^n \supseteq X \to \R$, $x_0$是$X$的一个极限点, 则$\lim_{x\to \x_0}f(x)=l$当且仅当对于任意收敛于$x_0$的点列$\{ x_k \} \in X$都有$\lim_{k\to \infty} f(x_k)=l$. 
\end{theorem}
\begin{proof}
	必要性由定义是显然的. 充分性: 用反证法. 假设$f$在$x_0$处的极限不为$l$, 则存在$B(l,\varepsilon)$使得对任意的正整数$k$, 存在$x_k \in B(x_0,1/k)$使得$f(x_k) \notin B(l,\varepsilon)$. 所有这样的$x_k$构成一个收敛于$x_0$的数列, 但不符合题目条件, 即得矛盾. 
\end{proof}

利用Heine归结原理, 容易说明$f(x,y)$分别对$x,y$连续不一定能得到$f$对$(x,y)$连续. (至于何时是一定能得到的, 我们会在下一小节给出)

\begin{example}
	设函数$f(x,y) = \begin{cases}
 \frac{xy}{x^2+y^2} &  (x,y) \neq (0,0) \\
 0 &  (x,y) = (0,0)
\end{cases}$. 显然$f$对两个变量分别连续, 说明$f$对$(x,y)$不连续. 
\end{example}
\begin{proof}
	考虑收敛到$(0,0)$的点列$\{ \frac{1}{k}(1,\lambda) \}$, 但是$f(\frac{1}{k}(1,\lambda)) \to \frac{\lambda}{1+\lambda ^2} \neq 0$, 说明$f$在$(0,0)$处不连续. 或者, 注意到对$f(a,a) \equiv \frac{1}{2}, a \neq 0$. 
\end{proof}

\begin{theorem}{多元函数极限的算术运算}
	设函数$f:\R ^n \supseteq X \to \R, g:X \to \R$, $\mathcal{B}$是$X$上的基. 记$\lim_{\mathcal{B}} f(x) = A, \lim_{\mathcal{B}} g(x) = B$. 
	
	a) 加减法. $\lim_{\mathcal{B}} (f\pm g)(x) = A\pm B.$
	
	b) 乘法. $\lim_{\mathcal{B}} (f\cdot g)(x) = A \cdot B.$
	
	c) 除法, 其中$B\neq 0$. $\lim_{\mathcal{B}} \ssb{\frac{f}{g}}(x) = \frac{A}{B}.$
\end{theorem}

\begin{theorem}{多元函数极限的Cauchy收敛准则}
	设函数$f:\R ^n \supseteq X \to \R$, $\mathcal{B}$是$X$上的基, 则$f$在$\mathcal{B}$上存在极限当且仅当对任意$\varepsilon >0$都存在$B \in \mathcal{B}$使得任意$x,y \in B$有$|f(x)-f(y)|<\varepsilon$. 
\end{theorem}

\begin{theorem}{多元函数复合的极限}
	设一元函数$f$和$n$元函数$g$, 若$\lim_{y\to y_0} f(y)=l, \lim_{x \to x_0}g(x) = y_0$且存在$r$使得$0 \notin g(B(x_0,r))$, 则$\lim_{x \to x_0} f(g(x)) = l$. 
\end{theorem}
\begin{remark}
	这个定理不用基的形式写是为了直观考虑, 实际上对任意基都是成立的. 
\end{remark}

\subsection{多元函数的累次极限}

相对于重极限, 多元函数的累次极限将每个变量逐个取极限而不是同时取极限. 容易发现, 我们只需要研究二元函数的累次极限就足够了. 

正如上一小节例题所述, 累次极限不一定等于重极限. 而且, 不同次序的累次极限也不一定相等. 一般地, 我们有如下的两个条件: 

\begin{proposition}{}
	设二元函数$f:\R ^2 \supseteq X \to \R$, $(x_0,y_0)$是$X$的一个极限点. 若$x \to x_0$时$f(x,y)$一致收敛, $y \to y_0$时$f(x,y)$逐点收敛, 则$$\lim_{y\to y_0} \lim_{x \to x_0} f(x,y) = \lim_{x\to x_0} \lim_{y \to y_0} f(x,y). $$
\end{proposition}
\begin{proof}
	这是Moore-Osgood定理的直接推论. 
\end{proof}

\begin{proposition}{}
	设二元函数$f:\R ^2 \supseteq X \to \R$, $(x_0,y_0)$是$X$的一个极限点. 若重极限$\lim_{(x,y) \to (x_0,y_0)} f(x,y)$存在, 且对所有$y \neq y_0$, $\lim_{x \to x_0} f(x,y)$存在, 那么$$\lim_{y\to y_0} \lim_{x \to x_0} f(x,y) = \lim_{(x,y) \to (x_0,y_0)} f(x,y).$$
\end{proposition}
\begin{remark}
	反过来就得到重极限不存在的充分条件: 若两种累次极限均存在且不相等, 则重极限一定不存在. 
\end{remark}
\begin{proof}
	由重极限存在可知, 对于任意$\varepsilon >0$, 存在$\delta >0$使得只要$|x-x_0|<\delta,|y-y_0|<\delta$就有$|f(x,y)-l|<\varepsilon$. 对于满足$|y-y_0|<\delta$的$y$, 记$\varphi (y) = \lim_{x \to x_0} f(x,y), y \in (y_0-\delta ,y_0) \cup (y_0,y_0 + \delta)$, 那么在上面的不等式中令$x \to x_0$就有$|\varphi (y) - l | \leq \varepsilon$, 说明$\lim_{y \to y_0} \varphi (y) = l$. 
\end{proof}

以上两条命题说明, 在研究累次极限换序问题时, 应该优先考虑重极限是否存在, 若存在则看单次极限是否逐点收敛, 若不存在则看单次极限是一致收敛还是逐点收敛. 

\begin{example}
	分别计算下列二元函数在$(x,y) \to (0,0)$时的累次极限和二重极限: $$1)~~f(x,y) = xy,\qquad 2)~~f(x,y) = (x+y)\sin \frac{1}{x} \sin \frac{1}{y},\qquad 3)~~f(x,y) = \begin{cases} x+y\sin \frac{1}{x} & x\neq 0 \\ 0 & x=0 \end{cases}, $$
	$$4)~~f(x,y) = \begin{cases} \frac{xy}{x^2+y^2} &  (x,y) \neq (0,0) \\ 0 &  (x,y) = (0,0) \end{cases},\qquad 5)~~f(x,y)=\frac{x-y}{x+y},\qquad 6)~~f(x,y)=\frac{x}{y}. $$
\end{example}

\subsection{多元函数的连续性}

\begin{definition}{多元函数的连续性}
	设$f:\R ^n \supseteq X \to \R$, 若$X \ni x_0$是$X$的一个极限点, 我们称$f$在$x_0$点处\textit{连续}(continuous), 如果$\lim_{X \ni x \to x_0} f(x) = f(x_0)$. 等价地有: 
	\begin{itemize}
		\item 对于任意$\varepsilon >0$, 存在$\delta >0$使得$f(B(x_0,\delta) - \{ x_0 \}) \subseteq B(f(x_0),\varepsilon)$; 
		\item 对任意的点列$\{ x_n \}$, 若$\lim_{n \to \infty} x_n = x_0$, 则$\lim_{n \to \infty} f(x_n) = f(x_0)$. 
	\end{itemize}
\end{definition}
\begin{remark}
	和一元函数连续性的定义类似, 有些书将定义扩大到了$X$的所有点, 而我们可以验证在第一种等价说法下$f$在孤立点处总是连续的. 
\end{remark}

多元函数连续性的基本性质可以由其极限性质直接得到, 请读者参考一元函数连续性章节, 这里不再一一罗列. 

\begin{example}
	设投影算子$\pi _i :\R ^n \to \R ,(x^1,\cdots ,x^n) \mapsto x^i,i=1,\cdots ,n$, 则$\pi _i$在$\R ^n$上连续. 
\end{example}
\begin{proof}
	利用不等式$|f(x)-f(y)|=|x^i - y^i| \leq \| x-y \|$控制即可. 
\end{proof}

类似于上一小节的Moore-Osgood定理, 我们有: 

\begin{proposition}{}
	设$f:\R ^2 \supseteq X \to \R$. 若$f$在$X$上对$x,y$分别连续, 且满足下列两个条件之一, 则$f$在$X$上连续. 
	
	1)~~$f$对某个变量一致连续;\qquad 2)~~$f$对某个变量单调. 
\end{proposition}
\begin{proof}
	(1) 不妨$f$对$x$一致连续. 对任意$\varepsilon _1>0$, 存在$\delta _1 >0$, 对任意$y$, 当$|x_1-x_2|<\delta _1$时$|f(x_1,y)-f(x_2,y)|<\varepsilon _1$; 对任意$\varepsilon _2>0$和任意$x$, 存在$\delta _2 >0$, 当$|y_1-y_2|<\delta _2$时$|f(x,y_1)-f(x,y_2)|<\varepsilon _2$. 所以, 当$\| (x,y)-(x_0,y_0) \|<\min \{ \delta _1,\delta _2 \}$时, $$|f(x,y)-f(x_0,y_0)| \leq |f(x,y)-f(x_0,y)| + |f(x_0,y)-f(x_0,y_0)| < \varepsilon _1 + \varepsilon _2. $$
	
	(2) 不妨$f$对$x$单调. 反过来说, 若能找到控制$|\Delta x| < \delta _0$, 有
	\begin{align*}
		|f(x+\Delta x,y+\Delta y)-f(x,y)| &\leq |f(x+\Delta x,y+\Delta y) - f(x+\Delta x,y)| + |f(x+\Delta x,y) - f(x,y)|  \\
		&\leq \max \{ |f(x \pm \delta _0 , y+\Delta y) - f(x \pm \delta _0 , y)| + |f(x \pm \delta _0 , y) - f(x,y)| \}.
	\end{align*}
	实际上, 重复(1)的过程, 我们令$\delta _0 = \min \{ \delta _1,\delta _2 \}$即可. 细节留给不放心的读者自行验证. 
\end{proof}

由于$\R ^n$是度量空间, 函数连续的拓扑表示自然是成立的. 

\subsection{有界闭集上连续函数的性质}

我们可以定义多元函数的一致连续. 

\begin{definition}{多元函数的一致连续性}
	设$f:\R ^n \supseteq X \to \R$, 称$f$在$X$上\textit{一致连续}(uniformly continuous), 如果对任意$\varepsilon >0$都存在$\delta >0$使得对任意$x,y \in X$, 只要$\| x-y \|<\delta$就有$|f(x)-f(y)|<\varepsilon$. 
\end{definition}

\begin{theorem}{Cantor-Heine的一致连续性定理}
	设$f:\R ^n \supseteq X \to \R$. 若$X$是有界闭集, 则$f$在$X$上连续. 
\end{theorem}
\begin{proof}
	重复一元函数对应定理中利用Bolzano-Weierstrass证明的部分即可. 
\end{proof}

\begin{theorem}{Weierstrass最大值定理}
	设$f:\R ^n \supseteq X \to \R$. 若$X$是有界闭集, 则$f$在$X$上有界且能取到极值. 
\end{theorem}
\begin{proof}
	重复一元函数对应定理中利用Bolzano-Weierstrass证明的部分即可. 
\end{proof}



\newpage
\section{度量空间上的点列极限与连续映射}

由于度量空间上定义了距离函数(重要! 这是一般的拓扑空间所不具有的), 我们可以很轻易地将$\R$中的一些概念和定理延拓过来. 

\newpage
\section{拓扑空间上的点列极限与连续映射}




\subsection{拓扑空间中点列的收敛性}

\subsection{拓扑空间之间的连续映射}

\begin{definition}{拓扑空间之间的连续映射}
	设拓扑空间$(X,\tau _X),(Y,\tau _Y)$, 称$f: X \to Y$是\textit{连续映射}(continuous mapping), 如果对任意$U \in \tau _Y$, $f^{-1}(U) \in \tau _X$. 
\end{definition}

\begin{proposition}{}
	设度量空间$(X,d_X),(Y,d_Y)$, 则$f:X \to Y$是连续映射当且仅当对任意$x,y \in X$和任意$\varepsilon >0$, 存在$\delta >0$使得只要$d_X(x,y)<\delta$就有$d_Y(f(x),f(y))<\varepsilon$. 
\end{proposition}

\newpage
\section{多元函数的极限和连续性}

\newpage
\section{拓扑不变量}

\subsection{紧集, 列紧集, 有界闭集}

回顾$\R$上闭区间$I$所拥有的良好性质: 连续函数$f$在$I$上有界且能取到极值, 而且$f$在$I$上一致连续. 在上述定理的证明中, 我们都用到了\textit{$I$中的数列存在子列使得其极限也在$I$中}这一事实, 即是说$I$是闭集. 另一方面, 在后者中, 由于利用有限覆盖定理也能完成证明, 我们会猜想是否闭集都能满足有限覆盖定理. 实际上, 这就是所谓紧性的一般化. 

本节的核心任务是下面的表格: 

\begin{table}[h]
	\centering
	\renewcommand\arraystretch{1.5}
	\begin{tabular}{|c|c|}
\hline
$\R^n$或有限维赋范向量空间 & 有界闭集$\stackrel{\textit{定理}1}{\Longleftrightarrow}$紧集$\Longleftrightarrow$列紧集 \\ \hline
度量空间             & 有界闭集$\Longleftarrow$紧集$\Longleftrightarrow$列紧集      \\ \hline
\end{tabular}
\end{table}

\begin{definition}{开覆盖, 紧集}
	设度量空间$(X,d)$, $S \subseteq X$. 
	\begin{itemize}
		\item 称集合族$\mathcal{U} = \{ U_{\alpha} \}_{\alpha \in A}$为$S$的一个\textit{开覆盖}(open cover), 如果$S \subseteq \bigcup_{\alpha \in A} U_{\alpha}$. 
		\item 承上述定义, 设指标集$A' \subseteq A$, 称$\mathcal{U}$的子集$\mathcal{U}'= \{ U_{\alpha} \}_{\alpha \in A'}$是$U$的\textit{子覆盖}(subcover), 如果$S \subseteq \bigcup_{\alpha \in A'} U_{\alpha}$. 
		\item 称$S$是\textit{紧集}(compact set), 如果$S$的任意开覆盖都存在一个有限子覆盖. 
	\end{itemize}
\end{definition}

\begin{lemma}{紧集在连续映射下被保持}
	设度量空间$(X,d_X),(Y,d_Y)$, $f:X \to Y$是连续映射, 若$K \subseteq X$是紧集, 则$f(K) \subseteq Y$也是紧集. 
\end{lemma}

先关注$\R$和$\R ^n$(以及有限维赋范向量空间, 只要选取一组基即可转化为$\R ^n$的问题)上的紧集. 

\begin{proposition}{($\R$上的)Lebesgue数引理}
	设$K \subseteq \R$是有界闭集, $\mathcal{U} = \{ U_{\alpha} \}_{\alpha \in A}$是$K$的开覆盖, 则存在$\mathcal{U}$的\textit{Lebesgue数}$\delta >0$, 使得对任意$x,y \in K$, 只要$|x-y|<\delta$, 就存在$\alpha \in A$使得$[x,y] \cap K \subseteq U_{\alpha}$. 
\end{proposition}
\begin{proof}
	XXX
\end{proof}

\begin{theorem}{Heine-Borel}
	设$K \subseteq \R$, 则$K$是紧集当且仅当$K$是有界闭集. 
\end{theorem}
\begin{proof}
	XXX
\end{proof}